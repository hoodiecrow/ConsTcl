\documentclass{report}
\usepackage{newpxmath}
\usepackage{graphicx}
\usepackage{listings}
\lstset{
  showstringspaces=false,
  language=tcl,
}
\renewcommand{\thesection}{\arabic{section}}
\title{ConsTcl}
\author{Peter Lewerin}
\date{\today}
\begin{document}
\maketitle
\tableofcontents
 
\section{Introduction}
\label{introduction}
\subsection{To run the software}
\label{to-run-the-software}

To run, source the file \textbf{constcl.tcl} (with \textbf{schemebase.lsp} in the directory) in a Tcl console (I use \textbf{tkcon}) and use the command \texttt{::constcl::repl} for a primitive command dialog. Source \textbf{all.tcl} to run the test suite.

\subsection{Background}
\label{background}

ConsTcl is a second try at a Lisp interpreter written in Tcl--the first one was Thtcl\footnote{See \texttt{https://github.com/hoodiecrow/thtcl}}, this time with a real Lisp-like type system. It steps over and back over the border between Tcl and Lisp a lot of times while working, and as a result is fairly slow.

\subsubsection{Benchmark}
\label{benchmark}

On my cheap computer, the following code takes 0.027 seconds to run.

\noindent\makebox[\linewidth]{\rule{\linewidth}{0.4pt}}
\begin{lstlisting}
namespace eval ::constcl {
    time {eval [parse "(fact 100)"]} 10
}
\end{lstlisting}
\noindent\makebox[\linewidth]{\rule{\linewidth}{0.4pt}}

Speed aside, it is an amusing piece of machinery. The types are implemented as TclOO classes, and evaluation is to a large extent applying Lisp methods to Tcl data.


It is limited. Quite a few standard procedures are missing. It doesn't come near to having call/cc or tail recursion. It doesn't have exact/inexact numbers, or most of the numerical tower. Error reporting is spotty, and there is no error recovery.

\section{Initial declarations}
\label{initial-declarations}

First, I need to create the namespace that will be used for most identifiers:

\noindent\makebox[\linewidth]{\rule{\linewidth}{0.4pt}}
\begin{lstlisting}
namespace eval ::constcl {}
\end{lstlisting}
\noindent\makebox[\linewidth]{\rule{\linewidth}{0.4pt}}

Next, some procedures that make my life as developer somewhat easier, but don't really matter to the interpreter (except the first one, \texttt{reg}, which registers built-in procedures in the definitions register). The other ones will show up a lot in the test cases.

\noindent\makebox[\linewidth]{\rule{\linewidth}{0.4pt}}
\begin{lstlisting}
proc ::reg {key args} {
        set val ::constcl::$key
    } else {
    }
    dict set ::constcl::defreg $key $val
}
 
proc ::pep {str} {
    ::constcl::write [
        ::constcl::eval [
            ::constcl::parse [
                ::constcl::IB new $str]]]
}
 
proc ::pp {str} {
    ::constcl::write [
        ::constcl::parse [
            ::constcl::IB new $str]]
}
 
proc ::prp {str} {
    set val [::constcl::parse [::constcl::IB new $str]]
    set op [::constcl::car $val]
    set args [::constcl::cdr $val]
    set env ::constcl::global_env
    while {[$op name] in {
            and case cond define del! for for/and
            for/list for/or let or pop! push! put!
            quasiquote unless when}} {
        ::constcl::expand-macro $env
    }
    set args [::constcl::resolve-local-defines $args]
    ::constcl::write $args
}
 
proc ::pxp {str} {
    set val [::constcl::parse [::constcl::IB new $str]]
    set op [::constcl::car $val]
    set args [::constcl::cdr $val]
    ::constcl::expand-macro ::constcl::global_env
    ::constcl::write [::constcl::cons $op $args]
}
 
proc ::constcl::check {cond msg} {
    ::if {[uplevel $cond] eq "#f"} {
        ::error [uplevel [::list subst $msg]]
    }
}
 
proc ::pn {} {
}
 
\end{lstlisting}
\noindent\makebox[\linewidth]{\rule{\linewidth}{0.4pt}}

This one is a little bit of both, a utility function that is also among the builtins in the library. It started out as a one-liner by Donal K. Fellows, but has grown a bit since then to suit my needs.

\noindent\makebox[\linewidth]{\rule{\linewidth}{0.4pt}}
\begin{lstlisting}
reg in-range ::constcl::in-range
 
#started out as DKF's code
proc ::constcl::in-range {args} {
    set step 1
    switch [llength $args] {
        1 {
            lassign $args e
            set end [$e value]
        }
        2 {
            lassign $args s e
            set start [$s value]
            set end [$e value]
        }
        3 {
            lassign $args s e t
            set start [$s value]
            set end [$e value]
            set step [$t value]
        }
    }
    set res $start
        lappend res $start
    }
    return [list {*}[lmap r $res {MkNumber $r}]]
}
\end{lstlisting}
\noindent\makebox[\linewidth]{\rule{\linewidth}{0.4pt}}

The \texttt{NIL} class has one object: the empty list called \texttt{\#NIL}. It is also base class for many other type classes.

\noindent\makebox[\linewidth]{\rule{\linewidth}{0.4pt}}
\begin{lstlisting}
catch { ::constcl::NIL destroy }
 
oo::class create ::constcl::NIL {
    constructor {} {}
    method bvalue {} {return #NIL}
    method car {} {::error "PAIR expected"}
    method cdr {} {::error "PAIR expected"}
    method set-car! {v} {::error "PAIR expected"}
    method set-cdr! {v} {::error "PAIR expected"}
    method numval {} {::error "Not a number"}
    method write {handle} {puts -nonewline $handle "()"}
    method display {} { puts -nonewline "()" }
    method show {} {format "()"}
}
\end{lstlisting}
\noindent\makebox[\linewidth]{\rule{\linewidth}{0.4pt}}

\textbf{null?}


The \texttt{null?} standard predicate recognizes the empty list. Predicates in ConsTcl return \#t or \#f for true or false, so some care is necessary when calling them from Tcl code.

\noindent\makebox[\linewidth]{\rule{\linewidth}{0.4pt}}
\begin{lstlisting}
reg null? ::constcl::null?
 
proc ::constcl::null? {obj} {
    ::if {$obj eq "#NIL"} {
        return #t
    } else {
        return #f
    }
}
\end{lstlisting}
\noindent\makebox[\linewidth]{\rule{\linewidth}{0.4pt}}

The \texttt{None} class serves but one purpose: to avoid printing a result after \texttt{define}.

\noindent\makebox[\linewidth]{\rule{\linewidth}{0.4pt}}
\begin{lstlisting}
catch { ::constcl::None destroy}
 
oo::class create ::constcl::None {}
\end{lstlisting}
\noindent\makebox[\linewidth]{\rule{\linewidth}{0.4pt}}

The \texttt{Dot} class is a helper class for the parser.

\noindent\makebox[\linewidth]{\rule{\linewidth}{0.4pt}}
\begin{lstlisting}
catch { ::constcl::Dot destroy }
 
oo::class create ::constcl::Dot {
    method mkconstant {} {}
    method write {handle} {puts -nonewline $handle "."}
    method display {} { puts -nonewline "." }
}
 
proc ::constcl::dot? {obj} {
    ::if {[info object isa typeof $obj Dot]} {
        return #t
    } elseif {[info object isa typeof [interp alias {} $obj] Dot]} {
        return #t
    } else {
        return #f
    }
}
\end{lstlisting}
\noindent\makebox[\linewidth]{\rule{\linewidth}{0.4pt}}

The \texttt{Unspecific} class is for unspecific things.

\noindent\makebox[\linewidth]{\rule{\linewidth}{0.4pt}}
\begin{lstlisting}
catch { ::constcl::Unspecific destroy }
 
oo::class create ::constcl::Unspecific {
    method mkconstant {} {}
}
\end{lstlisting}
\noindent\makebox[\linewidth]{\rule{\linewidth}{0.4pt}}

The \texttt{Undefined} class is for undefined things.

\noindent\makebox[\linewidth]{\rule{\linewidth}{0.4pt}}
\begin{lstlisting}
catch { ::constcl::Undefined destroy }
 
oo::class create ::constcl::Undefined {
    method mkconstant {} {}
    method write {} {puts -nonewline #<undefined>}
}
\end{lstlisting}
\noindent\makebox[\linewidth]{\rule{\linewidth}{0.4pt}}

The \texttt{EndOfFile} class is for end-of-file conditions.

\noindent\makebox[\linewidth]{\rule{\linewidth}{0.4pt}}
\begin{lstlisting}
catch { ::constcl::EndOfFile destroy }
 
oo::class create ::constcl::EndOfFile {
    method mkconstant {} {}
    method write {handle} {puts -nonewline #<eof>}
}
\end{lstlisting}
\noindent\makebox[\linewidth]{\rule{\linewidth}{0.4pt}}

\texttt{error} is used to signal an error, with \_msg\_ being a message string and the optional arguments being values to show after the message.

\begin{tabular}{ |l l| }
\hline
\multicolumn{2}{|l|}{error (public)} \\
\hline
msg & a message string \\
args & some expressions \\
\textit{Returns:} & -don't care- \\
\hline
\end{tabular}

\noindent\makebox[\linewidth]{\rule{\linewidth}{0.4pt}}
\begin{lstlisting}
reg error
 
proc ::constcl::error {msg args} {
    ::if {[llength $args]} {
        lappend msg "("
        foreach arg $args {
            ::if {$times} {
                ::append msg " "
            }
            ::append msg [$arg show]
            incr times
        }
        lappend msg ")"
    }
    ::error $msg
}
\end{lstlisting}
\noindent\makebox[\linewidth]{\rule{\linewidth}{0.4pt}}
\section{S9fES}
\label{s9fes}

I've begun porting parts of S9fES (\_Scheme 9 from Empty Space\_, by Nils M Holm) to fill out the blanks in e.g. I/O. It remains to be seen if it is successful.


I've already mixed this up with my own stuff.

\noindent\makebox[\linewidth]{\rule{\linewidth}{0.4pt}}
\begin{lstlisting}
proc ::constcl::new-atom {pa pd} {
    cons3 $pa $pd $::constcl::ATOM_TAG
}
\end{lstlisting}
\noindent\makebox[\linewidth]{\rule{\linewidth}{0.4pt}}
\noindent\makebox[\linewidth]{\rule{\linewidth}{0.4pt}}
\begin{lstlisting}
proc cons3 {pcar pcdr ptag} {
    # TODO counters
    set n [MkPair $pcar $pcdr]
    $n settag $ptag
    return $n
}
\end{lstlisting}
\noindent\makebox[\linewidth]{\rule{\linewidth}{0.4pt}}
\noindent\makebox[\linewidth]{\rule{\linewidth}{0.4pt}}
\begin{lstlisting}
proc ::constcl::xread {} {
    ::if {[$::constcl::InputPort handle] eq "#NIL"} {
        error "input port is not open"
    }
}
 
proc ::constcl::read_c_ci {} {
    return [tolower [::read [$::constcl::Input_port handle] 1]]
}
\end{lstlisting}
\noindent\makebox[\linewidth]{\rule{\linewidth}{0.4pt}}
\section{read}
\label{read}

\texttt{read} represents the interpreter's input facility. Currently input is faked with input strings.


\textbf{IB} class


A quick-and-dirty input simulator, using an input buffer object to hold characters to be read. The \texttt{fill} method fills the buffer and sets the first character in the peek position. The \texttt{advance} method consumes one character from the buffer. \texttt{first} peeks at the next character to be read. \texttt{skip-ws} advances past whitespace and comments. \texttt{unget} backs up one position and sets a given character in the peek position. The \texttt{find} method looks past whitespace and comments to find a given character. It returns Tcl truth if it is found. Or it gets the hose again.

\noindent\makebox[\linewidth]{\rule{\linewidth}{0.4pt}}
\begin{lstlisting}
catch { ::constcl::IB destroy }
 
oo::class create ::constcl::IB {
    variable peekc buffer
    constructor {str} {
        set peekc {}
        set buffer $str
        my advance
    }
    method advance {} {
        ::if {$buffer eq {}} {
            set peekc {}
        } else {
            set buffer [::string range $buffer 1 end]
        }
    }
    method first {} {
        return $peekc
    }
    method unget {char} {
        set buffer $peekc$buffer
        set peekc $char
    }
    method find {char} {
        ::if {[::string is space -strict $peekc]} {
                ::if {![::string is space -strict [::string index $buffer $cp]]} {
                    break
                }
            }
            return [expr {[::string index $buffer $cp] eq $char}]
        } else {
            return [expr {$peekc eq $char}]
        }
    }
    method skip-ws {} {
        while true {
            switch -regexp $peekc {
                {[[:space:]]} {
                    my advance
                }
                {;} {
                    while {$peekc ne "\n" && $peekc ne {}}  {
                        my advance
                    }
                }
                default {
                    return
                }
            }
        }
    }
}
 
\end{lstlisting}
\noindent\makebox[\linewidth]{\rule{\linewidth}{0.4pt}}

The parsing procedure translates an expression from external representation to internal representation. The external representation is a 'recipe' for an expression that expresses it in a unique way. For example, the external representation for a vector is a sharp sign (\#), a left parenthesis ((), the external representation for some values, and a right parenthesis ()). The parser takes in the input buffer character by character, matching each character against a fitting external representation. When done, it creates an object, which is the internal representation of an expression. The object can then be passed to the evaluator.

\subsection{parse}
\label{parse}

\textbf{parse}


Given a string, \texttt{parse} fills the input buffer. It then parses the input and produces an expression.


Example:

\noindent\makebox[\linewidth]{\rule{\linewidth}{0.4pt}}
\begin{lstlisting}
% ::constcl::parse "(+ 2 3)"
::oo::Obj491
\end{lstlisting}
\noindent\makebox[\linewidth]{\rule{\linewidth}{0.4pt}}

Here, \texttt{parse} parsed the external representation of a list with three elements, +, 2, and 3. It produced the expression that has the internal representation \texttt{::oo::Obj491}. We will later meet procedures like \texttt{eval}, which transforms an expression into a value, and \texttt{write}, which prints a printed representation of expressions and values. Putting them together: we can see

\noindent\makebox[\linewidth]{\rule{\linewidth}{0.4pt}}
\begin{lstlisting}
% ::constcl::write ::oo::Obj491
(+ 2 3)
% ::constcl::write [::constcl::eval ::oo::Obj491]
5
\end{lstlisting}
\noindent\makebox[\linewidth]{\rule{\linewidth}{0.4pt}}

Fortunately, we don't have to work at such a low level. We can use the \texttt{repl} instead:

\noindent\makebox[\linewidth]{\rule{\linewidth}{0.4pt}}
\begin{lstlisting}
ConsTcl> (+ 2 3)
5
\end{lstlisting}
\noindent\makebox[\linewidth]{\rule{\linewidth}{0.4pt}}

Then, parsing and evaluation and writing goes on in the background and the internal representations of expressions and values are hidden.


Anyway, here is how it really looks like. \texttt{::oo::Obj491} was just the head of the list.


![intreplist](/images/intreplist.png)

\begin{tabular}{ |l l| }
\hline
\multicolumn{2}{|l|}{parse (public)} \\
\hline
inp & a string or an input buffer \\
\textit{Returns:} & an expression \\
\hline
\end{tabular}

\noindent\makebox[\linewidth]{\rule{\linewidth}{0.4pt}}
\begin{lstlisting}
reg parse
 
proc ::constcl::parse {inp} {
    ::if {[info object isa object $inp]} {
        set ib $inp
    } else {
        set ib [IB new $inp]
    }
    return [parse-expression]
}
\end{lstlisting}
\noindent\makebox[\linewidth]{\rule{\linewidth}{0.4pt}}

\textbf{parse-expression}


The procedure \texttt{parse-expression} parses input by peeking at the first available character and delegating to one of the more detailed parsing procedures based on that, producing an expression of any kind.

\begin{tabular}{ |l l| }
\hline
\multicolumn{2}{|l|}{parse-expression (internal)} \\
\hline
\textit{Returns:} & an expression \\
\hline
\end{tabular}

\noindent\makebox[\linewidth]{\rule{\linewidth}{0.4pt}}
\begin{lstlisting}
proc ::constcl::parse-expression {} {
    upvar ib ib
    $ib skip-ws
    switch -regexp [$ib first] {
        {^$}          { return #NONE}
        {\"}          { return [parse-string-expression] }
        {\#}          { return [parse-sharp] }
        {\'}          { return [parse-quoted-expression] }
        {\(}          { return [parse-pair-expression ")"] }
        {\+} - {\-}   { return [parse-plus-minus] }
        {\,}          { return [parse-unquoted-expression] }
        {\.}          { $ib advance ; return [Dot new] }
        {\[}          { return [parse-pair-expression "\]"] }
        {\`}          { return [parse-quasiquoted-expression] }
        {\d}          { return [parse-number-expression] }
        {[[:graph:]]} { return [parse-identifier-expression] }
        default {
            ::error "unexpected character ([$ib first])"
        }
    }
}
\end{lstlisting}
\noindent\makebox[\linewidth]{\rule{\linewidth}{0.4pt}}

\textbf{parse-string-expression}


\texttt{parse-string-expression} parses input starting with a double quote and collects characters until it reaches another (unescaped) double quote. It then returns a string expression--a String (see page \pageref{strings}) object.

\begin{tabular}{ |l l| }
\hline
\multicolumn{2}{|l|}{parse-string-expression (internal)} \\
\hline
\textit{Returns:} & a string \\
\hline
\end{tabular}

\noindent\makebox[\linewidth]{\rule{\linewidth}{0.4pt}}
\begin{lstlisting}
proc ::constcl::parse-string-expression {} {
    upvar ib ib
    set str {}
    $ib advance
    while {[$ib first] ne "\"" && [$ib first] ne {}} {
        set c [$ib first]
        ::if {$c eq "\\"} {
            $ib advance
            ::append str [$ib first]
        } else {
            ::append str $c
        }
        $ib advance
    }
    ::if {[$ib first] ne "\""} {
        ::error "malformed string (no ending double quote)"
    }
    $ib advance
    $ib skip-ws
    set expr [MkString $str]
    $expr mkconstant
    return $expr
}
\end{lstlisting}
\noindent\makebox[\linewidth]{\rule{\linewidth}{0.4pt}}

\textbf{parse-sharp}


\texttt{parse-sharp} parses input starting with a sharp sign (\#) and produces the various kinds of expressions whose external representation begins with a sharp sign.

\begin{tabular}{ |l l| }
\hline
\multicolumn{2}{|l|}{parse-sharp (internal)} \\
\hline
\textit{Returns:} & a vector, boolean, or character value \\
\hline
\end{tabular}

\noindent\makebox[\linewidth]{\rule{\linewidth}{0.4pt}}
\begin{lstlisting}
proc ::constcl::parse-sharp {} {
    upvar ib ib
    $ib advance
    switch [$ib first] {
        (    { return [parse-vector-expression] }
        t    { $ib advance ; $ib skip-ws ; return #t }
        f    { $ib advance ; $ib skip-ws ; return #f }
        "\\" { return [parse-character-expression] }
        default {
            ::error "Illegal #-literal: #[$ib first]"
        }
    }
}
\end{lstlisting}
\noindent\makebox[\linewidth]{\rule{\linewidth}{0.4pt}}

\textbf{make-constant}


The \texttt{make-constant} helper procedure is called to set components of expressions to constants when read as a quoted literal.

\noindent\makebox[\linewidth]{\rule{\linewidth}{0.4pt}}
\begin{lstlisting}
proc ::constcl::make-constant {val} {
    ::if {[pair? $val] ne "#f"} {
        $val mkconstant
        make-constant [car $val]
        make-constant [cdr $val]
    } elseif {[null? $val] ne "#f"} {
        return #NIL
    } else {
        $val mkconstant
    }
}
\end{lstlisting}
\noindent\makebox[\linewidth]{\rule{\linewidth}{0.4pt}}

\textbf{parse-quoted-expression}


\texttt{parse-quoted-expression} parses input starting with a "'", and then parses an entire expression beyond that, returning it wrapped in a list with \texttt{quote}.

\begin{tabular}{ |l l| }
\hline
\multicolumn{2}{|l|}{parse-quoted-expression (internal)} \\
\hline
\textit{Returns:} & an expression wrapped in the quote symbol \\
\hline
\end{tabular}

\noindent\makebox[\linewidth]{\rule{\linewidth}{0.4pt}}
\begin{lstlisting}
proc ::constcl::parse-quoted-expression {} {
    upvar ib ib
    $ib advance
    set expr [parse-expression]
    $ib skip-ws
    make-constant $expr
    return [list #Q $expr]
}
\end{lstlisting}
\noindent\makebox[\linewidth]{\rule{\linewidth}{0.4pt}}

\textbf{parse-pair-expression}


The \texttt{parse-pair-expression} procedure parses input and produces a structure of Pair (see page \pageref{pairs-and-lists})s expression.

\begin{tabular}{ |l l| }
\hline
\multicolumn{2}{|l|}{parse-pair-expression (internal)} \\
\hline
char & the terminating paren or bracket \\
\textit{Returns:} & a structure of pair expressions \\
\hline
\end{tabular}

\noindent\makebox[\linewidth]{\rule{\linewidth}{0.4pt}}
\begin{lstlisting}
 
proc ::constcl::parse-pair {char} {
    upvar ib ib
    ::if {[$ib find $char]} {
        return #NIL
    }
    $ib skip-ws
    set a [parse-expression]
    $ib skip-ws
    set res $a
    set prev #NIL
    while {![$ib find $char]} {
        set x [parse-expression]
        $ib skip-ws
        ::if {[dot? $x] ne "#f"} {
            set prev [parse-expression]
            $ib skip-ws
        } else {
            lappend res $x
        }
        ::if {[llength $res] > 999} break
    }
    foreach r [lreverse $res] {
        set prev [cons $r $prev]
    }
    return $prev
}
 
proc ::constcl::parse-pair-expression {char} {
    upvar ib ib
    $ib advance
    $ib skip-ws
    set expr [parse-pair $char]
    $ib skip-ws
    ::if {[$ib first] ne $char} {
        ::if {$char eq ")"} {
            ::error "Missing right parenthesis (first=[$ib first])."
        } else {
            ::error "Missing right bracket (first=[$ib first])."
        }
    }
    $ib advance
    $ib skip-ws
    return $expr
}
\end{lstlisting}
\noindent\makebox[\linewidth]{\rule{\linewidth}{0.4pt}}

\textbf{parse-plus-minus}


\texttt{parse-plus-minus} reacts to a plus or minus in the input buffer, and either returns a \texttt{+} or \texttt{-} symbol, or a number.

\begin{tabular}{ |l l| }
\hline
\multicolumn{2}{|l|}{parse-plus-minus (internal)} \\
\hline
\textit{Returns:} & either the symbols + or - or a number \\
\hline
\end{tabular}

\noindent\makebox[\linewidth]{\rule{\linewidth}{0.4pt}}
\begin{lstlisting}
proc ::constcl::parse-plus-minus {} {
    upvar ib ib
    set c [$ib first]
    $ib advance
    ::if {[::string is digit -strict [$ib first]]} {
        $ib unget $c
        return [::constcl::parse-number-expression]
    } else {
        ::if {$c eq "+"} {
            $ib skip-ws
            return [MkSymbol "+"]
        } else {
            $ib skip-ws
            return [MkSymbol "-"]
        }
    }
}
\end{lstlisting}
\noindent\makebox[\linewidth]{\rule{\linewidth}{0.4pt}}

\textbf{parse-unquoted-expression}


\texttt{parse-unquoted-expression} parses input, producing an expression and returning it wrapped in \texttt{unquote}, or in \texttt{unquote-splicing} if an @-sign is present in the input stream.

\begin{tabular}{ |l l| }
\hline
\multicolumn{2}{|l|}{parse-unquoted-expression (internal)} \\
\hline
\textit{Returns:} & an expression wrapped in the unquote-splicing symbol \\
\hline
\end{tabular}

\noindent\makebox[\linewidth]{\rule{\linewidth}{0.4pt}}
\begin{lstlisting}
proc ::constcl::parse-unquoted-expression {} {
    upvar ib ib
    $ib advance
    set symbol "unquote"
    ::if {[$ib first] eq "@"} {
        set symbol "unquote-splicing"
        $ib advance
    }
    set expr [parse-expression]
    $ib skip-ws
    return [list [MkSymbol $symbol] $expr]
}
\end{lstlisting}
\noindent\makebox[\linewidth]{\rule{\linewidth}{0.4pt}}

\textbf{parse-quasiquoted-expression}


\texttt{parse-quasiquoted-expression} parses input, producing an expression and returning it wrapped in \texttt{quasiquote}.

\begin{tabular}{ |l l| }
\hline
\multicolumn{2}{|l|}{parse-quasiquoted-expression (internal)} \\
\hline
\textit{Returns:} & an expression wrapped in the quasiquote symbol \\
\hline
\end{tabular}

\noindent\makebox[\linewidth]{\rule{\linewidth}{0.4pt}}
\begin{lstlisting}
proc ::constcl::parse-quasiquoted-expression {} {
    upvar ib ib
    $ib advance
    set expr [parse-expression]
    $ib skip-ws
    make-constant $expr
    return [list [MkSymbol "quasiquote"] $expr]
}
\end{lstlisting}
\noindent\makebox[\linewidth]{\rule{\linewidth}{0.4pt}}

\textbf{interspace}


The \texttt{interspace} helper procedure recognizes whitespace or comments between value representations.

\noindent\makebox[\linewidth]{\rule{\linewidth}{0.4pt}}
\begin{lstlisting}
proc ::constcl::interspace {c} {
    # don't add #EOF: parse-* uses this one too
    ::if {$c eq {} || [::string is space -strict $c] || $c eq ";"} {
        return #t
    } else {
        return #f
    }
}
\end{lstlisting}
\noindent\makebox[\linewidth]{\rule{\linewidth}{0.4pt}}

\textbf{parse-number-expression}


\texttt{parse-number-expression} parses input, producing a number and returning a Number (see page \pageref{numbers}) object.

\begin{tabular}{ |l l| }
\hline
\multicolumn{2}{|l|}{parse-number-expression (internal)} \\
\hline
\textit{Returns:} & a number \\
\hline
\end{tabular}

\noindent\makebox[\linewidth]{\rule{\linewidth}{0.4pt}}
\begin{lstlisting}
proc ::constcl::parse-number-expression {} {
    upvar ib ib
    while {[interspace [$ib first]] ne "#t" && [$ib first] ni {) \]}} {
        ::append num [$ib first]
        $ib advance
    }
    $ib skip-ws
    check {::string is double -strict $num} {Invalid numeric constant $num}
    return [MkNumber $num]
}
\end{lstlisting}
\noindent\makebox[\linewidth]{\rule{\linewidth}{0.4pt}}

\textbf{parse-identifier-expression}


\texttt{parse-identifier-expression} parses input, producing an identifier expression and returning a Symbol (see page \pageref{symbols}) object.

\begin{tabular}{ |l l| }
\hline
\multicolumn{2}{|l|}{parse-identifier-expression (internal)} \\
\hline
\textit{Returns:} & a symbol \\
\hline
\end{tabular}

\noindent\makebox[\linewidth]{\rule{\linewidth}{0.4pt}}
\begin{lstlisting}
proc ::constcl::parse-identifier-expression {} {
    upvar ib ib
    while {[interspace [$ib first]] ne "#t" && [$ib first] ni {) \]}} {
        ::append name [$ib first]
        $ib advance
    }
    $ib skip-ws
    # idcheck throws error if invalid identifier
    return [MkSymbol [idcheck $name]]
}
\end{lstlisting}
\noindent\makebox[\linewidth]{\rule{\linewidth}{0.4pt}}

\textbf{character-check}


The \texttt{character-check} helper procedure compares a potential

\noindent\makebox[\linewidth]{\rule{\linewidth}{0.4pt}}
\begin{lstlisting}
proc ::constcl::character-check {name} {
    ::if {[regexp -nocase {^#\\([[:graph:]]|space|newline)$} $name]} {
        return #t
    } else {
        return #f
    }
}
\end{lstlisting}
\noindent\makebox[\linewidth]{\rule{\linewidth}{0.4pt}}

\textbf{parse-character-expression}


\texttt{parse-character-expression} parses input, producing a character and returning a Char (see page \pageref{characters}) object.

\begin{tabular}{ |l l| }
\hline
\multicolumn{2}{|l|}{parse-character-expression (internal)} \\
\hline
\textit{Returns:} & a character \\
\hline
\end{tabular}

\noindent\makebox[\linewidth]{\rule{\linewidth}{0.4pt}}
\begin{lstlisting}
proc ::constcl::parse-character-expression {} {
    upvar ib ib
    set name "#"
    while {[interspace [$ib first]] ne "#t" && [$ib first] ni {) ]}} {
        ::append name [$ib first]
        $ib advance
    }
    check {character-check $name} {Invalid character constant $name}
    $ib skip-ws
    return [MkChar $name]
}
\end{lstlisting}
\noindent\makebox[\linewidth]{\rule{\linewidth}{0.4pt}}

\textbf{parse-vector-expression}


\texttt{parse-vector-expression} parses input, producing a vector expression and returning a Vector (see page \pageref{vectors}) object.

\begin{tabular}{ |l l| }
\hline
\multicolumn{2}{|l|}{parse-vector-expression (internal)} \\
\hline
\textit{Returns:} & a vector \\
\hline
\end{tabular}

\noindent\makebox[\linewidth]{\rule{\linewidth}{0.4pt}}
\begin{lstlisting}
proc ::constcl::parse-vector-expression {} {
    upvar ib ib
    $ib advance
    $ib skip-ws
    set res {}
    while {[$ib first] ne {} && [$ib first] ne ")"} {
        lappend res [parse-expression]
        $ib skip-ws
    }
    set vec [MkVector $res]
    $vec mkconstant
    ::if {[$ib first] ne ")"} {
        ::error "Missing right parenthesis (first=[$ib first])."
    }
    $ib advance
    $ib skip-ws
    return $vec
}
\end{lstlisting}
\noindent\makebox[\linewidth]{\rule{\linewidth}{0.4pt}}
\subsection{read}
\label{read1}

\textbf{read}


The standard builtin \texttt{read} reads and parses input into a Lisp expression in a similar manner to how \texttt{parse} parses a string buffer.

\begin{tabular}{ |l l| }
\hline
\multicolumn{2}{|l|}{read (public)} \\
\hline
?port? & a port \\
\textit{Returns:} & an expression \\
\hline
\end{tabular}

\noindent\makebox[\linewidth]{\rule{\linewidth}{0.4pt}}
\begin{lstlisting}
reg read ::constcl::read
 
proc ::constcl::read {args} {
    set c {}
    set unget {}
    ::if {[llength $args]} {
        lassign $args port
    } else {
        set port $::constcl::Input_port
    }
    set oldport $::constcl::Input_port
    set ::constcl::Input_port $port
    set expr [read-expression]
    set ::constcl::Input_port $oldport
    set unget {}
    return $expr
}
\end{lstlisting}
\noindent\makebox[\linewidth]{\rule{\linewidth}{0.4pt}}

\textbf{read-expression}


The procedure \texttt{read-expression} parses input by reading the first available character and delegating to one of the more detailed reading procedures based on that, producing an expression of any kind. A Tcl character value can be passed to it, that character will be used first before reading from the input stream.

\begin{tabular}{ |l l| }
\hline
\multicolumn{2}{|l|}{read-expression (internal)} \\
\hline
?char? & a Tcl character \\
\textit{Returns:} & an expression \\
\hline
\end{tabular}

\noindent\makebox[\linewidth]{\rule{\linewidth}{0.4pt}}
\begin{lstlisting}
proc ::constcl::read-expression {args} {
    upvar c c unget unget
    ::if {[llength $args]} {
        lassign $args c
    } else {
        set c [readc]
    }
    read-eof $c
    ::if {[::string is space $c] || $c eq ";"} {
        skip-ws
        read-eof $c
    }
    switch -regexp $c {
        {^$}          { return #NONE}
        {\"}          { set n [read-string-expression]       ; read-eof $n; return $n }
        {\#}          { set n [read-sharp]                   ; read-eof $n; return $n }
        {\'}          { set n [read-quoted-expression]       ; read-eof $n; return $n }
        {\(}          { set n [read-pair-expression ")"]     ; read-eof $n; return $n }
        {\+} - {\-}   { set n [read-plus-minus $c]           ; read-eof $n; return $n }
        {\,}          { set n [read-unquoted-expression]     ; read-eof $n; return $n }
        {\.}          { set n [Dot new]; set c [readc]       ; read-eof $n; return $n }
        {\[}          { set n [read-pair-expression "\]"]    ; read-eof $n; return $n }
        {\`}          { set n [read-quasiquoted-expression]  ; read-eof $n; return $n }
        {\d}          { set n [read-number-expression $c]    ; read-eof $n; return $n }
        {[[:graph:]]} { set n [read-identifier-expression $c]; read-eof $n; return $n }
        default {
            read-eof $c
            ::error "unexpected character ($c)"
        }
    }
}
\end{lstlisting}
\noindent\makebox[\linewidth]{\rule{\linewidth}{0.4pt}}

\texttt{readc} reads one character either from the unget store or from the input stream. If the input stream is at end-of-file, an eof object is returned.

\begin{tabular}{ |l l| }
\hline
\multicolumn{2}{|l|}{readc (internal)} \\
\hline
\textit{Returns:} & a Tcl character or end of file \\
\hline
\end{tabular}

\noindent\makebox[\linewidth]{\rule{\linewidth}{0.4pt}}
\begin{lstlisting}
proc readc {} {
    upvar unget unget
    ::if {$unget ne {}} {
        set c $unget
        set unget {}
    } else {
        set c [::read [$::constcl::Input_port handle] 1]
        ::if {[eof [$::constcl::Input_port handle]]} {
            return #EOF
        }
    }
    return $c
}
\end{lstlisting}
\noindent\makebox[\linewidth]{\rule{\linewidth}{0.4pt}}

\texttt{read-find} reads ahead through whitespace to find a given character. Returns 1

\begin{tabular}{ |l l| }
\hline
\multicolumn{2}{|l|}{read-find (internal)} \\
\hline
char & a Tcl character \\
\textit{Returns:} & a Tcl truth value (1 or 0) \\
\hline
\end{tabular}

\noindent\makebox[\linewidth]{\rule{\linewidth}{0.4pt}}
\begin{lstlisting}
proc read-find {char} {
    upvar c c unget unget
    while {[::string is space -strict $c]} {
        set c [readc]
        read-eof $c
        set unget $c
    }
    return [expr {$c eq $char}]
}
\end{lstlisting}
\noindent\makebox[\linewidth]{\rule{\linewidth}{0.4pt}}

\texttt{skip-ws} skips whitespace and comments (the ; to end of line kind). Uses the shared \_c\_ character. It leaves the first character not to be skipped in \_c\_.

\begin{tabular}{ |l l| }
\hline
\multicolumn{2}{|l|}{skip-ws (internal)} \\
\hline
\textit{Returns:} & nothing \\
\hline
\end{tabular}

\noindent\makebox[\linewidth]{\rule{\linewidth}{0.4pt}}
\begin{lstlisting}
proc skip-ws {} {
    upvar c c unget unget
    while true {
        switch -regexp $c {
            {[[:space:]]} {
                set c [readc]
            }
            {;} {
                while {$c ne "\n" && $c ne "#EOF"}  {
                    set c [readc]
                }
            }
            default {
                return
            }
        }
    }
}
\end{lstlisting}
\noindent\makebox[\linewidth]{\rule{\linewidth}{0.4pt}}

\texttt{read-eof} checks a number of characters for possible end-of-file objects. If it finds one, it returns \_from its caller\_ with the EOF value.

\begin{tabular}{ |l l| }
\hline
\multicolumn{2}{|l|}{read-eof (internal)} \\
\hline
args & some characters \\
\hline
\end{tabular}

\noindent\makebox[\linewidth]{\rule{\linewidth}{0.4pt}}
\begin{lstlisting}
proc read-eof {args} {
    foreach val $args {
        ::if {$val eq "#EOF"} {
            return -level 1 -code return #EOF
        }
    }
}
\end{lstlisting}
\noindent\makebox[\linewidth]{\rule{\linewidth}{0.4pt}}

\textbf{read-string-expression}


\texttt{read-string-expression} parses input starting with a double quote and collects characters until it reaches another (unescaped) double quote. It then returns a string expression--an immutable String (see page \pageref{strings}) object.

\begin{tabular}{ |l l| }
\hline
\multicolumn{2}{|l|}{read-string-expression (internal)} \\
\hline
\textit{Returns:} & a string \\
\hline
\end{tabular}

\noindent\makebox[\linewidth]{\rule{\linewidth}{0.4pt}}
\begin{lstlisting}
proc ::constcl::read-string-expression {} {
    upvar c c unget unget
    set str {}
    set c [readc]
    read-eof $c
    while {$c ne "\"" && $c ne "#EOF"} {
        ::if {$c eq "\\"} {
            set c [readc]
        }
        ::append str $c
        set c [readc]
    }
    ::if {$c ne "\""} {
        error "malformed string (no ending double quote)"
    }
    set c [readc]
    set expr [MkString $str]
    $expr mkconstant
    return $expr
}
\end{lstlisting}
\noindent\makebox[\linewidth]{\rule{\linewidth}{0.4pt}}

\textbf{read-sharp}


\texttt{read-sharp} parses input starting with a sharp sign (\#) and produces the various kinds of expressions whose external representation begins with a sharp sign.

\begin{tabular}{ |l l| }
\hline
\multicolumn{2}{|l|}{read-sharp (internal)} \\
\hline
\textit{Returns:} & a vector, boolean, or character value \\
\hline
\end{tabular}

\noindent\makebox[\linewidth]{\rule{\linewidth}{0.4pt}}
\begin{lstlisting}
proc ::constcl::read-sharp {} {
    upvar c c unget unget
    set c [readc]
    read-eof $c
    switch $c {
        (    { set n [read-vector-expression]   ; set c [readc]; return $n }
        t    { set n #t                         ; set c [readc]; return $n }
        f    { set n #f                         ; set c [readc]; return $n }
        "\\" { set n [read-character-expression];                return $n }
        default {
            read-eof $c
            ::error "Illegal #-literal: #$c"
        }
    }
}
\end{lstlisting}
\noindent\makebox[\linewidth]{\rule{\linewidth}{0.4pt}}

\textbf{read-vector-expression}


\texttt{read-vector-expression} parses input, producing a vector expression and returning a Vector (see page \pageref{vectors}) object.

\begin{tabular}{ |l l| }
\hline
\multicolumn{2}{|l|}{read-vector-expression (internal)} \\
\hline
\textit{Returns:} & an expression \\
\hline
\end{tabular}

\noindent\makebox[\linewidth]{\rule{\linewidth}{0.4pt}}
\begin{lstlisting}
proc ::constcl::read-vector-expression {} {
    upvar c c unget unget
    set res {}
    set c [readc]
    while {$c ne "#EOF" && $c ne ")"} {
        lappend res [read-expression $c]
        skip-ws
        read-eof $c
    }
    set expr [MkVector $res]
    $expr mkconstant
    ::if {$c ne ")"} {
        ::error "Missing right parenthesis (first=$c)."
    }
    set c [readc]
    return $expr
}
\end{lstlisting}
\noindent\makebox[\linewidth]{\rule{\linewidth}{0.4pt}}

\textbf{read-character-expression}


\texttt{read-character-expression} parses input, producing a character and returning a Char (see page \pageref{characters}) object.

\begin{tabular}{ |l l| }
\hline
\multicolumn{2}{|l|}{read-character-expression (internal)} \\
\hline
\textit{Returns:} & a character \\
\hline
\end{tabular}

\noindent\makebox[\linewidth]{\rule{\linewidth}{0.4pt}}
\begin{lstlisting}
proc ::constcl::read-character-expression {} {
    upvar c c unget unget
    set name "#\\"
    set c [readc]
    read-eof $c
    while {[::string is alpha $c]} {
        ::append name $c
        set c [readc]
        read-eof $c
    }
    check {character-check $name} {Invalid character constant $name}
    set expr [MkChar $name]
    return $expr
}
\end{lstlisting}
\noindent\makebox[\linewidth]{\rule{\linewidth}{0.4pt}}

\textbf{read-quoted-expression}


\texttt{read-quoted-expression} parses input starting with a "'", and then parses an entire expression beyond that, returning it wrapped in a list with \texttt{quote}.

\begin{tabular}{ |l l| }
\hline
\multicolumn{2}{|l|}{read-quoted-expression (internal)} \\
\hline
\textit{Returns:} & an expression wrapped in the quote symbol \\
\hline
\end{tabular}

\noindent\makebox[\linewidth]{\rule{\linewidth}{0.4pt}}
\begin{lstlisting}
proc ::constcl::read-quoted-expression {} {
    upvar c c unget unget
    set expr [read-expression]
    read-eof $expr
    make-constant $expr
    return [list #Q $expr]
}
\end{lstlisting}
\noindent\makebox[\linewidth]{\rule{\linewidth}{0.4pt}}

\textbf{read-pair-expression}


The \texttt{read-pair-expression} procedure parses input and produces a structure of Pair (see page \pageref{pairs-and-lists})s expression.

\begin{tabular}{ |l l| }
\hline
\multicolumn{2}{|l|}{read-pair-expression (internal)} \\
\hline
char & the terminating paren or bracket \\
\textit{Returns:} & a structure of pair expressions \\
\hline
\end{tabular}

\noindent\makebox[\linewidth]{\rule{\linewidth}{0.4pt}}
\begin{lstlisting}
proc ::constcl::read-pair-expression {char} {
    upvar c c unget unget
    set expr [read-pair $char]
    skip-ws
    read-eof $c
    ::if {$c ne $char} {
        ::if {$char eq ")"} {
            ::error "Missing right parenthesis (first=$c)."
        } else {
            ::error "Missing right bracket (first=$c)."
        }
    } else {
        set unget {}
        set c [readc]
    }
    return $expr
}
 
proc ::constcl::read-pair {char} {
    upvar c c unget unget
    ::if {[read-find $char]} {
        # read right paren/brack
        set c [readc]
        return #NIL
    }
    set c [readc]
    read-eof $c
    set a [read-expression $c]
    set res $a
    skip-ws
    set prev #NIL
    while {![read-find $char]} {
        set x [read-expression $c]
        skip-ws
        read-eof $c
        ::if {[dot? $x] ne "#f"} {
            set prev [read-expression $c]
            skip-ws $c]
            read-eof $c
        } else {
            lappend res $x
        }
        ::if {[llength $res] > 999} break
    }
    # read right paren/brack
    foreach r [lreverse $res] {
        set prev [cons $r $prev]
    }
    return $prev
}
\end{lstlisting}
\noindent\makebox[\linewidth]{\rule{\linewidth}{0.4pt}}

\textbf{read-plus-minus}


\texttt{read-plus-minus} reacts to a plus or minus in the input stream, and either returns a \texttt{+} or \texttt{-} symbol, or a number.

\begin{tabular}{ |l l| }
\hline
\multicolumn{2}{|l|}{read-plus-minus (internal)} \\
\hline
\textit{Returns:} & either the symbols + or - or a number \\
\hline
\end{tabular}

\noindent\makebox[\linewidth]{\rule{\linewidth}{0.4pt}}
\begin{lstlisting}
proc ::constcl::read-plus-minus {char} {
    upvar c c unget unget
    set c [readc]
    read-eof $c
    ::if {[::string is digit -strict $c]} {
        set n [read-number-expression $c]
        ::if {$char eq "-"} {
            set n [- $n]
        }
        return $n
    } else {
        ::if {$char eq "+"} {
            return [MkSymbol "+"]
        } else {
            return [MkSymbol "-"]
        }
    }
}
\end{lstlisting}
\noindent\makebox[\linewidth]{\rule{\linewidth}{0.4pt}}

\textbf{read-number-expression}


\texttt{read-number-expression} parses input, producing a number and returning a Number (see page \pageref{numbers}) object.

\begin{tabular}{ |l l| }
\hline
\multicolumn{2}{|l|}{read-number-expression (internal)} \\
\hline
?char? & a Tcl character \\
\textit{Returns:} & a number \\
\hline
\end{tabular}

\noindent\makebox[\linewidth]{\rule{\linewidth}{0.4pt}}
\begin{lstlisting}
proc ::constcl::read-number-expression {args} {
    upvar c c unget unget
    ::if {[llength $args]} {
        lassign $args c
    } else {
        set c [readc]
    }
    read-eof $c
    while {[interspace $c] ne "#t" && $c ne "#EOF" && $c ni {) \]}} {
        ::append num $c
        set c [readc]
    }
    set unget $c
    check {::string is double -strict $num} {Invalid numeric constant $num}
    return [MkNumber $num]
}
\end{lstlisting}
\noindent\makebox[\linewidth]{\rule{\linewidth}{0.4pt}}

\textbf{read-unquoted-expression}


\texttt{read-unquoted-expression} parses input, producing an expression and returning it wrapped in \texttt{unquote}, or in \texttt{unquote-splicing} if an @-sign is present in the input stream.

\begin{tabular}{ |l l| }
\hline
\multicolumn{2}{|l|}{read-unquoted-expression (internal)} \\
\hline
\textit{Returns:} & an expression wrapped in the unquote-splicing symbol \\
\hline
\end{tabular}

\noindent\makebox[\linewidth]{\rule{\linewidth}{0.4pt}}
\begin{lstlisting}
proc ::constcl::read-unquoted-expression {} {
    upvar c c unget unget
    set c [readc]
    read-eof $c
    ::if {$c eq "@"} {
        set symbol "unquote-splicing"
        set expr [read-expression]
    } else {
        set symbol "unquote"
        set expr [read-expression $c]
    }
    read-eof $expr
    return [list [MkSymbol $symbol] $expr]
}
\end{lstlisting}
\noindent\makebox[\linewidth]{\rule{\linewidth}{0.4pt}}

\textbf{read-quasiquoted-expression}


\texttt{read-quasiquoted-expression} parses input, producing an expression and returning it wrapped in \texttt{quasiquote}.

\begin{tabular}{ |l l| }
\hline
\multicolumn{2}{|l|}{read-quasiquoted-expression (internal)} \\
\hline
\textit{Returns:} & an expression wrapped in the quasiquote symbol \\
\hline
\end{tabular}

\noindent\makebox[\linewidth]{\rule{\linewidth}{0.4pt}}
\begin{lstlisting}
proc ::constcl::read-quasiquoted-expression {} {
    upvar c c unget unget
    set expr [read-expression]
    skip-ws
    read-eof $expr
    make-constant $expr
    return [list [MkSymbol "quasiquote"] $expr]
}
\end{lstlisting}
\noindent\makebox[\linewidth]{\rule{\linewidth}{0.4pt}}

\textbf{read-identifier-expression}


\texttt{read-identifier-expression} parses input, producing an identifier expression and returning a Symbol (see page \pageref{symbols}) object.

\begin{tabular}{ |l l| }
\hline
\multicolumn{2}{|l|}{read-identifier-expression (internal)} \\
\hline
?char? & a Tcl character \\
\textit{Returns:} & a symbol \\
\hline
\end{tabular}

\noindent\makebox[\linewidth]{\rule{\linewidth}{0.4pt}}
\begin{lstlisting}
proc ::constcl::read-identifier-expression {args} {
    upvar c c unget unget
    ::if {[llength $args]} {
    } else {
        set c [readc]
    }
    read-eof $c
    set name {}
    while {[::string is graph -strict $c]} {
        ::if {$c eq "#EOF" || $c in {) \]}} {
            break
        }
        ::append name $c
        set c [readc]
    }
    ::if {$c ne "#EOF"} {
        set unget $c
    }
    # idcheck throws error if invalid identifier
    idcheck $name
    return [MkSymbol $name]
}
\end{lstlisting}
\noindent\makebox[\linewidth]{\rule{\linewidth}{0.4pt}}
\section{eval}
\label{eval}

The heart of the Lisp interpreter, \texttt{eval} takes a Lisp expression and processes it according to its form.

\begin{tabular}{|l l l|}
\hline
Syntactic form & Syntax & Example \\
\hline
Variable reference & variable & r => 10 \\
Constant literal & number or boolean, etc & 99 => 99 \\
Quotation & quote datum & (quote r) => r \\
Sequence & begin expression... & (begin (define r 10) (* r r)) => 100 \\
Conditional & if test conseq alt & (if (> 99 100) (* 2 2) (+ 2 4)) => 6 \\
Definition & define identifier expression & (define r 10) => \\
Assignment & set! variable expression & (set! r 20) => 20 \\
Procedure definition & lambda formals body & (lambda (r) (* r r)) => ::oo::Obj3601 \\
Procedure call & operator operand... & (+ 1 6) => 7 \\
\hline
\end{tabular}


\textbf{eval}


\texttt{eval}

\begin{enumerate}
\item  processes an \_expression\_ to get a \_value\_. The exact method depends on the form of expression, see above and below.
\item  does a simple form of macro expansion on \texttt{op} and \texttt{args} (the car and cdr of the expression) before processing them in the big \texttt{switch}. See the part about macros (see page \pageref{macros}) below.
\item  resolves local defines, acting on expressions of the form "(begin (define ..." when in a local environment. See the part about resolving local defines (see page \pageref{resolving-local-defines}).
\end{enumerate}
\begin{tabular}{ |l l| }
\hline
\multicolumn{2}{|l|}{eval (public)} \\
\hline
expr & an expression \\
env & an environment \\
\textit{Returns:} & a Lisp value \\
\hline
\end{tabular}

\noindent\makebox[\linewidth]{\rule{\linewidth}{0.4pt}}
\begin{lstlisting}
reg eval ::constcl::eval
 
proc ::constcl::eval {expr {env ::constcl::global_env}} {
    ::if {[symbol? $expr] ne "#f"} {
        lookup $expr $env
    } elseif {[null? $expr] ne "#f" || [atom? $expr] ne "#f"} {
        set expr
    } else {
        set op [car $expr]
        set args [cdr $expr]
        while {[$op name] in {
            and case cond define del! for for/and for/list for/or
            let or pop! push! put! quasiquote unless when}} {
                expand-macro $env
        }
        ::if {$env ne "::constcl::global_env" && [$op name] eq "begin" &&
            ([pair? [car $args]] ne "#f" && [[caar $args] name] eq "define")} {
            set expr [resolve-local-defines $args]
            set op [car $expr]
            set args [cdr $expr]
        }
        switch [$op name] {
            quote   { car $args }
            if      { ::if {[eval [car $args] $env] ne "#f"} \
                        {eval [cadr $args] $env} \
                        {eval [caddr $args] $env} }
            begin   { eprogn $args $env }
            define  { declare [car $args] [eval [cadr $args] $env] $env }
            set!    { update! [car $args] [eval [cadr $args] $env] $env }
            lambda  { make-function [car $args] [cdr $args] $env }
            default { invoke [eval $op $env] [eval-list $args $env] }
        }
    }
}
\end{lstlisting}
\noindent\makebox[\linewidth]{\rule{\linewidth}{0.4pt}}

\textbf{Variable reference}


A variable is an identifier (symbol) bound to a location in the environment. If an expression consists of the identifier it is evaluated to the value stored in that location. This is handled by the helper procedure \texttt{lookup}. It searches the environment chain for the identifier, and returns the value stored in the location it is bound to. It is an error to lookup an unbound symbol.


\textbf{lookup}

\begin{tabular}{ |l l| }
\hline
\multicolumn{2}{|l|}{lookup (internal)} \\
\hline
sym & a symbol \\
env & an environment \\
\textit{Returns:} & a Lisp value \\
\hline
\end{tabular}

\noindent\makebox[\linewidth]{\rule{\linewidth}{0.4pt}}
\begin{lstlisting}
proc ::constcl::lookup {sym env} {
    [$env find $sym] get $sym
}
\end{lstlisting}
\noindent\makebox[\linewidth]{\rule{\linewidth}{0.4pt}}

\textbf{Quotation}


According to the rules of Variable reference, a symbol evaluates to its stored value. Well, sometimes one wishes to use the symbol itself as a value. That is what quotation is for. \texttt{(quote x)} evaluates to the symbol x itself and not to any value that might be stored under it. This is so common that there is a shorthand notation for it: \texttt{'x} is interpreted as \texttt{(quote x)} by the Lisp reader.


\textbf{Conditional}


The conditional form \texttt{if} evaluates a Lisp list of three expressions. The first, the \_condition\_, is evaluated first. If it evaluates to anything other than \texttt{\#f} (false), the second expression (the \_consequent\_) is evaluated and the value returned. Otherwise, the third expression (the \_alternate\_) is evaluated and the value returned. One of the two latter expressions will be evaluated, and the other will remain unevaluated.


\textbf{if}

\begin{tabular}{ |l l| }
\hline
\multicolumn{2}{|l|}{if (internal)} \\
\hline
condition & an expression \\
consequent & an expression \\
alternate & an expression \\
\textit{Returns:} & a Lisp value \\
\hline
\end{tabular}

\noindent\makebox[\linewidth]{\rule{\linewidth}{0.4pt}}
\begin{lstlisting}
proc ::constcl::if {cond conseq altern} {
    ::if {[uplevel $cond] ne "#f"} {uplevel $conseq} {uplevel $altern}
}
\end{lstlisting}
\noindent\makebox[\linewidth]{\rule{\linewidth}{0.4pt}}

\textbf{Sequence}


When expressions are evaluated in sequence, the order is important for two reasons. If the expressions have any side effects, they happen in the same order of sequence. Also, if expressions are part of a pipeline of calculations, then they need to be processed in the order of that pipeline. The \texttt{eprogn} helper procedure takes a Lisp list of expressions and evaluates them in sequence, returning the value of the last one.


\textbf{eprogn}

\begin{tabular}{ |l l| }
\hline
\multicolumn{2}{|l|}{eprogn (internal)} \\
\hline
exps & a Lisp list of expressions \\
env & an environment \\
\textit{Returns:} & a Lisp value \\
\hline
\end{tabular}

\noindent\makebox[\linewidth]{\rule{\linewidth}{0.4pt}}
\begin{lstlisting}
proc ::constcl::eprogn {exps env} {
    if {pair? $exps} {
        if {pair? [cdr $exps]} {
            eval [car $exps] $env
            return [eprogn [cdr $exps] $env]
        } {
            return [eval [car $exps] $env]
        }
    } {
        return #NIL
    }
}
\end{lstlisting}
\noindent\makebox[\linewidth]{\rule{\linewidth}{0.4pt}}

\textbf{Definition}


We've already seen the relationship between symbols and values. A symbol is bound to a value (or rather to the location the value is in), creating a variable, through definition. The \texttt{declare} helper procedure adds a variable to the current environment. It first checks that the symbol name is a valid identifier, then it updates the environment with the new binding.


\textbf{declare}

\begin{tabular}{ |l l| }
\hline
\multicolumn{2}{|l|}{declare (internal)} \\
\hline
sym & a symbol \\
val & a Lisp value \\
env & an environment \\
\textit{Returns:} & nothing \\
\hline
\end{tabular}

\noindent\makebox[\linewidth]{\rule{\linewidth}{0.4pt}}
\begin{lstlisting}
proc ::constcl::declare {sym val env} {
    varcheck [idcheck [$sym name]]
    $env set $sym $val
    return #NONE
}
\end{lstlisting}
\noindent\makebox[\linewidth]{\rule{\linewidth}{0.4pt}}

\textbf{Assignment}


Once a variable has been created, the value at the location it is bound to can be changed (hence the name "variable", something that can be modified). The process is called assignment. The \texttt{update!} helper does assignment: it modifies an existing variable that is bound somewhere in the environment chain. It finds the variable's environment and updates the binding. It returns the value, so calls to \texttt{set!} can be chained: \texttt{(set! foo (set! bar 99))} sets both variables to 99.


\textbf{update!}

\begin{tabular}{ |l l| }
\hline
\multicolumn{2}{|l|}{update! (internal)} \\
\hline
var & a bound symbol \\
val & a Lisp value \\
env & an environment \\
\textit{Returns:} & a Lisp value \\
\hline
\end{tabular}

\noindent\makebox[\linewidth]{\rule{\linewidth}{0.4pt}}
\begin{lstlisting}
proc ::constcl::update! {var val env} {
    [$env find $var] set $var $val
    set val
}
\end{lstlisting}
\noindent\makebox[\linewidth]{\rule{\linewidth}{0.4pt}}

\textbf{Procedure definition}


In Lisp, procedures are values just like numbers or characters. They can be defined as the value of a symbol, passed to other procedures, and returned from procedures. One diffence from most values is that procedures need to be defined. Two questions must answered: what is the procedure meant to do? The code that does that will form the body of the procedure. Also, what, if any, items of data will have to be provided to the procedure to make it possible to calculate its result?


As an example, imagine that we want to have a procedure that calculates the square (\texttt{x · x}) of a given number. In Lisp, expressions are written with the operator first and then the operands: \texttt{(* x x)}. That is the body of the procedure. Now, what data will we have to provide to the procedure to make it work? A value stored in the variable \texttt{x} will do. It's only a single variable, but by custom we need to put it in a list: \texttt{(x)}. The operator that defines procedures is called \texttt{lambda}, and we define the function with \texttt{(lambda (x) (* x x))}.


One more step is needed before we can use the procedure. It must have a name. We could define it like this: \texttt{(define square (lambda (x) (* x x)))} but there is actually a shortcut notation for it: \texttt{(define (square x) (* x x))}.


Now, \texttt{square} is pretty tame. How about the \texttt{hypotenuse} procedure? \texttt{(define (hypotenuse a b) (sqrt (+ (square a) (square b))))}. It calculates the square root of the sum of two squares.


Under the hood, the helper \texttt{make-function} makes a Procedure (see page \pageref{control}) object. First it needs to convert the Lisp list \texttt{body}. It is packed inside a \texttt{begin} if it has more than one expression, and taken out of its list if not. The Lisp list \texttt{formals} is passed on as is.


A Scheme formals list is either:

\begin{itemize}
\item An *empty list*, \texttt{()}, meaning that no arguments are accepted,
\item A *proper list*, \texttt{(a b c)}, meaning it accepts three arguments, one in each symbol,
\item A *symbol*, \texttt{a}, meaning that all arguments go into \texttt{a}, or
\item A *dotted list*, \texttt{(a b . c)}, meaning that two arguments go into \texttt{a} and \texttt{b}, and the rest into \texttt{c}.
\end{itemize}

\textbf{make-function}

\begin{tabular}{ |l l| }
\hline
\multicolumn{2}{|l|}{make-function (internal)} \\
\hline
formals & a Scheme formals list \\
body & a Lisp list of expressions \\
env & an environment \\
\textit{Returns:} & a procedure \\
\hline
\end{tabular}

\noindent\makebox[\linewidth]{\rule{\linewidth}{0.4pt}}
\begin{lstlisting}
proc ::constcl::make-function {formals body env} {
    ::if {[[length $body] value] > 1} {
        set body [cons #B $body]
    } else {
        set body [car $body]
    }
    return [MkProcedure $formals $body $env]
}
\end{lstlisting}
\noindent\makebox[\linewidth]{\rule{\linewidth}{0.4pt}}

\textbf{Procedure call}


Once we have procedures, we can call them to have their calculations performed and yield results. The procedure name is put in the operator position at the front of a list, and the operands follow in the rest of the list. Our \texttt{square} procedure would be called for instance like this: \texttt{(square 11)}, and it will return 121.


\texttt{invoke} arranges for a procedure to be called with each of the values in the \_argument list\_ (the list of operands). It checks if pr really is a procedure, and determines whether to call pr as an object or as a Tcl command.


\textbf{invoke}

\begin{tabular}{ |l l| }
\hline
\multicolumn{2}{|l|}{invoke (internal)} \\
\hline
pr & a procedure \\
vals & a Lisp list of Lisp values \\
\textit{Returns:} & what pr returns \\
\hline
\end{tabular}

\noindent\makebox[\linewidth]{\rule{\linewidth}{0.4pt}}
\begin{lstlisting}
proc ::constcl::invoke {pr vals} {
    check {procedure? $pr} {PROCEDURE expected\n([$pr show] val ...)}
    ::if {[info object isa object $pr]} {
        $pr call {*}[splitlist $vals]
    } else {
        $pr {*}[splitlist $vals]
    }
}
\end{lstlisting}
\noindent\makebox[\linewidth]{\rule{\linewidth}{0.4pt}}

\textbf{splitlist}


\texttt{splitlist} converts a Lisp list to a Tcl list with Lisp objects.

\begin{tabular}{ |l l| }
\hline
\multicolumn{2}{|l|}{splitlist (internal)} \\
\hline
vals & a Lisp list of Lisp values \\
\textit{Returns:} & a Tcl list of Lisp values \\
\hline
\end{tabular}

\noindent\makebox[\linewidth]{\rule{\linewidth}{0.4pt}}
\begin{lstlisting}
proc ::constcl::splitlist {vals} {
    set result {}
    while {[pair? $vals] ne "#f"} {
        lappend result [car $vals]
        set vals [cdr $vals]
    }
    return $result
}
\end{lstlisting}
\noindent\makebox[\linewidth]{\rule{\linewidth}{0.4pt}}

\textbf{eval-list}


\texttt{eval-list} successively evaluates the elements of a Lisp list and returns the results as a Lisp list.

\begin{tabular}{ |l l| }
\hline
\multicolumn{2}{|l|}{eval-list (internal)} \\
\hline
exps & a Lisp list of expressions \\
env & an environment \\
\textit{Returns:} & a Lisp list of Lisp values \\
\hline
\end{tabular}

\noindent\makebox[\linewidth]{\rule{\linewidth}{0.4pt}}
\begin{lstlisting}
proc ::constcl::eval-list {exps env} {
    ::if {[pair? $exps] ne "#f"} {
        return [cons [eval [car $exps] $env] [eval-list [cdr $exps] $env]]
    } {
        return #NIL
    }
}
\end{lstlisting}
\noindent\makebox[\linewidth]{\rule{\linewidth}{0.4pt}}
\subsection{Macros}
\label{macros}

\textbf{expand-macro}


Macros that rewrite expressions into other, more concrete expressions is one of Lisp's strong points. This interpreter does macro expansion, but the user can't define new macros--the ones available are hardcoded in the code below.


\texttt{expand-macro} only takes the environment as a parameter, but internally it uses variable sharing to get the expression it is to process. It shares the variables \texttt{op} and \texttt{args} with its caller, \texttt{eval}. \texttt{op} is used to delegate to the correct expansion procedure, and the value of \texttt{args} is passed to the expansion procedures. In the end, the expanded expression is passed back to \texttt{eval} by assigning to \texttt{op} and \texttt{args}.

\begin{tabular}{ |l l| }
\hline
\multicolumn{2}{|l|}{expand-macro (internal)} \\
\hline
env & an environment \\
\textit{Returns:} & the empty list \\
\hline
\end{tabular}

\noindent\makebox[\linewidth]{\rule{\linewidth}{0.4pt}}
\begin{lstlisting}
proc ::constcl::expand-macro {env} {
    upvar op op args args
    ::if {[$op name] eq "define" && ([pair? [car $args]] eq "#f" || [[caar $args] name] eq "lambda")} {
        return -code break
    }
    switch [$op name] {
        and {
            set expr [expand-and $args $env]
        }
        case {
            set expr [expand-case [car $args] [cdr $args]]
        }
        cond {
            set expr [expand-cond $args]
        }
        define {
            set expr [expand-define $args $env]
        }
        del! {
            set expr [expand-del! $args $env]
        }
        for {
            set expr [expand-for $args $env]
        }
        for/and {
            set expr [expand-for/and $args $env]
        }
        for/list {
            set expr [expand-for/list $args $env]
        }
        for/or {
            set expr [expand-for/or $args $env]
        }
        let {
            set expr [expand-let $args $env]
        }
        or {
            set expr [expand-or $args $env]
        }
        pop! {
            set expr [expand-pop! $args $env]
        }
        push! {
            set expr [expand-push! $args $env]
        }
        put! {
            set expr [expand-put! $args $env]
        }
        quasiquote {
            set expr [expand-quasiquote $args $env]
        }
        unless {
            set expr [expand-unless $args $env]
        }
        when {
            set expr [expand-when $args $env]
        }
    }
    set op [car $expr]
    set args [cdr $expr]
    return #NIL
}
\end{lstlisting}
\noindent\makebox[\linewidth]{\rule{\linewidth}{0.4pt}}

\textbf{expand-and}


\texttt{expand-and} expands the \texttt{and} macro. It returns a \texttt{begin}-expression if the macro

\begin{tabular}{ |l l| }
\hline
\multicolumn{2}{|l|}{expand-and (internal)} \\
\hline
tail & an expression tail \\
env & an environment \\
\textit{Returns:} & an expression \\
\hline
\end{tabular}

\noindent\makebox[\linewidth]{\rule{\linewidth}{0.4pt}}
\begin{lstlisting}
proc ::constcl::expand-and {tail env} {
        return [list #B #t]
    } {
        if {eq? [length $tail] #1} {
            return [cons #B $tail]
        } {
            return [do-and $tail #NIL $env]
        }
    }
}
\end{lstlisting}
\noindent\makebox[\linewidth]{\rule{\linewidth}{0.4pt}}
\begin{tabular}{ |l l| }
\hline
\multicolumn{2}{|l|}{do-and (internal)} \\
\hline
tail & an expression tail \\
prev & an expression \\
env & an environment \\
\textit{Returns:} & an expression \\
\hline
\end{tabular}

\noindent\makebox[\linewidth]{\rule{\linewidth}{0.4pt}}
\begin{lstlisting}
proc ::constcl::do-and {tail prev env} {
    set env [::constcl::Environment new #NIL {} $env]
        return $prev
    } {
        $env setstr "first" [car $tail]
        $env setstr "rest" [do-and [cdr $tail] [car $tail] $env]
        set qq "`(if ,first ,rest #f)"
        return [expand-quasiquote [cdr [parse $qq]] $env]
    }
}
\end{lstlisting}
\noindent\makebox[\linewidth]{\rule{\linewidth}{0.4pt}}

\textbf{expand-case}


The \texttt{case} macro is expanded by \texttt{expand-case}. It returns \texttt{'()} if there are no clauses (left), and nested \texttt{if} constructs if there are some.

\begin{tabular}{ |l l| }
\hline
\multicolumn{2}{|l|}{expand-case (internal)} \\
\hline
keyexpr & an expression \\
clauses & a Lisp list of Lisp values \\
\textit{Returns:} & an expression \\
\hline
\end{tabular}

\noindent\makebox[\linewidth]{\rule{\linewidth}{0.4pt}}
\begin{lstlisting}
proc ::constcl::expand-case {keyexpr clauses} {
        return [list #Q #NIL]
    } else {
        set keyl [caar $clauses]
        set body [cdar $clauses]
        set keyl [list [MkSymbol "memv"] $keyexpr [list #Q $keyl]]
        ::if {[eq? [length $clauses] #1] ne "#f"} {
            ::if {[eq? [caar $clauses] [MkSymbol "else"]] ne "#f"} {
                set keyl #t
            }
        }
        return [list #I $keyl [cons #B $body] [expand-case $keyexpr [cdr $clauses]]]
    }
}
\end{lstlisting}
\noindent\makebox[\linewidth]{\rule{\linewidth}{0.4pt}}

\textbf{expand-cond}


The \texttt{cond} macro is expanded by \texttt{expand-cond}. It returns \texttt{'()} if there are no clauses (left), and nested \texttt{if} constructs if there are some.

\begin{tabular}{ |l l| }
\hline
\multicolumn{2}{|l|}{expand-cond (internal)} \\
\hline
clauses & a Lisp list of Lisp values \\
\textit{Returns:} & an expression \\
\hline
\end{tabular}

\noindent\makebox[\linewidth]{\rule{\linewidth}{0.4pt}}
\begin{lstlisting}
proc ::constcl::expand-cond {clauses} {
        return [list #Q #NIL]
    } else {
        set pred [caar $clauses]
        set body [cdar $clauses]
        ::if {[symbol? [car $body]] ne "#f" && [[car $body] name] eq "=>"} {
            set body [cddar $clauses]
        }
        ::if {[eq? [length $clauses] #1] ne "#f"} {
            ::if {[eq? $pred [MkSymbol "else"]] ne "#f"} {
                set pred #t
            }
        }
        ::if {[null? $body] ne "#f"} {set body $pred}
        return [list #I $pred [cons #B $body] [expand-cond [cdr $clauses]]]
    }
}
\end{lstlisting}
\noindent\makebox[\linewidth]{\rule{\linewidth}{0.4pt}}

\textbf{expand-define}


\texttt{define} has two variants, one of which requires some rewriting. It's the one with an implied \texttt{lambda} call, the one that defines a procedure.


(\textbf{define} (\_symbol\_ \_formals\_) \_body\_)


is transformed into


(\textbf{define} \_symbol\_ (\textbf{lambda} \_formals\_ \_body\_))


which conforms better to \texttt{eval}'s standard of (\textbf{define} \_symbol\_ \_value\_).

\begin{tabular}{ |l l| }
\hline
\multicolumn{2}{|l|}{expand-define (internal)} \\
\hline
tail & an expression tail \\
env & an environment \\
\textit{Returns:} & an expression \\
\hline
\end{tabular}

\noindent\makebox[\linewidth]{\rule{\linewidth}{0.4pt}}
\begin{lstlisting}
proc ::constcl::expand-define {tail env} {
    set env [::constcl::Environment new #NIL {} $env]
    $env setstr "tail" $tail
    set qq "`(define ,(caar tail) (lambda ,(cdar tail) ,@(cdr tail)))"
    return [expand-quasiquote [cdr [parse $qq]] $env]
}
\end{lstlisting}
\noindent\makebox[\linewidth]{\rule{\linewidth}{0.4pt}}

\textbf{expand-del!}


The macro \texttt{del!} updates a property list. It removes a key-value pair if the key is present, or leaves the list untouched if it isn't.

\begin{tabular}{ |l l| }
\hline
\multicolumn{2}{|l|}{expand-del! (internal)} \\
\hline
tail & an expression tail \\
env & an environment \\
\textit{Returns:} & an expression \\
\hline
\end{tabular}

\noindent\makebox[\linewidth]{\rule{\linewidth}{0.4pt}}
\begin{lstlisting}
proc ::constcl::expand-del! {tail env} {
    set env [::constcl::Environment new #NIL {} $env]
    $env setstr "listname" [car $tail]
    ::if {[null? [cdr $tail]] ne "#f"} {::error "too few arguments, 2 expected, got 1"}
    $env setstr "key" [cadr $tail]
    set qq "`(set! ,listname (delete! ,listname ,key))"
    return [expand-quasiquote [cdr [parse $qq]] $env]
}
\end{lstlisting}
\noindent\makebox[\linewidth]{\rule{\linewidth}{0.4pt}}

\textbf{expand-for}


The \texttt{expand-for} procedure expands the \texttt{for} macro. It returns a \texttt{begin} construct containing the iterations of each clause (multiple clauses weren't implemented, but I brought up my strongest brain cells and they did it).

\begin{tabular}{ |l l| }
\hline
\multicolumn{2}{|l|}{for-seq (internal)} \\
\hline
seq & a Lisp value \\
env & an environment \\
\textit{Returns:} & a Tcl list of Lisp values \\
\hline
\end{tabular}

\noindent\makebox[\linewidth]{\rule{\linewidth}{0.4pt}}
\begin{lstlisting}
proc ::constcl::for-seq {seq env} {
    ::if {[number? $seq] ne "#f"} {
        set seq [in-range $seq]
    } else {
        set seq [eval $seq $env]
    }
    # make it a Tcl list, one way or another
    ::if {[list? $seq] ne "#f"} {
        set seq [splitlist $seq]
    } elseif {[string? $seq] ne "#f"} { 
        set seq [lmap c [split [$seq value] {}] {MkChar #\\$c}]
    } elseif {[vector? $seq] ne "#f"} {
        set seq [$seq value]
    }
}
\end{lstlisting}
\noindent\makebox[\linewidth]{\rule{\linewidth}{0.4pt}}
\begin{tabular}{ |l l| }
\hline
\multicolumn{2}{|l|}{do-for (internal)} \\
\hline
tail & an expression tail \\
env & an environment \\
\textit{Returns:} & a Tcl list of expressions \\
\hline
\end{tabular}

\noindent\makebox[\linewidth]{\rule{\linewidth}{0.4pt}}
\begin{lstlisting}
proc ::constcl::do-for {tail env} {
    # make clauses a Tcl list
    set clauses [splitlist [car $tail]]
    set body [cdr $tail]
    set ids {}
    set seqs {}
        set clause [lindex $clauses $i]
        # insert the first part of the clause in the ids structure
        lset ids $i [car $clause]
        # run the second part of the clause through for-seq and insert in seqs
        lset seqs $i [for-seq [cadr $clause] $env]
    }
    set res {}
        # for each iteration of the sequences
        set x {}
            # for each clause
            # list append to x the Lisp list of the id and the iteration
            lappend x [list [lindex $ids $clause] [lindex $seqs $clause $item]]
        }
        # list append to res a let expression with the ids and iterations and the body
        lappend res [list #L [list {*}$x] {*}[splitlist $body]]
    }
    return $res
}
\end{lstlisting}
\noindent\makebox[\linewidth]{\rule{\linewidth}{0.4pt}}
\begin{tabular}{ |l l| }
\hline
\multicolumn{2}{|l|}{expand-for (internal)} \\
\hline
tail & an expression tail \\
env & an environment \\
\textit{Returns:} & an expression \\
\hline
\end{tabular}

\noindent\makebox[\linewidth]{\rule{\linewidth}{0.4pt}}
\begin{lstlisting}
proc ::constcl::expand-for {tail env} {
    set res [do-for $tail $env]
    lappend res [list #Q #NIL]
    return [list #B {*}$res]
}
\end{lstlisting}
\noindent\makebox[\linewidth]{\rule{\linewidth}{0.4pt}}

\textbf{expand-for/and}


The \texttt{expand-for/and} procedure expands the \texttt{for/and} macro. It returns an \texttt{and} construct containing the iterations of the clauses.

\begin{tabular}{ |l l| }
\hline
\multicolumn{2}{|l|}{expand-forand (internal)} \\
\hline
tail & an expression tail \\
env & an environment \\
\textit{Returns:} & an expression \\
\hline
\end{tabular}

\noindent\makebox[\linewidth]{\rule{\linewidth}{0.4pt}}
\begin{lstlisting}
proc ::constcl::expand-for/and {tail env} {
    set res [do-for $tail $env]
    return [list [MkSymbol "and"] {*}$res]
}
\end{lstlisting}
\noindent\makebox[\linewidth]{\rule{\linewidth}{0.4pt}}

\textbf{expand-for/list}


The \texttt{expand-for/list} procedure expands the \texttt{for/list} macro. It returns a \texttt{list} construct containing the iterations of each clause.

\begin{tabular}{ |l l| }
\hline
\multicolumn{2}{|l|}{expand forlist (internal)} \\
\hline
tail & an expression tail \\
env & an environment \\
\textit{Returns:} & an expression \\
\hline
\end{tabular}

\noindent\makebox[\linewidth]{\rule{\linewidth}{0.4pt}}
\begin{lstlisting}
proc ::constcl::expand-for/list {tail env} {
    set res [do-for $tail $env]
    return [list [MkSymbol "list"] {*}$res]
}
\end{lstlisting}
\noindent\makebox[\linewidth]{\rule{\linewidth}{0.4pt}}

\textbf{expand-for/or}


The \texttt{expand-for/or} procedure expands the \texttt{for/or} macro. It returns an \texttt{or} construct containing the iterations of each clause.

\begin{tabular}{ |l l| }
\hline
\multicolumn{2}{|l|}{expand-foror (internal)} \\
\hline
tail & an expression tail \\
env & an environment \\
\textit{Returns:} & an expression \\
\hline
\end{tabular}

\noindent\makebox[\linewidth]{\rule{\linewidth}{0.4pt}}
\begin{lstlisting}
proc ::constcl::expand-for/or {tail env} {
    set res [do-for $tail $env]
    return [list [MkSymbol "or"] {*}$res]
}
\end{lstlisting}
\noindent\makebox[\linewidth]{\rule{\linewidth}{0.4pt}}

\textbf{expand-let}


\texttt{expand-let} expands the named \texttt{let} and 'regular' \texttt{let} macros. They ultimately expand to \texttt{lambda} constructs.

\begin{tabular}{ |l l| }
\hline
\multicolumn{2}{|l|}{expand-let (internal)} \\
\hline
tail & an expression tail \\
env & an environment \\
\textit{Returns:} & an expression \\
\hline
\end{tabular}

\noindent\makebox[\linewidth]{\rule{\linewidth}{0.4pt}}
\begin{lstlisting}
proc ::constcl::expand-let {tail env} {
    set env [::constcl::Environment new #NIL {} $env]
    ::if {[symbol? [car $tail]] ne "#f"} {
        # named let
        set variable [car $tail]
        set bindings [cadr $tail]
        set body [cddr $tail]
        set vars [dict create $variable #f]
        parse-bindings vars $bindings
        $env setstr "decl" [list {*}[dict values [dict map {k v} $vars {list $k $v}]]]
        $env setstr "variable" $variable
        $env setstr "varlist" [list {*}[lrange [dict keys $vars] 1 end]]
        $env setstr "body" $body
        $env setstr "call" [list {*}[dict keys $vars]]
        set qq "`(let ,decl (set! ,variable (lambda ,varlist ,@body)) ,call)"
        return [expand-quasiquote [cdr [parse $qq]] $env]
    } else {
        # regular let
        set bindings [car $tail]
        set body [cdr $tail]
        set vars [dict create]
        parse-bindings vars $bindings
        $env setstr "varlist" [list {*}[dict keys $vars]]
        $env setstr "body" $body
        $env setstr "vallist" [list {*}[dict values $vars]]
        set qq "`((lambda ,varlist ,@body) ,@vallist)"
        return [expand-quasiquote [cdr [parse $qq]] $env]
    }
}
 
proc ::constcl::parse-bindings {name bindings} {
    upvar $name vars
    foreach binding [splitlist $bindings] {
        set var [car $binding]
        set val [cadr $binding]
        ::if {$var in [dict keys $vars]} {::error "variable '$var' occurs more than once in let construct"}
        dict set vars $var $val
    }
}
\end{lstlisting}
\noindent\makebox[\linewidth]{\rule{\linewidth}{0.4pt}}

\textbf{expand-or}


\texttt{expand-or} expands the \texttt{or} macro. It returns a \texttt{begin}-expression if the macro

\begin{tabular}{ |l l| }
\hline
\multicolumn{2}{|l|}{expand-or (internal)} \\
\hline
tail & an expression tail \\
env & an environment \\
\textit{Returns:} & an expression \\
\hline
\end{tabular}

\noindent\makebox[\linewidth]{\rule{\linewidth}{0.4pt}}
\begin{lstlisting}
proc ::constcl::expand-or {tail env} {
        return [list #B #f]
    } elseif {[eq? [length $tail] #1] ne "#f"} {
        return [cons #B $tail]
    } else {
        return [do-or $tail $env]
    }
}
\end{lstlisting}
\noindent\makebox[\linewidth]{\rule{\linewidth}{0.4pt}}
\begin{tabular}{ |l l| }
\hline
\multicolumn{2}{|l|}{do-or (internal)} \\
\hline
tail & an expression tail \\
env & an environment \\
\textit{Returns:} & an expression \\
\hline
\end{tabular}

\noindent\makebox[\linewidth]{\rule{\linewidth}{0.4pt}}
\begin{lstlisting}
proc ::constcl::do-or {tail env} {
    set env [::constcl::Environment new #NIL {} $env]
        return #f
    } {
        $env setstr "first" [car $tail]
        $env setstr "rest" [do-or [cdr $tail] $env]
        set qq "`(let ((x ,first)) (if x x ,rest))"
        return [expand-quasiquote [cdr [parse $qq]] $env]
    }
}
\end{lstlisting}
\noindent\makebox[\linewidth]{\rule{\linewidth}{0.4pt}}

\textbf{expand-pop!}


The macro \texttt{push!} updates a list. It adds a new element as the new first element.

\begin{tabular}{ |l l| }
\hline
\multicolumn{2}{|l|}{expand-pop! (internal)} \\
\hline
tail & an expression tail \\
env & an environment \\
\textit{Returns:} & an expression \\
\hline
\end{tabular}

\noindent\makebox[\linewidth]{\rule{\linewidth}{0.4pt}}
\begin{lstlisting}
proc ::constcl::expand-pop! {tail env} {
    set env [::constcl::Environment new #NIL {} $env]
    ::if {[null? $tail] ne "#f"} {::error "too few arguments:\n(push! obj listname)"}
    $env set [MkSymbol "obj"] [car $tail]
    ::if {[null? [cdr $tail]] ne "#f"} {::error "too few arguments:\n(push! obj listname)"}
    ::if {[symbol? [cadr $tail]] eq "#f"} {::error "SYMBOL expected:\n(push! obj listname)"}
    $env set [MkSymbol "listname"] [cadr $tail]
    set qq "`(set! ,listname (cdr ,listname))"
    return [expand-quasiquote [cdr [parse $qq]] $env]
}
\end{lstlisting}
\noindent\makebox[\linewidth]{\rule{\linewidth}{0.4pt}}

\textbf{expand-push!}


The macro \texttt{push!} updates a list. It adds a new element as the new first element.

\begin{tabular}{ |l l| }
\hline
\multicolumn{2}{|l|}{expand-push! (internal)} \\
\hline
tail & an expression tail \\
env & an environment \\
\textit{Returns:} & an expression \\
\hline
\end{tabular}

\noindent\makebox[\linewidth]{\rule{\linewidth}{0.4pt}}
\begin{lstlisting}
proc ::constcl::expand-push! {tail env} {
    set env [::constcl::Environment new #NIL {} $env]
    ::if {[null? $tail] ne "#f"} {::error "too few arguments:\n(push! obj listname)"}
    $env set [MkSymbol "obj"] [car $tail]
    ::if {[null? [cdr $tail]] ne "#f"} {::error "too few arguments:\n(push! obj listname)"}
    ::if {[symbol? [cadr $tail]] eq "#f"} {::error "SYMBOL expected:\n(push! obj listname)"}
    $env set [MkSymbol "listname"] [cadr $tail]
    set qq "`(set! ,listname (cons ,obj ,listname))"
    return [expand-quasiquote [cdr [parse $qq]] $env]
}
\end{lstlisting}
\noindent\makebox[\linewidth]{\rule{\linewidth}{0.4pt}}

\textbf{expand-put!}


The macro \texttt{put!} updates a property list. It adds a key-value pair if the key isn't present, or changes the value in place if it is.

\begin{tabular}{ |l l| }
\hline
\multicolumn{2}{|l|}{expand-put! (internal)} \\
\hline
tail & an expression tail \\
env & an environment \\
\textit{Returns:} & an expression \\
\hline
\end{tabular}

\noindent\makebox[\linewidth]{\rule{\linewidth}{0.4pt}}
\begin{lstlisting}
proc ::constcl::expand-put! {tail env} {
    set env [::constcl::Environment new #NIL {} $env]
    $env set [MkSymbol "listname"] [car $tail]
    ::if {[null? [cdr $tail]] ne "#f"} {::error "too few arguments, 3 expected, got 1"}
    $env set [MkSymbol "key"] [cadr $tail]
    ::if {[null? [cddr $tail]] ne "#f"} {::error "too few arguments, 3 expected, got 2"}
    $env set [MkSymbol "val"] [caddr $tail]
    set qq "`(let ((idx (list-find-key ,listname ,key)))
                 (set! ,listname (append (list ,key ,val) ,listname))
                 (begin (list-set! ,listname (+ idx 1) ,val) ,listname)))"
    return [expand-quasiquote [cdr [parse $qq]] $env]
}
\end{lstlisting}
\noindent\makebox[\linewidth]{\rule{\linewidth}{0.4pt}}

\textbf{expand-quasiquote}


A quasi-quote isn't a macro, but we will deal with it in this section anyway. \texttt{expand-quasiquote} traverses the quasi-quoted structure searching for \texttt{unquote} and \texttt{unquote-splicing}. This code is brittle and sprawling and I barely understand it myself.

\begin{tabular}{ |l l| }
\hline
\multicolumn{2}{|l|}{qq-visit-child (internal)} \\
\hline
node & a Lisp list of expressions \\
qqlevel & a Tcl number \\
env & an environment \\
\textit{Returns:} & a Tcl list of expressions \\
\hline
\end{tabular}

\noindent\makebox[\linewidth]{\rule{\linewidth}{0.4pt}}
\begin{lstlisting}
proc ::constcl::qq-visit-child {node qqlevel env} {
    }
    ::if {[list? $node] ne "#f"} {
        set res {}
        foreach child [splitlist $node] {
            ::if {[pair? $child] ne "#f" && [eq? [car $child] [MkSymbol "unquote"]] ne "#f"} {
                    lappend res [eval [cadr $child] $env]
                } else {
                    lappend res [list #U [qq-visit-child [cadr $child] [expr {$qqlevel - 1}] $env]]
                }
            } elseif {[pair? $child] ne "#f" && [eq? [car $child] [MkSymbol "unquote-splicing"]] ne "#f"} {
                    lappend res {*}[splitlist [eval [cadr $child] $env]]
                }
            } elseif {[pair? $child] ne "#f" && [eq? [car $child] [MkSymbol "quasiquote"]] ne "#f"} {
                lappend res [list [MkSymbol "quasiquote"] [car [qq-visit-child [cdr $child] [expr {$qqlevel + 1}] $env]]] 
            } elseif {[atom? $child] ne "#f"} {
                lappend res $child
            } else {
                lappend res [qq-visit-child $child $qqlevel $env]
            }
        }
    }
    return [list {*}$res]
}
\end{lstlisting}
\noindent\makebox[\linewidth]{\rule{\linewidth}{0.4pt}}
\begin{tabular}{ |l l| }
\hline
\multicolumn{2}{|l|}{expand-quasiquote (internal)} \\
\hline
tail & an expression tail \\
env & an environment \\
\textit{Returns:} & an expression \\
\hline
\end{tabular}

\noindent\makebox[\linewidth]{\rule{\linewidth}{0.4pt}}
\begin{lstlisting}
proc ::constcl::expand-quasiquote {tail env} {
    ::if {[list? [car $tail]] ne "#f"} {
        set node [car $tail]
    } elseif {[vector? [car $tail]] ne "#f"} {
        set vect [car $tail]
        set res {}
            set idx [MkNumber $i]
            set vecref [vector-ref $vect $idx]
            ::if {[pair? $vecref] ne "#f" && [eq? [car $vecref] [MkSymbol "unquote"]] ne "#f"} {
                    lappend res [eval [cadr $vecref] $env]
                }
            } elseif {[pair? $vecref] ne "#f" && [eq? [car $vecref] [MkSymbol "unquote-splicing"]] ne "#f"} {
                    lappend res {*}[splitlist [eval [cadr $vecref] $env]]
                }
            } elseif {[atom? $vecref] ne "#f"} {
                lappend res $vecref
            } else {
            }
        }
        return [list [MkSymbol "vector"] {*}$res]
    }
}
\end{lstlisting}
\noindent\makebox[\linewidth]{\rule{\linewidth}{0.4pt}}

\texttt{unless} is a conditional like \texttt{if}, with the differences that it takes a number of expressions and only executes them for a false outcome of \texttt{car \$tail}.

\begin{tabular}{ |l l| }
\hline
\multicolumn{2}{|l|}{expand-unless (internal)} \\
\hline
tail & an expression tail \\
env & an environment \\
\textit{Returns:} & an expression \\
\hline
\end{tabular}

\noindent\makebox[\linewidth]{\rule{\linewidth}{0.4pt}}
\begin{lstlisting}
proc ::constcl::expand-unless {tail env} {
    set env [::constcl::Environment new #NIL {} $env]
    $env setstr "tail" $tail
    set qq "`(if ,(car tail) (quote ()) (begin ,@(cdr tail)))"
    return [expand-quasiquote [cdr [parse $qq]] $env]
}
\end{lstlisting}
\noindent\makebox[\linewidth]{\rule{\linewidth}{0.4pt}}

\texttt{when} is a conditional like \texttt{if}, with the differences that it takes a number of expressions and only executes them for a true outcome of \texttt{car \$tail}.

\begin{tabular}{ |l l| }
\hline
\multicolumn{2}{|l|}{expand-when (internal)} \\
\hline
tail & an expression tail \\
env & an environment \\
\textit{Returns:} & an expression \\
\hline
\end{tabular}

\noindent\makebox[\linewidth]{\rule{\linewidth}{0.4pt}}
\begin{lstlisting}
proc ::constcl::expand-when {tail env} {
    set env [::constcl::Environment new #NIL {} $env]
    $env setstr "tail" $tail
    set qq "`(if ,(car tail) (begin ,@(cdr tail)) (quote ()))"
    return [expand-quasiquote [cdr [parse $qq]] $env]
}
\end{lstlisting}
\noindent\makebox[\linewidth]{\rule{\linewidth}{0.4pt}}
\subsection{Resolving local defines}
\label{resolving-local-defines}

This section is ported from 'Scheme 9 from Empty Space'. \texttt{resolve-local-defines} is the topmost procedure in rewriting local defines as essentially a \texttt{letrec} form. It takes a list of expressions and extracts variables and values from the defines in the beginning of the list. It builds a double lambda expression with the variables and values, and the rest of the expressions from the original list as body.

\begin{tabular}{ |l l| }
\hline
\multicolumn{2}{|l|}{resolve-local-defines} \\
\hline
exps & a Lisp list of expressions \\
\textit{Returns:} & an expression \\
\hline
\end{tabular}

\noindent\makebox[\linewidth]{\rule{\linewidth}{0.4pt}}
\begin{lstlisting}
proc ::constcl::resolve-local-defines {exps} {
    set rest [lassign [extract-from-defines $exps VALS] a error]
    ::if {$error ne "#f"} {
        return #NIL
    }
    set rest [lassign [extract-from-defines $exps VARS] v error]
    ::if {$rest eq "#NIL"} {
        set rest [cons #UNSP #NIL]
    }
    return [make-recursive-lambda $v $a $rest]
}
\end{lstlisting}
\noindent\makebox[\linewidth]{\rule{\linewidth}{0.4pt}}

\texttt{extract-from-defines} visits every define in the given list of expressions and extracts either a variable name or a value, depending on the state of the \_part\_ flag, from each one of them. A Tcl list of 1) the resulting list of names or values, 2) error state, and 3) the rest of the expressions in the original list is returned.

\begin{tabular}{ |l l| }
\hline
\multicolumn{2}{|l|}{extract-from-defines (internal)} \\
\hline
exps & a Lisp list of expressions \\
part & a flag, VARS or VALS \\
\textit{Returns:} & a Tcl list of Lisp values \\
\hline
\end{tabular}

\noindent\makebox[\linewidth]{\rule{\linewidth}{0.4pt}}
\begin{lstlisting}
proc ::constcl::extract-from-defines {exps part} {
    set a #NIL
    while {$exps ne "#NIL"} {
        ::if {[atom? $exps] ne "#f" || [atom? [car $exps]] ne "#f" || [eq? [caar $exps] [MkSymbol "define"]] eq "#f"} {
            break
        }
        set n [car $exps]
        set k [length $n]
        ::if {[list? $n] eq "#f" || [$k numval] < 3 || [$k numval] > 3 ||
            ([argument-list? [cadr $n]] ne "#f" || [symbol? [cadr $n]] eq "#f")
            eq "#f"} {
            return [::list {} "#t" {}]
        }
        ::if {[pair? [cadr $n]] ne "#f"} {
            ::if {$part eq "VARS"} {
                set a [cons [caadr $n] $a]
            } else {
                set a [cons #NIL $a]
                set new [cons [cdadr $n] [cddr $n]]
                set new [cons #λ $new]
                set-car! $a $new
            }
        } else {
            ::if {$part eq "VARS"} {
                set a [cons [cadr $n] $a]
            } else {
                set a [cons [caddr $n] $a]
            }
        }
        set exps [cdr $exps]
    }
    return [::list $a #f $exps]
}
\end{lstlisting}
\noindent\makebox[\linewidth]{\rule{\linewidth}{0.4pt}}

\texttt{argument-list?} accepts a Scheme formals list and rejects other values.

\begin{tabular}{ |l l| }
\hline
\multicolumn{2}{|l|}{argument-list? (internal)} \\
\hline
val & a Lisp value \\
\textit{Returns:} & a boolean \\
\hline
\end{tabular}

\noindent\makebox[\linewidth]{\rule{\linewidth}{0.4pt}}
\begin{lstlisting}
proc ::constcl::argument-list? {val} {
    ::if {$val eq "#NIL"} {
        return #t
    } elseif {[symbol? $val] ne "#f"} {
        return #t
    } elseif {[atom? $val] ne "#f"} {
        return #f
    }
    while {[pair? $val] ne "#f"} {
        ::if {[symbol? [car $val]] eq "#f"} {
            return #f
        }
        set val [cdr $val]
    }
    ::if {$val eq "#NIL"} {
        return #t
    } elseif {[symbol? $val] ne "#f"} {
        return #t
    }
}
\end{lstlisting}
\noindent\makebox[\linewidth]{\rule{\linewidth}{0.4pt}}

\texttt{make-recursive-lambda} builds the \texttt{letrec} structure.

\begin{tabular}{ |l l| }
\hline
\multicolumn{2}{|l|}{make-recursive-lambda (internal)} \\
\hline
vars & a Lisp list of symbols \\
args & a Lisp list of expressions \\
body & a Lisp list of expressions \\
\textit{Returns:} & an expression \\
\hline
\end{tabular}

\noindent\makebox[\linewidth]{\rule{\linewidth}{0.4pt}}
\begin{lstlisting}
proc ::constcl::make-recursive-lambda {vars args body} {
    set tmps [make-temporaries $vars]
    set body [append-b [make-assignments $vars $tmps] $body]
    set body [cons $body #NIL]
    set n [cons $tmps $body]
    set n [cons #λ $n]
    set n [cons $n $args]
    set n [cons $n #NIL]
    set n [cons $vars $n]
    set n [cons #λ $n]
    set n [cons $n [make-undefineds $vars]]
    return $n
}
\end{lstlisting}
\noindent\makebox[\linewidth]{\rule{\linewidth}{0.4pt}}

\texttt{make-temporaries} creates the symbols that will act as middlemen in transferring the values to the variables.

\begin{tabular}{ |l l| }
\hline
\multicolumn{2}{|l|}{make-temporaries (internal)} \\
\hline
vals & a Lisp list of Lisp values \\
\textit{Returns:} & a Lisp list of Lisp values \\
\hline
\end{tabular}

\noindent\makebox[\linewidth]{\rule{\linewidth}{0.4pt}}
\begin{lstlisting}
proc ::constcl::make-temporaries {vals} {
    set n #NIL
    while {$vals ne "#NIL"} {
        set sym [gensym "g"]
        set n [cons $sym $n]
        set vals [cdr $vals]
    }
    return $n
}
\end{lstlisting}
\noindent\makebox[\linewidth]{\rule{\linewidth}{0.4pt}}

\texttt{gensym} generates an unique symbol.

\begin{tabular}{ |l l| }
\hline
\multicolumn{2}{|l|}{gensym (internal)} \\
\hline
prefix & a string \\
\textit{Returns:} & a symbol \\
\hline
\end{tabular}

\noindent\makebox[\linewidth]{\rule{\linewidth}{0.4pt}}
\begin{lstlisting}
proc ::constcl::gensym {prefix} {
    set symbolnames [lmap s [info class instances ::constcl::Symbol] {$s name}]
    set s $prefix<[incr ::constcl::gensymnum]>
    while {$s in $symbolnames} {
        set s $prefix[incr ::constcl::gensymnum]
    }
    return [MkSymbol $s]
}
\end{lstlisting}
\noindent\makebox[\linewidth]{\rule{\linewidth}{0.4pt}}

\texttt{append-b} joins two lists together.

\begin{tabular}{ |l l| }
\hline
\multicolumn{2}{|l|}{append-b (internal)} \\
\hline
a & a Lisp list of Lisp values \\
b & a Lisp list of Lisp values \\
\textit{Returns:} & a Lisp list of Lisp values \\
\hline
\end{tabular}

\noindent\makebox[\linewidth]{\rule{\linewidth}{0.4pt}}
\begin{lstlisting}
proc ::constcl::append-b {a b} {
    ::if {$a eq "#NIL"} {
        return $b
    }
    set p $a
    while {$p ne "#NIL"} {
        ::if {[atom? $p] ne "#f"} {
            ::error "append: improper list"
        }
        set last $p
        set p [cdr $p]
    }
    set-cdr! $last $b
    return $a
}
\end{lstlisting}
\noindent\makebox[\linewidth]{\rule{\linewidth}{0.4pt}}

\texttt{make-assignments} creates the structure that holds the assignment statements. Later on, it will be joined to the body of the finished expression.

\begin{tabular}{ |l l| }
\hline
\multicolumn{2}{|l|}{make-assignments (internal)} \\
\hline
vars & a Lisp list of symbols \\
tmps & a Lisp list of symbols \\
\textit{Returns:} & an expression \\
\hline
\end{tabular}

\noindent\makebox[\linewidth]{\rule{\linewidth}{0.4pt}}
\begin{lstlisting}
proc ::constcl::make-assignments {vars tmps} {
    set n #NIL
    while {$vars ne "#NIL"} {
       set asg [cons [car $tmps] #NIL]
       set asg [cons [car $vars] $asg]
       set asg [cons #S $asg]
       set n [cons $asg $n]
       set vars [cdr $vars]
       set tmps [cdr $tmps]
   }
   return [cons #B $n]
}
\end{lstlisting}
\noindent\makebox[\linewidth]{\rule{\linewidth}{0.4pt}}

Due to a mysterious bug, \texttt{make-undefineds} actually creates a list of NIL values instead of undefined values.

\begin{tabular}{ |l l| }
\hline
\multicolumn{2}{|l|}{make-undefineds (internal)} \\
\hline
vals & a Lisp list of Lisp values \\
\textit{Returns:} & a Lisp list of nil values \\
\hline
\end{tabular}

\noindent\makebox[\linewidth]{\rule{\linewidth}{0.4pt}}
\begin{lstlisting}
proc ::constcl::make-undefineds {vals} {
    # Use #NIL instead of #UNDF because of some strange bug with eval-list.
    set n #NIL
    while {$vals ne "#NIL"} {
        set n [cons #NIL $n]
        set vals [cdr $vals]
    }
    return $n
}
\end{lstlisting}
\noindent\makebox[\linewidth]{\rule{\linewidth}{0.4pt}}
\noindent\makebox[\linewidth]{\rule{\linewidth}{0.4pt}}
\begin{lstlisting}
proc ::constcl::scheme-report-environment {version} {
    # TODO
}
\end{lstlisting}
\noindent\makebox[\linewidth]{\rule{\linewidth}{0.4pt}}
\noindent\makebox[\linewidth]{\rule{\linewidth}{0.4pt}}
\begin{lstlisting}
proc ::constcl::null-environment {version} {
    # TODO
}
\end{lstlisting}
\noindent\makebox[\linewidth]{\rule{\linewidth}{0.4pt}}
\noindent\makebox[\linewidth]{\rule{\linewidth}{0.4pt}}
\begin{lstlisting}
proc ::constcl::interaction-environment {} {
    # TODO
}
\end{lstlisting}
\noindent\makebox[\linewidth]{\rule{\linewidth}{0.4pt}}
\section{write}
\label{write}

\textbf{write}


The third member in the great triad is \texttt{write}. As long as the object given to it isn't \texttt{\#NONE}, it passes it to \texttt{write-value} and prints a newline.

\begin{tabular}{ |l l| }
\hline
\multicolumn{2}{|l|}{write (public)} \\
\hline
val & a Lisp value \\
args & -don't care- \\
\textit{Returns:} & nothing \\
\hline
\end{tabular}

\noindent\makebox[\linewidth]{\rule{\linewidth}{0.4pt}}
\begin{lstlisting}
reg write ::constcl::write
 
proc ::constcl::write {val args} {
    ::if {$val ne "#NONE"} {
        ::if {[llength $args]} {
            lassign $args port
        } else {
            set port [MkOutputPort stdout]
        }
        set ::constcl::Output_port $port
        write-value [$::constcl::Output_port handle] $val
        puts [$::constcl::Output_port handle] {}
        set ::constcl::Output_port [MkOutputPort stdout]
    }
    return
}
\end{lstlisting}
\noindent\makebox[\linewidth]{\rule{\linewidth}{0.4pt}}

\textbf{write-value}


\texttt{write-value} simply calls an object's \texttt{write} method, letting the object write itself.

\begin{tabular}{ |l l| }
\hline
\multicolumn{2}{|l|}{write-value (internal)} \\
\hline
handle & a channel handle \\
val & a Lisp value \\
\textit{Returns:} & nothing \\
\hline
\end{tabular}

\noindent\makebox[\linewidth]{\rule{\linewidth}{0.4pt}}
\begin{lstlisting}
proc ::constcl::write-value {handle val} {
    $val write $handle
    return
}
\end{lstlisting}
\noindent\makebox[\linewidth]{\rule{\linewidth}{0.4pt}}

\textbf{display}


The \texttt{display} procedure is like \texttt{write} but doesn't print a newline.

\begin{tabular}{ |l l| }
\hline
\multicolumn{2}{|l|}{display (public)} \\
\hline
val & a Lisp value \\
args & -don't care- \\
\textit{Returns:} & nothing \\
\hline
\end{tabular}

\noindent\makebox[\linewidth]{\rule{\linewidth}{0.4pt}}
\begin{lstlisting}
reg display ::constcl::display
 
proc ::constcl::display {val args} {
    ::if {$val ne "#NONE"} {
        $val display
        flush stdout
    }
    return
}
\end{lstlisting}
\noindent\makebox[\linewidth]{\rule{\linewidth}{0.4pt}}

\textbf{write-pair}


The \texttt{write-pair} procedure prints a Pair object.

\begin{tabular}{ |l l| }
\hline
\multicolumn{2}{|l|}{write-pair (internal)} \\
\hline
handle & a channel handle \\
pair & a pair \\
\textit{Returns:} & nothing \\
\hline
\end{tabular}

\noindent\makebox[\linewidth]{\rule{\linewidth}{0.4pt}}
\begin{lstlisting}
proc ::constcl::write-pair {handle pair} {
    # take an object and print the car and the cdr of the stored value
    set a [car $pair]
    set d [cdr $pair]
    # print car
    write-value $handle $a
    ::if {[pair? $d] ne "#f"} {
        # cdr is a cons pair
        puts -nonewline $handle " "
        write-pair $handle $d
    } elseif {[null? $d] ne "#f"} {
        # cdr is nil
        return
    } else {
        # it is an atom
        puts -nonewline $handle " . "
        write-value $handle $d
    }
    return
}
\end{lstlisting}
\noindent\makebox[\linewidth]{\rule{\linewidth}{0.4pt}}
\section{Built-in procedures}
\label{built-in-procedures}
\subsection{Equivalence predicates}
\label{equivalence-predicates}

\textbf{eq}


\textbf{eqv}


\textbf{equal}


Of the three equivalence predicates, \texttt{eq} generally tests for identity (with exception for numbers), \texttt{eqv} tests for value equality (except for booleans and procedures, where it tests for identity), and \texttt{equal} tests for whether the output strings are equal.

\begin{tabular}{ |l l| }
\hline
\multicolumn{2}{|l|}{eq?, eqv?, equal? (public)} \\
\hline
val1 & a Lisp value \\
val2 & a Lisp value \\
\textit{Returns:} & a boolean \\
\hline
\end{tabular}

\noindent\makebox[\linewidth]{\rule{\linewidth}{0.4pt}}
\begin{lstlisting}
reg eq? ::constcl::eq?
 
proc ::constcl::eq? {val1 val2} {
    ::if {[typeeq boolean? $val1 $val2] && $val1 eq $val2} {
        return #t
    } elseif {[typeeq symbol? $val1 $val2] && $val1 eq $val2} {
        return #t
    } elseif {[typeeq number? $val1 $val2] && [valeq $val1 $val2]} {
        return #t
    } elseif {[typeeq char? $val1 $val2] && $val1 eq $val2} {
        return #t
    } elseif {[typeeq null? $val1 $val2]} {
        return #t
    } elseif {[typeeq pair? $val1 $val2] && $val1 eq $val2} {
        return #t
    } elseif {[typeeq string? $val1 $val2] && $val1 eq $val2} {
        return #t
    } elseif {[typeeq vector? $val1 $val2] && $val1 eq $val2} {
        return #t
    } elseif {[typeeq procedure? $val1 $val2] && $val1 eq $val2} {
        return #t
    } else {
        return #f
    }
}
 
proc ::constcl::typeeq {typep val1 val2} {
    return [expr {[$typep $val1] ne "#f" && [$typep $val2] ne "#f"}]
}
 
proc ::constcl::valeq {val1 val2} {
    return [expr {[$val1 value] eq [$val2 value]}]
}
\end{lstlisting}
\noindent\makebox[\linewidth]{\rule{\linewidth}{0.4pt}}
\noindent\makebox[\linewidth]{\rule{\linewidth}{0.4pt}}
\begin{lstlisting}
reg eqv? ::constcl::eqv?
 
proc ::constcl::eqv? {val1 val2} {
    ::if {[typeeq boolean? $val1 $val2] && $val1 eq $val2} {
        return #t
    } elseif {[typeeq symbol? $val1 $val2] && [valeq $val1 $val2]} {
        return #t
    } elseif {[typeeq number? $val1 $val2] && [valeq $val1 $val2]} {
        return #t
    } elseif {[typeeq char? $val1 $val2] && [valeq $val1 eq $val2]} {
        return #t
    } elseif {[typeeq null? $val1 $val2]} {
        return #t
    } elseif {[pair? $val1] ne "#f" && [pair? $val2] ne "#f" &&
        [$val1 car] eq [$val2 car] && [$val1 cdr] eq [$val2 cdr]} {
        return #t
    } elseif {[typeeq string? $val1 $val2] && [valeq $val1 $val2]} {
        return #t
    } elseif {[typeeq vector? $val1 $val2] && [valeq $val1 $val2]} {
        return #t
    } elseif {[typeeq procedure? $val1 $val2] && $val1 eq $val2} {
        return #t
    } else {
        return #f
    }
}
\end{lstlisting}
\noindent\makebox[\linewidth]{\rule{\linewidth}{0.4pt}}
\noindent\makebox[\linewidth]{\rule{\linewidth}{0.4pt}}
\begin{lstlisting}
reg equal? ::constcl::equal?
 
proc ::constcl::equal? {val1 val2} {
    ::if {[$val1 show] eq [$val2 show]} {
        return #t
    } else {
        return #f
    }
    # TODO
}
\end{lstlisting}
\noindent\makebox[\linewidth]{\rule{\linewidth}{0.4pt}}
\subsection{Numbers}
\label{numbers}

I have only implemented a bare-bones version of Scheme's numerical library. The following is a reasonably complete framework for operations on integers and floating-point numbers. No rationals, no complex numbers, no gcd or lcm.


\textbf{Number} class

\noindent\makebox[\linewidth]{\rule{\linewidth}{0.4pt}}
\begin{lstlisting}
oo::class create ::constcl::Number {
    superclass ::constcl::NIL
    variable value
    constructor {v} {
        ::if {[::string is double -strict $v]} {
            set value $v
        } else {
            ::error "NUMBER expected\n$v"
        }
    }
    method odd? {} {::if {$value % 2 == 1} then {return #t} else {return #f}}
    method value {} { set value }
    method numval {} {set value}
    method mkconstant {} {}
    method constant {} {return 1}
    method write {handle} { puts -nonewline $handle [my value] }
    method display {} { puts -nonewline [my value] }
    method show {} { set value }
}
 
interp alias {} ::constcl::MkNumber {} ::constcl::Number new
 
\end{lstlisting}
\noindent\makebox[\linewidth]{\rule{\linewidth}{0.4pt}}

\textbf{number?}


\texttt{number?} recognizes a number by object type, not by content.

\begin{tabular}{ |l l| }
\hline
\multicolumn{2}{|l|}{number? (public)} \\
\hline
val & a Lisp value \\
\textit{Returns:} & a boolean \\
\hline
\end{tabular}

\noindent\makebox[\linewidth]{\rule{\linewidth}{0.4pt}}
\begin{lstlisting}
reg number? ::constcl::number?
 
proc ::constcl::number? {val} {
    ::if {[info object isa typeof $val ::constcl::Number]} {
        return #t
    } elseif {[info object isa typeof [interp alias {} $val] ::constcl::Number]} {
        return #t
    } else {
        return #f
    }
}
\end{lstlisting}
\noindent\makebox[\linewidth]{\rule{\linewidth}{0.4pt}}

\textbf{=}


\textbf{<}


\textbf{>}


\textbf{<=}


\textbf{>=}


The predicates \texttt{=}, \texttt{<}, \texttt{>}, \texttt{<=}, and \texttt{>=} are implemented.

\begin{tabular}{ |l l| }
\hline
\multicolumn{2}{|l|}{=, <, >, <=, >= (public)} \\
\hline
args & some numbers \\
\textit{Returns:} & a boolean \\
\hline
\end{tabular}

\noindent\makebox[\linewidth]{\rule{\linewidth}{0.4pt}}
\begin{lstlisting}
reg = ::constcl::=
 
proc ::constcl::= {args} {
    try {
        set vals [lmap arg $args {$arg numval}]
    } on error {} {
        ::error "NUMBER expected\n(= num ...)"
    }
    ::if {[::tcl::mathop::== {*}$vals]} {
        return #t
    } else {
        return #f
    }
}
\end{lstlisting}
\noindent\makebox[\linewidth]{\rule{\linewidth}{0.4pt}}
\noindent\makebox[\linewidth]{\rule{\linewidth}{0.4pt}}
\begin{lstlisting}
reg < ::constcl::<
 
proc ::constcl::< {args} {
    try {
        set vals [lmap arg $args {$arg numval}]
    } on error {} {
        ::error "NUMBER expected\n(< num ...)"
    }
    ::if {[::tcl::mathop::< {*}$vals]} {
        return #t
    } else {
        return #f
    }
}
\end{lstlisting}
\noindent\makebox[\linewidth]{\rule{\linewidth}{0.4pt}}
\noindent\makebox[\linewidth]{\rule{\linewidth}{0.4pt}}
\begin{lstlisting}
reg > ::constcl::>
 
proc ::constcl::> {args} {
    try {
        set vals [lmap arg $args {$arg numval}]
    } on error {} {
        ::error "NUMBER expected\n(> num ...)"
    }
    ::if {[::tcl::mathop::> {*}$vals]} {
        return #t
    } else {
        return #f
    }
}
\end{lstlisting}
\noindent\makebox[\linewidth]{\rule{\linewidth}{0.4pt}}
\noindent\makebox[\linewidth]{\rule{\linewidth}{0.4pt}}
\begin{lstlisting}
reg <= ::constcl::<=
 
proc ::constcl::<= {args} {
    try {
        set vals [lmap arg $args {$arg numval}]
    } on error {} {
        ::error "NUMBER expected\n(<= num ...)"
    }
    ::if {[::tcl::mathop::<= {*}$vals]} {
        return #t
    } else {
        return #f
    }
}
\end{lstlisting}
\noindent\makebox[\linewidth]{\rule{\linewidth}{0.4pt}}
\noindent\makebox[\linewidth]{\rule{\linewidth}{0.4pt}}
\begin{lstlisting}
reg >= ::constcl::>=
 
proc ::constcl::>= {args} {
    try {
        set vals [lmap arg $args {$arg numval}]
    } on error {} {
        ::error "NUMBER expected\n(>= num ...)"
    }
    ::if {[::tcl::mathop::>= {*}$vals]} {
        return #t
    } else {
        return #f
    }
}
\end{lstlisting}
\noindent\makebox[\linewidth]{\rule{\linewidth}{0.4pt}}

\textbf{zero?}


The \texttt{zero?} predicate tests if a given number is equal to zero.

\begin{tabular}{ |l l| }
\hline
\multicolumn{2}{|l|}{zero? (public)} \\
\hline
num & a number \\
\textit{Returns:} & a boolean \\
\hline
\end{tabular}

\noindent\makebox[\linewidth]{\rule{\linewidth}{0.4pt}}
\begin{lstlisting}
reg zero? ::constcl::zero?
 
proc ::constcl::zero? {num} {
    check {number? $num} {NUMBER expected\n([pn] [$num show])}
    return [$num zero?]
}
\end{lstlisting}
\noindent\makebox[\linewidth]{\rule{\linewidth}{0.4pt}}

\textbf{positive?}


\textbf{negative?}


\textbf{even?}


\textbf{odd?}


The \texttt{positive?}/\texttt{negative?}/\texttt{even?}/\texttt{odd?} predicates test a number for those traits.

\begin{tabular}{ |l l| }
\hline
\multicolumn{2}{|l|}{positive?, negative?, even?, odd? (public)} \\
\hline
num & a number \\
\textit{Returns:} & a boolean \\
\hline
\end{tabular}

\noindent\makebox[\linewidth]{\rule{\linewidth}{0.4pt}}
\begin{lstlisting}
reg positive? ::constcl::positive?
 
proc ::constcl::positive? {num} {
    check {number? $num} {NUMBER expected\n([pn] [$num show])}
    return [$num positive?]
}
\end{lstlisting}
\noindent\makebox[\linewidth]{\rule{\linewidth}{0.4pt}}
\noindent\makebox[\linewidth]{\rule{\linewidth}{0.4pt}}
\begin{lstlisting}
reg negative? ::constcl::negative?
 
proc ::constcl::negative? {num} {
    check {number? $num} {NUMBER expected\n([pn] [$num show])}
    return [$num negative?]
}
\end{lstlisting}
\noindent\makebox[\linewidth]{\rule{\linewidth}{0.4pt}}
\noindent\makebox[\linewidth]{\rule{\linewidth}{0.4pt}}
\begin{lstlisting}
reg even? ::constcl::even?
 
proc ::constcl::even? {num} {
    check {number? $num} {NUMBER expected\n([pn] [$num show])}
    return [$num even?]
}
\end{lstlisting}
\noindent\makebox[\linewidth]{\rule{\linewidth}{0.4pt}}
\noindent\makebox[\linewidth]{\rule{\linewidth}{0.4pt}}
\begin{lstlisting}
reg odd? ::constcl::odd?
 
proc ::constcl::odd? {num} {
    check {number? $num} {NUMBER expected\n([pn] [$num show])}
    return [$num odd?]
}
\end{lstlisting}
\noindent\makebox[\linewidth]{\rule{\linewidth}{0.4pt}}

\textbf{max}


\textbf{min}


The \texttt{max} function selects the largest number, and the \texttt{min} function selects the smallest number.

\begin{tabular}{ |l l| }
\hline
\multicolumn{2}{|l|}{max, min (public)} \\
\hline
num & a number \\
args & some numbers \\
\textit{Returns:} & a number \\
\hline
\end{tabular}


Example:

\noindent\makebox[\linewidth]{\rule{\linewidth}{0.4pt}}
\begin{lstlisting}
\end{lstlisting}
\noindent\makebox[\linewidth]{\rule{\linewidth}{0.4pt}}
\noindent\makebox[\linewidth]{\rule{\linewidth}{0.4pt}}
\begin{lstlisting}
reg max ::constcl::max
 
proc ::constcl::max {num args} {
    try {
        set vals [lmap arg [::list $num {*}$args] {$arg numval}]
    } on error {} {
        ::error "NUMBER expected\n(max num...)"
    }
    MkNumber [::tcl::mathfunc::max {*}$vals]
}
\end{lstlisting}
\noindent\makebox[\linewidth]{\rule{\linewidth}{0.4pt}}
\noindent\makebox[\linewidth]{\rule{\linewidth}{0.4pt}}
\begin{lstlisting}
reg min ::constcl::min
 
proc ::constcl::min {num args} {
    try {
        set vals [lmap arg [::list $num {*}$args] {$arg numval}]
    } on error {} {
        ::error "NUMBER expected\n(min num...)"
    }
    MkNumber [::tcl::mathfunc::min {*}$vals]
}
\end{lstlisting}
\noindent\makebox[\linewidth]{\rule{\linewidth}{0.4pt}}

\textbf{+}


\textbf{*}


\textbf{-}


\textbf{/}


The operators \texttt{+}, \texttt{*}, \texttt{-}, and \texttt{/} stand for the respective mathematical operations. They take a number of operands, but at least one for \texttt{-} and \texttt{/}.

\begin{tabular}{ |l l| }
\hline
\multicolumn{2}{|l|}{+, * (public)} \\
\hline
args & some numbers \\
\textit{Returns:} & a number \\
\hline
\end{tabular}

\begin{tabular}{ |l l| }
\hline
\multicolumn{2}{|l|}{-,  (public)} \\
\hline
num & a number \\
args & some numbers \\
\textit{Returns:} & a number \\
\hline
\end{tabular}


Example:

\noindent\makebox[\linewidth]{\rule{\linewidth}{0.4pt}}
\begin{lstlisting}
(+ 21 7 3)                                 ⇒  31
(* 21 7 3)                                 ⇒  441
(- 21 7 3)                                 ⇒  11
(/ 21 7 3)                                 ⇒  1
(- 5)                                      ⇒  -5
\end{lstlisting}
\noindent\makebox[\linewidth]{\rule{\linewidth}{0.4pt}}
\noindent\makebox[\linewidth]{\rule{\linewidth}{0.4pt}}
\begin{lstlisting}
reg + ::constcl::+
 
proc ::constcl::+ {args} {
    try {
        set vals [lmap arg $args {$arg numval}]
    } on error {} {
        ::error "NUMBER expected\n(+ num ...)"
    }
    MkNumber [::tcl::mathop::+ {*}$vals]
}
\end{lstlisting}
\noindent\makebox[\linewidth]{\rule{\linewidth}{0.4pt}}
\noindent\makebox[\linewidth]{\rule{\linewidth}{0.4pt}}
\begin{lstlisting}
reg * ::constcl::*
 
proc ::constcl::* {args} {
    try {
        set vals [lmap arg $args {$arg numval}]
    } on error {} {
        ::error "NUMBER expected\n(* num ...)"
    }
    MkNumber [::tcl::mathop::* {*}$vals]
}
\end{lstlisting}
\noindent\makebox[\linewidth]{\rule{\linewidth}{0.4pt}}
\noindent\makebox[\linewidth]{\rule{\linewidth}{0.4pt}}
\begin{lstlisting}
reg - ::constcl::-
 
proc ::constcl::- {num args} {
    try {
        set vals [lmap arg $args {$arg numval}]
    } on error {} {
        ::error "NUMBER expected\n(- num ...)"
    }
    MkNumber [::tcl::mathop::- [$num numval] {*}$vals]
}
\end{lstlisting}
\noindent\makebox[\linewidth]{\rule{\linewidth}{0.4pt}}
\noindent\makebox[\linewidth]{\rule{\linewidth}{0.4pt}}
\begin{lstlisting}
reg / ::constcl::/
 
proc ::constcl::/ {num args} {
    try {
        set vals [lmap arg $args {$arg numval}]
    } on error {} {
        ::error "NUMBER expected\n(/ num ...)"
    }
    MkNumber [::tcl::mathop::/ [$num numval] {*}$vals]
}
\end{lstlisting}
\noindent\makebox[\linewidth]{\rule{\linewidth}{0.4pt}}

\textbf{abs}


The \texttt{abs} function yields the absolute value of a number.

\begin{tabular}{ |l l| }
\hline
\multicolumn{2}{|l|}{abs (public)} \\
\hline
num & a number \\
\textit{Returns:} & a number \\
\hline
\end{tabular}

\noindent\makebox[\linewidth]{\rule{\linewidth}{0.4pt}}
\begin{lstlisting}
reg abs ::constcl::abs
 
proc ::constcl::abs {num} {
    check {number? $num} {NUMBER expected\n([pn] [$num show])}
    ::if {[$num negative?] ne "#f"} {
        return [MkNumber [expr {[$num numval] * -1}]]
    } else {
        return $num
    }
}
\end{lstlisting}
\noindent\makebox[\linewidth]{\rule{\linewidth}{0.4pt}}

\textbf{quotient}


\texttt{quotient} calculates the quotient between two numbers.

\begin{tabular}{ |l l| }
\hline
\multicolumn{2}{|l|}{quotient (public)} \\
\hline
num1 & a number \\
num2 & a number \\
\textit{Returns:} & a number \\
\hline
\end{tabular}


Example:

\noindent\makebox[\linewidth]{\rule{\linewidth}{0.4pt}}
\begin{lstlisting}
\end{lstlisting}
\noindent\makebox[\linewidth]{\rule{\linewidth}{0.4pt}}
\noindent\makebox[\linewidth]{\rule{\linewidth}{0.4pt}}
\begin{lstlisting}
reg quotient
 
proc ::constcl::quotient {num1 num2} {
    set q [::tcl::mathop::/ [$num1 numval] [$num2 numval]]
        return [MkNumber [::tcl::mathfunc::floor $q]]
        return [MkNumber [::tcl::mathfunc::ceil $q]]
    } else {
    }
}
\end{lstlisting}
\noindent\makebox[\linewidth]{\rule{\linewidth}{0.4pt}}

\textbf{remainder}


\texttt{remainder} is a variant of the modulus function. (I'm a programmer, not a mathematician!)

\begin{tabular}{ |l l| }
\hline
\multicolumn{2}{|l|}{remainder (public)} \\
\hline
num1 & a number \\
num2 & a number \\
\textit{Returns:} & a number \\
\hline
\end{tabular}


Example:

\noindent\makebox[\linewidth]{\rule{\linewidth}{0.4pt}}
\begin{lstlisting}
(remainder 7 3)   ⇒  1
\end{lstlisting}
\noindent\makebox[\linewidth]{\rule{\linewidth}{0.4pt}}
\noindent\makebox[\linewidth]{\rule{\linewidth}{0.4pt}}
\begin{lstlisting}
reg remainder
 
proc ::constcl::remainder {num1 num2} {
    set n [::tcl::mathop::% [[abs $num1] numval] [[abs $num2] numval]]
    ::if {[$num1 negative?] ne "#f"} {
        set n -$n
    }
    return [MkNumber $n]
}
\end{lstlisting}
\noindent\makebox[\linewidth]{\rule{\linewidth}{0.4pt}}

\textbf{modulo}

\begin{tabular}{ |l l| }
\hline
\multicolumn{2}{|l|}{modulo (public)} \\
\hline
num1 & a number \\
num2 & a number \\
\textit{Returns:} & a number \\
\hline
\end{tabular}


Example:

\noindent\makebox[\linewidth]{\rule{\linewidth}{0.4pt}}
\begin{lstlisting}
(modulo 7 3)   ⇒  1
\end{lstlisting}
\noindent\makebox[\linewidth]{\rule{\linewidth}{0.4pt}}
\noindent\makebox[\linewidth]{\rule{\linewidth}{0.4pt}}
\begin{lstlisting}
reg modulo
 
proc ::constcl::modulo {num1 num2} {
    return [MkNumber [::tcl::mathop::% [$num1 numval] [$num2 numval]]]
}
\end{lstlisting}
\noindent\makebox[\linewidth]{\rule{\linewidth}{0.4pt}}
\noindent\makebox[\linewidth]{\rule{\linewidth}{0.4pt}}
\begin{lstlisting}
proc ::constcl::gcd {args} {
    # TODO
}
\end{lstlisting}
\noindent\makebox[\linewidth]{\rule{\linewidth}{0.4pt}}
\noindent\makebox[\linewidth]{\rule{\linewidth}{0.4pt}}
\begin{lstlisting}
proc ::constcl::lcm {args} {
    # TODO
}
\end{lstlisting}
\noindent\makebox[\linewidth]{\rule{\linewidth}{0.4pt}}
\noindent\makebox[\linewidth]{\rule{\linewidth}{0.4pt}}
\begin{lstlisting}
proc ::constcl::numerator {q} {
    # TODO
}
\end{lstlisting}
\noindent\makebox[\linewidth]{\rule{\linewidth}{0.4pt}}
\noindent\makebox[\linewidth]{\rule{\linewidth}{0.4pt}}
\begin{lstlisting}
proc ::constcl::denominator {q} {
    # TODO
}
\end{lstlisting}
\noindent\makebox[\linewidth]{\rule{\linewidth}{0.4pt}}

\textbf{floor}


\textbf{ceiling}


\textbf{truncate}


\textbf{round}


\texttt{floor}, \texttt{ceiling}, \texttt{truncate}, and \texttt{round} are different methods for converting a real number to an integer.

\begin{tabular}{ |l l| }
\hline
\multicolumn{2}{|l|}{floor, ceiling, truncate, round (public)} \\
\hline
num & a number \\
\textit{Returns:} & a number \\
\hline
\end{tabular}


Example:

\noindent\makebox[\linewidth]{\rule{\linewidth}{0.4pt}}
\begin{lstlisting}
(round 7.5)      ⇒  8
\end{lstlisting}
\noindent\makebox[\linewidth]{\rule{\linewidth}{0.4pt}}
\noindent\makebox[\linewidth]{\rule{\linewidth}{0.4pt}}
\begin{lstlisting}
reg floor ::constcl::floor
 
proc ::constcl::floor {num} {
    check {number? $num} {NUMBER expected\n([pn] [$num show])}
    MkNumber [::tcl::mathfunc::floor [$num numval]]
}
\end{lstlisting}
\noindent\makebox[\linewidth]{\rule{\linewidth}{0.4pt}}
\noindent\makebox[\linewidth]{\rule{\linewidth}{0.4pt}}
\begin{lstlisting}
reg ceiling ::constcl::ceiling
 
proc ::constcl::ceiling {num} {
    check {number? $num} {NUMBER expected\n([pn] [$num show])}
    MkNumber [::tcl::mathfunc::ceil [$num numval]]
}
\end{lstlisting}
\noindent\makebox[\linewidth]{\rule{\linewidth}{0.4pt}}
\noindent\makebox[\linewidth]{\rule{\linewidth}{0.4pt}}
\begin{lstlisting}
reg truncate ::constcl::truncate
 
proc ::constcl::truncate {num} {
    check {number? $num} {NUMBER expected\n([pn] [$num show])}
    ::if {[$num negative?] ne "#f"} {
        MkNumber [::tcl::mathfunc::ceil [$num numval]]
    } else {
        MkNumber [::tcl::mathfunc::floor [$num numval]]
    }
}
\end{lstlisting}
\noindent\makebox[\linewidth]{\rule{\linewidth}{0.4pt}}
\noindent\makebox[\linewidth]{\rule{\linewidth}{0.4pt}}
\begin{lstlisting}
reg round ::constcl::round
 
proc ::constcl::round {num} {
    check {number? $num} {NUMBER expected\n([pn] [$num show])}
    MkNumber [::tcl::mathfunc::round [$num numval]]
}
\end{lstlisting}
\noindent\makebox[\linewidth]{\rule{\linewidth}{0.4pt}}
\noindent\makebox[\linewidth]{\rule{\linewidth}{0.4pt}}
\begin{lstlisting}
proc ::constcl::rationalize {x y} {
    # TODO
}
\end{lstlisting}
\noindent\makebox[\linewidth]{\rule{\linewidth}{0.4pt}}

\textbf{exp}


\textbf{log}


\textbf{sin}


\textbf{cos}


\textbf{tan}


\textbf{asin}


\textbf{acos}


\textbf{atan}


The mathematical functions \_e<sup>x</sup>\_, natural logarithm, sine, cosine, tangent, arcsine, arccosine, and arctangent are calculated by \texttt{exp}, \texttt{log}, \texttt{sin}, \texttt{cos}, \texttt{tan}, \texttt{asin}, \texttt{acos}, and \texttt{atan}, respectively.

\begin{tabular}{ |l l| }
\hline
\multicolumn{2}{|l|}{exp, log, sin, cos, tan, asin, acos, atan (public)} \\
\hline
num & a number \\
\textit{Returns:} & a number \\
\hline
\end{tabular}

\begin{tabular}{ |l l| }
\hline
\multicolumn{2}{|l|}{(binary) atan (public)} \\
\hline
num1 & a number \\
num2 & a number \\
\textit{Returns:} & a number \\
\hline
\end{tabular}


Example:

\noindent\makebox[\linewidth]{\rule{\linewidth}{0.4pt}}
\begin{lstlisting}
(let ((x (log 2))) (= 2 (exp x)))                         ⇒  #t
(let ((a (/ pi 3))) (let ((s (sin a))) (= a (asin s))))   ⇒  #t
\end{lstlisting}
\noindent\makebox[\linewidth]{\rule{\linewidth}{0.4pt}}
\noindent\makebox[\linewidth]{\rule{\linewidth}{0.4pt}}
\begin{lstlisting}
reg exp ::constcl::exp
 
proc ::constcl::exp {num} {
    check {number? $num} {NUMBER expected\n([pn] [$num show])}
    MkNumber [::tcl::mathfunc::exp [$num numval]]
}
\end{lstlisting}
\noindent\makebox[\linewidth]{\rule{\linewidth}{0.4pt}}
\noindent\makebox[\linewidth]{\rule{\linewidth}{0.4pt}}
\begin{lstlisting}
reg log ::constcl::log
 
proc ::constcl::log {num} {
    check {number? $num} {NUMBER expected\n([pn] [$num show])}
    MkNumber [::tcl::mathfunc::log [$num numval]]
}
\end{lstlisting}
\noindent\makebox[\linewidth]{\rule{\linewidth}{0.4pt}}
\noindent\makebox[\linewidth]{\rule{\linewidth}{0.4pt}}
\begin{lstlisting}
reg sin ::constcl::sin
 
proc ::constcl::sin {num} {
    check {number? $num} {NUMBER expected\n([pn] [$num show])}
    MkNumber [::tcl::mathfunc::sin [$num numval]]
}
\end{lstlisting}
\noindent\makebox[\linewidth]{\rule{\linewidth}{0.4pt}}
\noindent\makebox[\linewidth]{\rule{\linewidth}{0.4pt}}
\begin{lstlisting}
reg cos ::constcl::cos
 
proc ::constcl::cos {num} {
    check {number? $num} {NUMBER expected\n([pn] [$num show])}
    MkNumber [::tcl::mathfunc::cos [$num numval]]
}
\end{lstlisting}
\noindent\makebox[\linewidth]{\rule{\linewidth}{0.4pt}}
\noindent\makebox[\linewidth]{\rule{\linewidth}{0.4pt}}
\begin{lstlisting}
reg tan ::constcl::tan
 
proc ::constcl::tan {num} {
    check {number? $num} {NUMBER expected\n([pn] [$num show])}
    MkNumber [::tcl::mathfunc::tan [$num numval]]
}
\end{lstlisting}
\noindent\makebox[\linewidth]{\rule{\linewidth}{0.4pt}}
\noindent\makebox[\linewidth]{\rule{\linewidth}{0.4pt}}
\begin{lstlisting}
reg asin ::constcl::asin
 
proc ::constcl::asin {num} {
    check {number? $num} {NUMBER expected\n([pn] [$num show])}
    MkNumber [::tcl::mathfunc::asin [$num numval]]
}
\end{lstlisting}
\noindent\makebox[\linewidth]{\rule{\linewidth}{0.4pt}}
\noindent\makebox[\linewidth]{\rule{\linewidth}{0.4pt}}
\begin{lstlisting}
reg acos ::constcl::acos
 
proc ::constcl::acos {num} {
    check {number? $num} {NUMBER expected\n([pn] [$num show])}
    MkNumber [::tcl::mathfunc::acos [$num numval]]
}
\end{lstlisting}
\noindent\makebox[\linewidth]{\rule{\linewidth}{0.4pt}}
\noindent\makebox[\linewidth]{\rule{\linewidth}{0.4pt}}
\begin{lstlisting}
reg atan ::constcl::atan
 
proc ::constcl::atan {args} {
    ::if {[llength $args] == 1} {
        check {number? $num} {NUMBER expected\n([pn] [$num show])}
        MkNumber [::tcl::mathfunc::atan [$num numval]]
    } else {
        lassign $args num1 num2
        check {number? $num1} {NUMBER expected\n([pn] [$num1 show])}
        check {number? $num2} {NUMBER expected\n([pn] [$num2 show])}
        MkNumber [::tcl::mathfunc::atan2 [$num1 numval] [$num2 numval]]
    }
}
\end{lstlisting}
\noindent\makebox[\linewidth]{\rule{\linewidth}{0.4pt}}

\textbf{sqrt}


\texttt{sqrt} calculates the square root.

\begin{tabular}{ |l l| }
\hline
\multicolumn{2}{|l|}{sqrt (public)} \\
\hline
num & a number \\
\textit{Returns:} & a number \\
\hline
\end{tabular}

\noindent\makebox[\linewidth]{\rule{\linewidth}{0.4pt}}
\begin{lstlisting}
reg sqrt ::constcl::sqrt
 
proc ::constcl::sqrt {num} {
    check {number? $num} {NUMBER expected\n([pn] [$num show])}
    MkNumber [::tcl::mathfunc::sqrt [$num numval]]
}
\end{lstlisting}
\noindent\makebox[\linewidth]{\rule{\linewidth}{0.4pt}}

\textbf{expt}


\texttt{expt} calculates the \_x\_ to the power of \_y\_, or \_x<sup>y</sup>\_.

\begin{tabular}{ |l l| }
\hline
\multicolumn{2}{|l|}{expt (public)} \\
\hline
num1 & a number \\
num2 & a number \\
\textit{Returns:} & a number \\
\hline
\end{tabular}

\noindent\makebox[\linewidth]{\rule{\linewidth}{0.4pt}}
\begin{lstlisting}
reg expt ::constcl::expt
 
proc ::constcl::expt {num1 num2} {
    check {number? $num1} {NUMBER expected\n([pn] [$num1 show] [$num2 show])}
    check {number? $num2} {NUMBER expected\n([pn] [$num1 show] [$num2 show])}
    MkNumber [::tcl::mathfunc::pow [$num1 numval] [$num2 numval]]
}
\end{lstlisting}
\noindent\makebox[\linewidth]{\rule{\linewidth}{0.4pt}}
\noindent\makebox[\linewidth]{\rule{\linewidth}{0.4pt}}
\begin{lstlisting}
proc ::constcl::make-rectangular {x1 x2} {
    # TODO
}
\end{lstlisting}
\noindent\makebox[\linewidth]{\rule{\linewidth}{0.4pt}}
\noindent\makebox[\linewidth]{\rule{\linewidth}{0.4pt}}
\begin{lstlisting}
proc ::constcl::make-polar {x3 x4} {
    # TODO
}
\end{lstlisting}
\noindent\makebox[\linewidth]{\rule{\linewidth}{0.4pt}}
\noindent\makebox[\linewidth]{\rule{\linewidth}{0.4pt}}
\begin{lstlisting}
proc ::constcl::real-part {z} {
    # TODO
}
\end{lstlisting}
\noindent\makebox[\linewidth]{\rule{\linewidth}{0.4pt}}
\noindent\makebox[\linewidth]{\rule{\linewidth}{0.4pt}}
\begin{lstlisting}
proc ::constcl::imag-part {z} {
    # TODO
}
\end{lstlisting}
\noindent\makebox[\linewidth]{\rule{\linewidth}{0.4pt}}
\noindent\makebox[\linewidth]{\rule{\linewidth}{0.4pt}}
\begin{lstlisting}
proc ::constcl::magnitude {z} {
    # TODO
}
\end{lstlisting}
\noindent\makebox[\linewidth]{\rule{\linewidth}{0.4pt}}
\noindent\makebox[\linewidth]{\rule{\linewidth}{0.4pt}}
\begin{lstlisting}
proc ::constcl::angle {z} {
    # TODO
}
\end{lstlisting}
\noindent\makebox[\linewidth]{\rule{\linewidth}{0.4pt}}
\noindent\makebox[\linewidth]{\rule{\linewidth}{0.4pt}}
\begin{lstlisting}
proc ::constcl::exact->inexact {z} {
    # TODO
}
\end{lstlisting}
\noindent\makebox[\linewidth]{\rule{\linewidth}{0.4pt}}
\noindent\makebox[\linewidth]{\rule{\linewidth}{0.4pt}}
\begin{lstlisting}
proc ::constcl::inexact->exact {z} {
    # TODO
}
\end{lstlisting}
\noindent\makebox[\linewidth]{\rule{\linewidth}{0.4pt}}

\textbf{number->string}


The procedures \texttt{number->string} and \texttt{string->number} convert between number and string with optional radix conversion.

\begin{tabular}{ |l l| }
\hline
\multicolumn{2}{|l|}{number->string (public)} \\
\hline
num & a number \\
?radix? & a number \\
\textit{Returns:} & a string \\
\hline
\end{tabular}


Example:

\noindent\makebox[\linewidth]{\rule{\linewidth}{0.4pt}}
\begin{lstlisting}
(number->string 23)      ⇒  "23"
(number->string 23 8)    ⇒  "27"
(number->string 23 16)   ⇒  "17"
\end{lstlisting}
\noindent\makebox[\linewidth]{\rule{\linewidth}{0.4pt}}
\noindent\makebox[\linewidth]{\rule{\linewidth}{0.4pt}}
\begin{lstlisting}
reg number->string ::constcl::number->string
 
proc ::constcl::number->string {num args} {
        check {number? $num} {NUMBER expected\n([pn] [$num show])}
        return [MkString [$num numval]]
    } else {
        lassign $args radix
        check {number? $num} {NUMBER expected\n([pn] [$num show])}
        check {number? $radix} {NUMBER expected\n([pn] [$num show] [$radix show])}
            return [MkString [$num numval]]
        } else {
            return [MkString [base [$radix numval] [$num numval]]]
        }
    }
}
 
proc base {base number} {
    set negative [regexp ^-(.+) $number -> number]
    set res {}
    while {$number} {
        set digit [expr {$number % $base}]
        set res [lindex $digits $digit]$res
        set number [expr {$number / $base}]
    }
    ::if $negative {set res -$res}
    set res
}
\end{lstlisting}
\noindent\makebox[\linewidth]{\rule{\linewidth}{0.4pt}}

\textbf{string->number}


As with \texttt{number->string}, above.

\begin{tabular}{ |l l| }
\hline
\multicolumn{2}{|l|}{string->number (public)} \\
\hline
str & a string \\
?radix? & a number \\
\textit{Returns:} & a number \\
\hline
\end{tabular}


Example:

\noindent\makebox[\linewidth]{\rule{\linewidth}{0.4pt}}
\begin{lstlisting}
(string->number "23")        ⇒  23
(string->number "27" 8)      ⇒  23
(string->number "17" 16)     ⇒  23
\end{lstlisting}
\noindent\makebox[\linewidth]{\rule{\linewidth}{0.4pt}}
\noindent\makebox[\linewidth]{\rule{\linewidth}{0.4pt}}
\begin{lstlisting}
reg string->number ::constcl::string->number
 
proc ::constcl::string->number {str args} {
        check {string? $str} {STRING expected\n([pn] [$str show])}
        return [MkNumber [$str value]]
    } else {
        lassign $args radix
        check {string? $str} {STRING expected\n([pn] [$str show])}
            return [MkNumber [$str value]]
        } else {
            return [MkNumber [frombase [$radix numval] [$str value]]]
        }
    }
}
 
proc frombase {base number} {
    set negative [regexp ^-(.+) $number -> number]
    foreach digit [split $number {}] {
        set decimalvalue [lsearch $digits $digit]
            ::error "bad digit $decimalvalue for base $base"
        }
        set res [expr {$res * $base + $decimalvalue}]
    }
    ::if $negative {set res -$res}
    set res
}
\end{lstlisting}
\noindent\makebox[\linewidth]{\rule{\linewidth}{0.4pt}}
\subsection{Booleans}
\label{booleans}

Booleans are logic values, either true (\texttt{\#t}) or false (\texttt{\#f}). All predicates (procedures whose name end with -?) return boolean values. The conditional \texttt{if} operator considers all values except for \texttt{\#f} to be true.


\textbf{Boolean} class

\noindent\makebox[\linewidth]{\rule{\linewidth}{0.4pt}}
\begin{lstlisting}
oo::class create ::constcl::Boolean {
    superclass ::constcl::NIL
    variable bvalue
    constructor {v} {
        ::if {$v ni {#t #f}} {
            ::error "bad boolean value $v"
        }
        set bvalue $v
    }
    method mkconstant {} {}
    method constant {} {return 1}
    method bvalue {} { set bvalue }
    method value {} { set bvalue }
    method write {handle} { puts -nonewline $handle [my bvalue] }
    method display {} { puts -nonewline [my bvalue] }
    method show {} {set bvalue}
}
 
proc ::constcl::MkBoolean {v} {
    foreach instance [info class instances ::constcl::Boolean] {
        ::if {[$instance bvalue] eq $v} {
            return $instance
        }
    }
    return [::constcl::Boolean new $v]
}
\end{lstlisting}
\noindent\makebox[\linewidth]{\rule{\linewidth}{0.4pt}}

\textbf{boolean?}


The \texttt{boolean?} predicate recognizes a Boolean by type.

\begin{tabular}{ |l l| }
\hline
\multicolumn{2}{|l|}{boolean? (public)} \\
\hline
val & a Lisp value \\
\textit{Returns:} & a boolean \\
\hline
\end{tabular}

\noindent\makebox[\linewidth]{\rule{\linewidth}{0.4pt}}
\begin{lstlisting}
reg boolean? ::constcl::boolean?
 
proc ::constcl::boolean? {val} {
    ::if {[info object isa typeof $val ::constcl::Boolean]} {
        return #t
    } elseif {[info object isa typeof [interp alias {} $val] ::constcl::Boolean]} {
        return #t
    } else {
        return #f
    }
}
\end{lstlisting}
\noindent\makebox[\linewidth]{\rule{\linewidth}{0.4pt}}

\textbf{not}


The only operation on booleans: \texttt{not}, or logical negation.

\begin{tabular}{ |l l| }
\hline
\multicolumn{2}{|l|}{not (public)} \\
\hline
val & a Lisp value \\
\textit{Returns:} & a boolean \\
\hline
\end{tabular}


Example:

\noindent\makebox[\linewidth]{\rule{\linewidth}{0.4pt}}
\begin{lstlisting}
(not #f)    ⇒  #t   ; the only argument that returns #t, all others return #f
(not nil)   ⇒  #f   ; see?
\end{lstlisting}
\noindent\makebox[\linewidth]{\rule{\linewidth}{0.4pt}}
\noindent\makebox[\linewidth]{\rule{\linewidth}{0.4pt}}
\begin{lstlisting}
reg not ::constcl::not
 
proc ::constcl::not {val} {
    ::if {[$val bvalue] eq "#f"} {
        return #t
    } else {
        return #f
    }
}
\end{lstlisting}
\noindent\makebox[\linewidth]{\rule{\linewidth}{0.4pt}}
\subsection{Characters}
\label{characters}

Characters are any Unicode printing character, and also space and newline space characters.


\textbf{Char} class

\noindent\makebox[\linewidth]{\rule{\linewidth}{0.4pt}}
\begin{lstlisting}
oo::class create ::constcl::Char {
    superclass ::constcl::NIL
    variable value
    constructor {v} {
        ::if {[regexp {^#\\([[:graph:]]|space|newline)$} $v]} {
            set value $v
        } else {
            ::if {$v eq "#\\ "} {
                set value #\\space
            } elseif {$v eq "#\\\n"} {
                set value #\\newline
            } else {
                ::error "CHAR expected\n$v"
            }
        }
    }
    method char {} {
        switch $value {
            "#\\space" {
                return " "
            }
            "#\\newline" {
                return "\n"
            }
            default {
                return [::string index [my value] 2]
            }
        }
    }
    method alphabetic? {} {
        ::if {[::string is alpha -strict [my char]]} {
            return #t
        } else {
            return #f
        }
    }
    method numeric? {} {
        ::if {[::string is digit -strict [my char]]} {
            return #t
        } else {
            return #f
        }
    }
    method whitespace? {} {
        ::if {[::string is space -strict [my char]]} {
            return #t
        } else {
            return #f
        }
    }
    method upper-case? {} {
        ::if {[::string is upper -strict [my char]]} {
            return #t
        } else {
            return #f
        }
    }
    method lower-case? {} {
        ::if {[::string is lower -strict [my char]]} {
            return #t
        } else {
            return #f
        }
    }
    method mkconstant {} {}
    method constant {} {return 1}
    method value {} {return $value}
    method write {handle} { puts -nonewline $handle $value }
    method display {} { puts -nonewline [my char] }
    method show {} {set value}
}
 
proc ::constcl::MkChar {v} {
    ::if {[regexp -nocase {^#\\(space|newline)$} $v]} {
        set v [::string tolower $v]
    }
    foreach instance [info class instances ::constcl::Char] {
        ::if {[$instance value] eq $v} {
            return $instance
        }
    }
    return [::constcl::Char new $v]
}
\end{lstlisting}
\noindent\makebox[\linewidth]{\rule{\linewidth}{0.4pt}}

\textbf{char?}


\texttt{char?} recognizes Char values by type.

\begin{tabular}{ |l l| }
\hline
\multicolumn{2}{|l|}{char? (public)} \\
\hline
val & a Lisp value \\
\textit{Returns:} & a boolean \\
\hline
\end{tabular}

\noindent\makebox[\linewidth]{\rule{\linewidth}{0.4pt}}
\begin{lstlisting}
reg char? ::constcl::char?
 
proc ::constcl::char? {val} {
    ::if {[info object isa typeof $val ::constcl::Char]} {
        return #t
    } elseif {[info object isa typeof [interp alias {} $val] ::constcl::Char]} {
        return #t
    } else {
        return #f
    }
}
\end{lstlisting}
\noindent\makebox[\linewidth]{\rule{\linewidth}{0.4pt}}

\textbf{char=?}


\textbf{char<?}


\textbf{char>?}


\textbf{char<=?}


\textbf{char>=?}


\texttt{char=?}, \texttt{char<?}, \texttt{char>?}, \texttt{char<=?}, and \texttt{char>=?} compare character values. They only compare two characters at a time.

\begin{tabular}{ |l l| }
\hline
\multicolumn{2}{|l|}{char=?, char<?, char>?, char<=?, char>=? (public)} \\
\hline
char1 & a character \\
char2 & a character \\
\textit{Returns:} & a boolean \\
\hline
\end{tabular}

\noindent\makebox[\linewidth]{\rule{\linewidth}{0.4pt}}
\begin{lstlisting}
reg char=? ::constcl::char=?
 
proc ::constcl::char=? {char1 char2} {
    check {char? $char1} {CHAR expected\n([pn] [$char1 show] [$char2 show])}
    check {char? $char2} {CHAR expected\n([pn] [$char1 show] [$char2 show])}
    ::if {$char1 eq $char2} {
        return #t
    } else {
        return #f
    }
}
\end{lstlisting}
\noindent\makebox[\linewidth]{\rule{\linewidth}{0.4pt}}
\noindent\makebox[\linewidth]{\rule{\linewidth}{0.4pt}}
\begin{lstlisting}
reg char<? ::constcl::char<?
 
proc ::constcl::char<? {char1 char2} {
    check {char? $char1} {CHAR expected\n([pn] [$char1 show] [$char2 show])}
    check {char? $char2} {CHAR expected\n([pn] [$char1 show] [$char2 show])}
    ::if {[$char1 char] < [$char2 char]} {
        return #t
    } else {
        return #f
    }
}
\end{lstlisting}
\noindent\makebox[\linewidth]{\rule{\linewidth}{0.4pt}}
\noindent\makebox[\linewidth]{\rule{\linewidth}{0.4pt}}
\begin{lstlisting}
reg char>? ::constcl::char>?
 
proc ::constcl::char>? {char1 char2} {
    check {char? $char1} {CHAR expected\n([pn] [$char1 show] [$char2 show])}
    check {char? $char2} {CHAR expected\n([pn] [$char1 show] [$char2 show])}
    ::if {[$char1 char] > [$char2 char]} {
        return #t
    } else {
        return #f
    }
}
\end{lstlisting}
\noindent\makebox[\linewidth]{\rule{\linewidth}{0.4pt}}
\noindent\makebox[\linewidth]{\rule{\linewidth}{0.4pt}}
\begin{lstlisting}
reg char<=? ::constcl::char<=?
 
proc ::constcl::char<=? {char1 char2} {
    check {char? $char1} {CHAR expected\n([pn] [$char1 show] [$char2 show])}
    check {char? $char2} {CHAR expected\n([pn] [$char1 show] [$char2 show])}
    ::if {[$char1 char] <= [$char2 char]} {
        return #t
    } else {
        return #f
    }
}
\end{lstlisting}
\noindent\makebox[\linewidth]{\rule{\linewidth}{0.4pt}}
\noindent\makebox[\linewidth]{\rule{\linewidth}{0.4pt}}
\begin{lstlisting}
reg char>=? ::constcl::char>=?
 
proc ::constcl::char>=? {char1 char2} {
    check {char? $char1} {CHAR expected\n([pn] [$char1 show] [$char2 show])}
    check {char? $char2} {CHAR expected\n([pn] [$char1 show] [$char2 show])}
    ::if {[$char1 char] >= [$char2 char]} {
        return #t
    } else {
        return #f
    }
}
\end{lstlisting}
\noindent\makebox[\linewidth]{\rule{\linewidth}{0.4pt}}

\textbf{char-ci=?}


\textbf{char-ci<?}


\textbf{char-ci>?}


\textbf{char-ci<=?}


\textbf{char-ci>=?}


\texttt{char-ci=?}, \texttt{char-ci<?}, \texttt{char-ci>?}, \texttt{char-ci<=?}, and \texttt{char-ci>=?} compare character values in a case insensitive manner. They only compare two characters at a time.

\begin{tabular}{ |l l| }
\hline
\multicolumn{2}{|l|}{char-ci=?, char-ci<?, char-ci>?, char-ci<=?, char-ci>=? (public)} \\
\hline
char1 & a character \\
char2 & a character \\
\textit{Returns:} & a boolean \\
\hline
\end{tabular}

\noindent\makebox[\linewidth]{\rule{\linewidth}{0.4pt}}
\begin{lstlisting}
reg char-ci=? ::constcl::char-ci=?
 
proc ::constcl::char-ci=? {char1 char2} {
    check {char? $char1} {CHAR expected\n([pn] [$char1 show] [$char2 show])}
    check {char? $char2} {CHAR expected\n([pn] [$char1 show] [$char2 show])}
    ::if {[::string tolower [$char1 char]] eq [::string tolower [$char2 char]]} {
        return #t
    } else {
        return #f
    }
}
\end{lstlisting}
\noindent\makebox[\linewidth]{\rule{\linewidth}{0.4pt}}
\noindent\makebox[\linewidth]{\rule{\linewidth}{0.4pt}}
\begin{lstlisting}
reg char-ci<? ::constcl::char-ci<?
 
proc ::constcl::char-ci<? {char1 char2} {
    check {char? $char1} {CHAR expected\n([pn] [$char1 show] [$char2 show])}
    check {char? $char2} {CHAR expected\n([pn] [$char1 show] [$char2 show])}
    ::if {[::string tolower [$char1 char]] < [::string tolower [$char2 char]]} {
        return #t
    } else {
        return #f
    }
}
\end{lstlisting}
\noindent\makebox[\linewidth]{\rule{\linewidth}{0.4pt}}
\noindent\makebox[\linewidth]{\rule{\linewidth}{0.4pt}}
\begin{lstlisting}
reg char-ci>? ::constcl::char-ci>?
 
proc ::constcl::char-ci>? {char1 char2} {
    check {char? $char1} {CHAR expected\n([pn] [$char1 show] [$char2 show])}
    check {char? $char2} {CHAR expected\n([pn] [$char1 show] [$char2 show])}
    ::if {[::string tolower [$char1 char]] > [::string tolower [$char2 char]]} {
        return #t
    } else {
        return #f
    }
}
\end{lstlisting}
\noindent\makebox[\linewidth]{\rule{\linewidth}{0.4pt}}
\noindent\makebox[\linewidth]{\rule{\linewidth}{0.4pt}}
\begin{lstlisting}
reg char-ci<=? ::constcl::char-ci<=?
 
proc ::constcl::char-ci<=? {char1 char2} {
    check {char? $char1} {CHAR expected\n([pn] [$char1 show] [$char2 show])}
    check {char? $char2} {CHAR expected\n([pn] [$char1 show] [$char2 show])}
    ::if {[::string tolower [$char1 char]] <= [::string tolower [$char2 char]]} {
        return #t
    } else {
        return #f
    }
}
\end{lstlisting}
\noindent\makebox[\linewidth]{\rule{\linewidth}{0.4pt}}
\noindent\makebox[\linewidth]{\rule{\linewidth}{0.4pt}}
\begin{lstlisting}
reg char-ci>=? ::constcl::char-ci>=?
 
proc ::constcl::char-ci>=? {char1 char2} {
    check {char? $char1} {CHAR expected\n([pn] [$char1 show] [$char2 show])}
    check {char? $char2} {CHAR expected\n([pn] [$char1 show] [$char2 show])}
    ::if {[::string tolower [$char1 char]] >= [::string tolower [$char2 char]]} {
        return #t
    } else {
        return #f
    }
}
\end{lstlisting}
\noindent\makebox[\linewidth]{\rule{\linewidth}{0.4pt}}

\textbf{char-alphabetic}


\textbf{char-numeric}


\textbf{char-whitespace}


\textbf{char-upper-case}


\textbf{char-lower-case}


The predicates \texttt{char-alphabetic}, \texttt{char-numeric}, \texttt{char-whitespace}, \texttt{char-upper-case}, and \texttt{char-lower-case} test a character for these conditions.

\begin{tabular}{ |l l| }
\hline
\multicolumn{2}{|l|}{char-alphabetic?, char-numeric?, char-whitespace?, char-upper-case?, char-lower-case? (public)} \\
\hline
char & a character \\
\textit{Returns:} & a boolean \\
\hline
\end{tabular}

\noindent\makebox[\linewidth]{\rule{\linewidth}{0.4pt}}
\begin{lstlisting}
reg char-alphabetic? ::constcl::char-alphabetic?
 
proc ::constcl::char-alphabetic? {char} {
    check {char? $char} {CHAR expected\n([pn] [$char show])}
    return [$char alphabetic?]
}
\end{lstlisting}
\noindent\makebox[\linewidth]{\rule{\linewidth}{0.4pt}}
\noindent\makebox[\linewidth]{\rule{\linewidth}{0.4pt}}
\begin{lstlisting}
reg char-numeric? ::constcl::char-numeric?
 
proc ::constcl::char-numeric? {char} {
    check {char? $char} {CHAR expected\n([pn] [$char show])}
    return [$char numeric?]
}
\end{lstlisting}
\noindent\makebox[\linewidth]{\rule{\linewidth}{0.4pt}}
\noindent\makebox[\linewidth]{\rule{\linewidth}{0.4pt}}
\begin{lstlisting}
reg char-whitespace? ::constcl::char-whitespace?
 
proc ::constcl::char-whitespace? {char} {
    check {char? $char} {CHAR expected\n([pn] [$char show])}
    return [$char whitespace?]
}
\end{lstlisting}
\noindent\makebox[\linewidth]{\rule{\linewidth}{0.4pt}}
\noindent\makebox[\linewidth]{\rule{\linewidth}{0.4pt}}
\begin{lstlisting}
reg char-upper-case? ::constcl::char-upper-case?
 
proc ::constcl::char-upper-case? {char} {
    check {char? $char} {CHAR expected\n([pn] [$char show])}
    return [$char upper-case?]
}
\end{lstlisting}
\noindent\makebox[\linewidth]{\rule{\linewidth}{0.4pt}}
\noindent\makebox[\linewidth]{\rule{\linewidth}{0.4pt}}
\begin{lstlisting}
reg char-lower-case? ::constcl::char-lower-case?
 
proc ::constcl::char-lower-case? {char} {
    check {char? $char} {CHAR expected\n([pn] [$char show])}
    return [$char lower-case?]
}
\end{lstlisting}
\noindent\makebox[\linewidth]{\rule{\linewidth}{0.4pt}}

\textbf{char->integer}


\textbf{integer->char}


\texttt{char->integer} and \texttt{integer->char} convert between characters and their 16-bit numeric codes.

\begin{tabular}{ |l l| }
\hline
\multicolumn{2}{|l|}{char->integer (public)} \\
\hline
char & a character \\
\textit{Returns:} & an integer \\
\hline
\end{tabular}


Example:

\noindent\makebox[\linewidth]{\rule{\linewidth}{0.4pt}}
\begin{lstlisting}
(char->integer #\A)   =>  65
\end{lstlisting}
\noindent\makebox[\linewidth]{\rule{\linewidth}{0.4pt}}
\noindent\makebox[\linewidth]{\rule{\linewidth}{0.4pt}}
\begin{lstlisting}
reg char->integer
 
proc ::constcl::char->integer {char} {
    return [MkNumber [scan [$char char] %c]]
}
\end{lstlisting}
\noindent\makebox[\linewidth]{\rule{\linewidth}{0.4pt}}
\begin{tabular}{ |l l| }
\hline
\multicolumn{2}{|l|}{integer->char (public)} \\
\hline
int & an integer \\
\textit{Returns:} & a character \\
\hline
\end{tabular}


Example:

\noindent\makebox[\linewidth]{\rule{\linewidth}{0.4pt}}
\begin{lstlisting}
(integer->char 97)   =>  #\a
\end{lstlisting}
\noindent\makebox[\linewidth]{\rule{\linewidth}{0.4pt}}
\noindent\makebox[\linewidth]{\rule{\linewidth}{0.4pt}}
\begin{lstlisting}
reg integer->char
 
proc ::constcl::integer->char {int} {
        return [MkChar #\\newline]
    } elseif {$int == 32} {
        return [MkChar #\\space]
    } else {
        return [MkChar #\\[format %c [$int numval]]]
    }
}
\end{lstlisting}
\noindent\makebox[\linewidth]{\rule{\linewidth}{0.4pt}}

\textbf{char-upcase}


\textbf{char-downcase}


\texttt{char-upcase} and \texttt{char-downcase} alter the case of a character.

\begin{tabular}{ |l l| }
\hline
\multicolumn{2}{|l|}{char-upcase, char-downcase (public)} \\
\hline
char & a character \\
\textit{Returns:} & a character \\
\hline
\end{tabular}

\noindent\makebox[\linewidth]{\rule{\linewidth}{0.4pt}}
\begin{lstlisting}
reg char-upcase ::constcl::char-upcase
 
proc ::constcl::char-upcase {char} {
    check {char? $char} {CHAR expected\n([pn] [$char show])}
    ::if {[::string is alpha -strict [$char char]]} {
        return [MkChar [::string toupper [$char value]]]
    } else {
        return $char
    }
}
\end{lstlisting}
\noindent\makebox[\linewidth]{\rule{\linewidth}{0.4pt}}
\noindent\makebox[\linewidth]{\rule{\linewidth}{0.4pt}}
\begin{lstlisting}
reg char-downcase ::constcl::char-downcase
 
proc ::constcl::char-downcase {char} {
    check {char? $char} {CHAR expected\n([pn] [$char show])}
    ::if {[::string is alpha -strict [$char char]]} {
        return [MkChar [::string tolower [$char value]]]
    } else {
        return $char
    }
}
\end{lstlisting}
\noindent\makebox[\linewidth]{\rule{\linewidth}{0.4pt}}
\subsection{Control}
\label{control}

This section concerns itself with procedures and the application of the same.


A \texttt{Procedure} object is a closure\footnote{See \texttt{https://en.wikipedia.org/wiki/Closure\_(computer\_programming)}}, storing the procedure's parameter list, the body, and the environment that is current when the object is created, i.e. when the procedure is defined.


When a \texttt{Procedure} object is called, the body is evaluated in a new environment where the parameters are given values from the argument list and the outer link goes to the closure environment.


\textbf{Procedure} class

\noindent\makebox[\linewidth]{\rule{\linewidth}{0.4pt}}
\begin{lstlisting}
catch { ::constcl::Procedure destroy }
 
oo::class create ::constcl::Procedure {
    superclass ::constcl::NIL
    variable parms body env
    constructor {p b e} {
        set parms $p         ;# a Lisp list|improper list|symbol denoting parameter names
        set body $b          ;# a Lisp list of expressions under 'begin, or a single expression
        set env $e           ;# the closed over environment
    }
    method value {} {}
    method write {handle} {
        regexp {(\d+)} [self] -> num
        puts -nonewline $handle "#<proc-$num>"
    }
    method display {} {my write}
    method show {} { return [self] }
    method call {args} {
        ::constcl::eval $body [::constcl::Environment new $parms $args $env]
    }
 
}
 
interp alias {} ::constcl::MkProcedure {} ::constcl::Procedure new
\end{lstlisting}
\noindent\makebox[\linewidth]{\rule{\linewidth}{0.4pt}}

\textbf{procedure?}

\begin{tabular}{ |l l| }
\hline
\multicolumn{2}{|l|}{procedure? (public)} \\
\hline
val & a Lisp value \\
\textit{Returns:} & a boolean \\
\hline
\end{tabular}

\noindent\makebox[\linewidth]{\rule{\linewidth}{0.4pt}}
\begin{lstlisting}
reg procedure? ::constcl::procedure?
 
proc ::constcl::procedure? {val} {
    ::if {[info object isa typeof $val ::constcl::Procedure]} {
        return #t
    } elseif {[info object isa typeof [interp alias {} $val] ::constcl::Procedure]} {
        return #t
    } elseif {[::string match "::constcl::*" $val]} {
        return #t
    } else {
        return #f
    }
}
\end{lstlisting}
\noindent\makebox[\linewidth]{\rule{\linewidth}{0.4pt}}

\textbf{apply}


\texttt{apply} applies a procedure to a Lisp list of Lisp arguments.

\begin{tabular}{ |l l| }
\hline
\multicolumn{2}{|l|}{apply (public)} \\
\hline
pr & a procedure \\
vals & a Lisp list of Lisp values \\
\textit{Returns:} & what pr returns \\
\hline
\end{tabular}


Example:

\noindent\makebox[\linewidth]{\rule{\linewidth}{0.4pt}}
\begin{lstlisting}
(apply + (list 2 3))   ⇒  5
\end{lstlisting}
\noindent\makebox[\linewidth]{\rule{\linewidth}{0.4pt}}
\noindent\makebox[\linewidth]{\rule{\linewidth}{0.4pt}}
\begin{lstlisting}
reg apply ::constcl::apply
 
proc ::constcl::apply {pr vals} {
    check {procedure? $pr} {PROCEDURE expected\n([pn] [$pr show] ...)}
    invoke $pr $vals
}
\end{lstlisting}
\noindent\makebox[\linewidth]{\rule{\linewidth}{0.4pt}}

\textbf{map}


\texttt{map} iterates over one or more lists, taking an element from each list to pass to a procedure as an argument. The Lisp list of the results of the invocations is returned.

\begin{tabular}{ |l l| }
\hline
\multicolumn{2}{|l|}{map (public)} \\
\hline
pr & a procedure \\
args & some lists \\
\textit{Returns:} & a Lisp list of Lisp values \\
\hline
\end{tabular}


Example:

\noindent\makebox[\linewidth]{\rule{\linewidth}{0.4pt}}
\begin{lstlisting}
\end{lstlisting}
\noindent\makebox[\linewidth]{\rule{\linewidth}{0.4pt}}
\noindent\makebox[\linewidth]{\rule{\linewidth}{0.4pt}}
\begin{lstlisting}
reg map ::constcl::map
 
proc ::constcl::map {pr args} {
    check {procedure? $pr} {PROCEDURE expected\n([pn] [$pr show] ...)}
    set arglists $args
        lset arglists $i [splitlist [lindex $arglists $i]]
    }
    set res {}
        set arguments {}
            lappend arguments [lindex $arglists $arg $item]
        }
        lappend res [invoke $pr [list {*}$arguments]]
    }
    return [list {*}$res]
}
\end{lstlisting}
\noindent\makebox[\linewidth]{\rule{\linewidth}{0.4pt}}

\textbf{for-each}


\texttt{for-each} iterates over one or more lists, taking an element from each list to pass to a procedure as an argument. The empty list is returned.

\begin{tabular}{ |l l| }
\hline
\multicolumn{2}{|l|}{for-each (public)} \\
\hline
pr & a procedure \\
args & some lists \\
\textit{Returns:} & the empty list \\
\hline
\end{tabular}


Example: (from R5RS; must be pasted as a oneliner for the ConsTcl repl to stomach it.)

\noindent\makebox[\linewidth]{\rule{\linewidth}{0.4pt}}
\begin{lstlisting}
(let ((v (make-vector 5)))
  (for-each (lambda (i)
              (vector-set! v i (* i i)))
\end{lstlisting}
\noindent\makebox[\linewidth]{\rule{\linewidth}{0.4pt}}
\noindent\makebox[\linewidth]{\rule{\linewidth}{0.4pt}}
\begin{lstlisting}
reg for-each ::constcl::for-each
 
proc ::constcl::for-each {proc args} {
    check {procedure? $proc} {PROCEDURE expected\n([pn] [$proc show] ...)}
    set arglists $args
        lset arglists $i [splitlist [lindex $arglists $i]]
    }
        set arguments {}
            lappend arguments [lindex $arglists $arg $item]
        }
        invoke $proc [list {*}$arguments]
    }
    return [list]
}
\end{lstlisting}
\noindent\makebox[\linewidth]{\rule{\linewidth}{0.4pt}}
\noindent\makebox[\linewidth]{\rule{\linewidth}{0.4pt}}
\begin{lstlisting}
proc ::constcl::force {promise} {
    # TODO
}
\end{lstlisting}
\noindent\makebox[\linewidth]{\rule{\linewidth}{0.4pt}}
\noindent\makebox[\linewidth]{\rule{\linewidth}{0.4pt}}
\begin{lstlisting}
proc ::constcl::call-with-current-continuation {proc} {
    # TODO
}
\end{lstlisting}
\noindent\makebox[\linewidth]{\rule{\linewidth}{0.4pt}}
\noindent\makebox[\linewidth]{\rule{\linewidth}{0.4pt}}
\begin{lstlisting}
proc ::constcl::values {args} {
    # TODO
}
\end{lstlisting}
\noindent\makebox[\linewidth]{\rule{\linewidth}{0.4pt}}
\noindent\makebox[\linewidth]{\rule{\linewidth}{0.4pt}}
\begin{lstlisting}
proc ::constcl::call-with-values {producer consumer} {
    # TODO
}
\end{lstlisting}
\noindent\makebox[\linewidth]{\rule{\linewidth}{0.4pt}}
\noindent\makebox[\linewidth]{\rule{\linewidth}{0.4pt}}
\begin{lstlisting}
proc ::constcl::dynamic-wind {before thunk after} {
    # TODO
}
\end{lstlisting}
\noindent\makebox[\linewidth]{\rule{\linewidth}{0.4pt}}
\subsection{Input and output}
\label{input-and-output}
\noindent\makebox[\linewidth]{\rule{\linewidth}{0.4pt}}
\begin{lstlisting}
catch { Port destroy }
 
oo::class create Port {
    variable handle
    constructor {args} {
        if {[llength $args]} {
            lassign $args handle
        } else {
            set handle #NIL
        }
    }
    method handle {} {set handle}
    method close {} {
        close $handle
        set handle #NIL
    }
}
 
oo::class create InputPort {
    superclass Port
    variable handle
    method open {name} {
        try {
            set handle [open $name "r"]
        } on error {} {
            set handle #NIL
            return -1
        }
    }
}
 
oo::class create OutputPort {
    superclass Port
    variable handle
    method open {name} {
        try {
            set handle [open $name "w"]
        } on error {} {
            set handle #NIL
            return -1
        }
    }
}
 
interp alias {} ::constcl::MkInputPort {} InputPort new
interp alias {} ::constcl::MkOutputPort {} OutputPort new
 
set ::constcl::Input_port [::constcl::MkInputPort stdin]
set ::constcl::Output_port [::constcl::MkOutputPort stdout]
 
proc ::constcl::port? {val} {
    ::if {[info object isa typeof $val ::constcl::Port]} {
        return #t
    } elseif {[info object isa typeof [interp alias {} $val] ::constcl::Port]} {
        return #t
    } else {
        return #f
    }
}
\end{lstlisting}
\noindent\makebox[\linewidth]{\rule{\linewidth}{0.4pt}}
\noindent\makebox[\linewidth]{\rule{\linewidth}{0.4pt}}
\begin{lstlisting}
proc ::constcl::call-with-input-file {string proc} {
    # TODO
}
\end{lstlisting}
\noindent\makebox[\linewidth]{\rule{\linewidth}{0.4pt}}
\noindent\makebox[\linewidth]{\rule{\linewidth}{0.4pt}}
\begin{lstlisting}
proc ::constcl::call-with-output-file {string proc} {
    # TODO
}
\end{lstlisting}
\noindent\makebox[\linewidth]{\rule{\linewidth}{0.4pt}}
\noindent\makebox[\linewidth]{\rule{\linewidth}{0.4pt}}
\begin{lstlisting}
proc ::constcl::input-port? {obj} {
    ::if {[info object isa typeof $val ::constcl::InputPort]} {
        return #t
    } elseif {[info object isa typeof [interp alias {} $val] ::constcl::InputPort]} {
        return #t
    } else {
        return #f
    }
}
\end{lstlisting}
\noindent\makebox[\linewidth]{\rule{\linewidth}{0.4pt}}
\noindent\makebox[\linewidth]{\rule{\linewidth}{0.4pt}}
\begin{lstlisting}
proc ::constcl::output-port? {obj} {
    ::if {[info object isa typeof $val ::constcl::OutputPort]} {
        return #t
    } elseif {[info object isa typeof [interp alias {} $val] ::constcl::OutputPort]} {
        return #t
    } else {
        return #f
    }
}
\end{lstlisting}
\noindent\makebox[\linewidth]{\rule{\linewidth}{0.4pt}}
\noindent\makebox[\linewidth]{\rule{\linewidth}{0.4pt}}
\begin{lstlisting}
proc ::constcl::current-input-port {} {
    return $::constcl::Input_port
}
\end{lstlisting}
\noindent\makebox[\linewidth]{\rule{\linewidth}{0.4pt}}
\noindent\makebox[\linewidth]{\rule{\linewidth}{0.4pt}}
\begin{lstlisting}
proc ::constcl::current-output-port {} {
    return $::constcl::Output_port
}
\end{lstlisting}
\noindent\makebox[\linewidth]{\rule{\linewidth}{0.4pt}}
\noindent\makebox[\linewidth]{\rule{\linewidth}{0.4pt}}
\begin{lstlisting}
proc ::constcl::with-input-from-file {string thunk} {
    # TODO
}
\end{lstlisting}
\noindent\makebox[\linewidth]{\rule{\linewidth}{0.4pt}}
\noindent\makebox[\linewidth]{\rule{\linewidth}{0.4pt}}
\begin{lstlisting}
proc ::constcl::with-output-to-file {string thunk} {
    # TODO
}
\end{lstlisting}
\noindent\makebox[\linewidth]{\rule{\linewidth}{0.4pt}}
\noindent\makebox[\linewidth]{\rule{\linewidth}{0.4pt}}
\begin{lstlisting}
reg open-input-file
 
proc ::constcl::open-input-file {filename} {
    set p [MkInputPort]
    $p open $filename
    ::if {[$p handle] eq "#NIL"} {
        error "open-input-file: could not open file $filename"
    }
    return $p
}
\end{lstlisting}
\noindent\makebox[\linewidth]{\rule{\linewidth}{0.4pt}}
\noindent\makebox[\linewidth]{\rule{\linewidth}{0.4pt}}
\begin{lstlisting}
reg open-output-file
 
proc ::constcl::open-output-file {filename} {
    ::if {[file exists $filename]} {
        error "open-output-file: file already exists $filename"
    }
    set p [MkOutputPort]
    $p open $filename
    ::if {[$p handle] eq "#NIL"} {
        error "open-output-file: could not open file $filename"
    }
    return $p
}
\end{lstlisting}
\noindent\makebox[\linewidth]{\rule{\linewidth}{0.4pt}}
\noindent\makebox[\linewidth]{\rule{\linewidth}{0.4pt}}
\begin{lstlisting}
proc ::constcl::close-input-port {port} {
    ::if {[$port handle] eq "stdin"} {
        error "don't close the standard input port"
    }
    $port close
}
\end{lstlisting}
\noindent\makebox[\linewidth]{\rule{\linewidth}{0.4pt}}
\noindent\makebox[\linewidth]{\rule{\linewidth}{0.4pt}}
\begin{lstlisting}
proc ::constcl::close-output-port {port} {
    ::if {[$port handle] eq "stdout"} {
        error "don't close the standard output port"
    }
    $port close
}
\end{lstlisting}
\noindent\makebox[\linewidth]{\rule{\linewidth}{0.4pt}}

\texttt{read} is implemented in the read (see page \pageref{read}) section.

\noindent\makebox[\linewidth]{\rule{\linewidth}{0.4pt}}
\begin{lstlisting}
proc ::constcl::__read {args} {
    ::if {[llength $args]} {
    } else {
        set new_port $::constcl::Input_port
    }
    set old_port $::constcl::Input_port
    set ::constcl::Input_port $new_port
    set n [xread]
    set ::constcl::Input_port $old_port
    return $n
}
\end{lstlisting}
\noindent\makebox[\linewidth]{\rule{\linewidth}{0.4pt}}
\noindent\makebox[\linewidth]{\rule{\linewidth}{0.4pt}}
\begin{lstlisting}
proc ::constcl::read-char {args} {
    # TODO
}
\end{lstlisting}
\noindent\makebox[\linewidth]{\rule{\linewidth}{0.4pt}}
\noindent\makebox[\linewidth]{\rule{\linewidth}{0.4pt}}
\begin{lstlisting}
proc ::constcl::peek-char {args} {
    # TODO
}
\end{lstlisting}
\noindent\makebox[\linewidth]{\rule{\linewidth}{0.4pt}}
\noindent\makebox[\linewidth]{\rule{\linewidth}{0.4pt}}
\begin{lstlisting}
proc ::constcl::char-ready? {args} {
    # TODO
}
\end{lstlisting}
\noindent\makebox[\linewidth]{\rule{\linewidth}{0.4pt}}

\texttt{write} is implemented in the write (see page \pageref{write}) section.


\texttt{display} is implemented in the write section.

\noindent\makebox[\linewidth]{\rule{\linewidth}{0.4pt}}
\begin{lstlisting}
reg newline
 
proc ::constcl::newline {args} {
    ::if {[llength $args]} {
        lassign $args port
    } else {
        set port [current-output-port]
    }
    write #\\newline $port
}
\end{lstlisting}
\noindent\makebox[\linewidth]{\rule{\linewidth}{0.4pt}}
\noindent\makebox[\linewidth]{\rule{\linewidth}{0.4pt}}
\begin{lstlisting}
proc ::constcl::write-char {args} {
    # TODO
}
\end{lstlisting}
\noindent\makebox[\linewidth]{\rule{\linewidth}{0.4pt}}

\texttt{--load} is a raw port of the S9fES implementation. \texttt{----load} is my original straight-Tcl version. \texttt{load} is my ConsTcl mix of Scheme calls and Tcl syntax.

\noindent\makebox[\linewidth]{\rule{\linewidth}{0.4pt}}
\begin{lstlisting}
proc ::constcl::--load {filename} {
    set new_port [MkInputPort]
    $new_port open $filename
    if {[$new_port handle] eq "#NIL"} {
        return -1
    }
    set ::constcl::File_list [cons [MkString $filename] $::constcl::File_list]
    set save_env $env
    set env ::constcl::global_env
    set outer_loading [$::constcl::S_loading cdr]
    set-cdr! ::constcl::S_loading #t
    set old_port $::constcl::Input_port
    set outer_lno $::constcl::Line_no
    set ::constcl::Line_no 1
    while true {
        set ::constcl::Input_port $new_port
        set n [xread]
        set ::constcl::Input_port $old_port
        ::if {$n == $::constcl::END_OF_FILE} {
            break
        }
        set n [eval $n $env]
    }
    $new_port close
    set $::constcl::Line_no $outer_lno
    set-cdr! ::constcl::S_loading $outer_loading
    set ::constcl::File_list [cdr $::constcl::File_list]
    set env $save_env
}
 
proc ::constcl::----load {filename} {
    set f [open $filename]
    set src [::read $f]
    close $f
    set ib [::constcl::IB new $src]
    while {[$ib first] ne {}} {
        eval [parse $ib]
    }
}
 
proc ::constcl::load {filename} {
    set p [open-input-file $filename]
    set n [read $p]
    while {$n ne "#EOF"} {
        eval $n ::constcl::global_env
        set n [read $p]
    }
    close-input-port $p
}
\end{lstlisting}
\noindent\makebox[\linewidth]{\rule{\linewidth}{0.4pt}}
\noindent\makebox[\linewidth]{\rule{\linewidth}{0.4pt}}
\begin{lstlisting}
proc ::constcl::transcript-on {filename} {
    # TODO
}
\end{lstlisting}
\noindent\makebox[\linewidth]{\rule{\linewidth}{0.4pt}}
\noindent\makebox[\linewidth]{\rule{\linewidth}{0.4pt}}
\begin{lstlisting}
proc ::constcl::transcript-off {} {
    # TODO
}
\end{lstlisting}
\noindent\makebox[\linewidth]{\rule{\linewidth}{0.4pt}}
\subsection{Pairs and lists}
\label{pairs-and-lists}

List processing is another of Lisp's great strengths.


\textbf{Pair} class

\noindent\makebox[\linewidth]{\rule{\linewidth}{0.4pt}}
\begin{lstlisting}
catch { ::constcl::Pair destroy }
 
oo::class create ::constcl::Pair {
    superclass ::constcl::NIL
    variable car cdr constant
    constructor {a d} {
        set car $a
        set cdr $d
    }
    method name {} {} ;# for eval to call when dealing with an application form
    method value {} {my show}
    method car {} { set car }
    method cdr {} { set cdr }
    method set-car! {val} {
        ::constcl::check {my mutable?} {Can't modify a constant pair}
        set car $val
        self
    }
    method set-cdr! {val} {
        ::constcl::check {my mutable?} {Can't modify a constant pair}
        set cdr $val
        self
    }
    method mkconstant {} {set constant 1}
    method constant {} {return $constant}
    method mutable? {} {expr {$constant?"#f":"#t"}}
    method write {handle} {
        puts -nonewline $handle "("
        ::constcl::write-pair $handle [self]
        puts -nonewline $handle ")"
    }
    method display {} { [my write] }
    method show {} {format "(%s)" [::constcl::show-pair [self]]}
}
 
 
interp alias {} ::constcl::MkPair {} ::constcl::Pair new
\end{lstlisting}
\noindent\makebox[\linewidth]{\rule{\linewidth}{0.4pt}}

\textbf{pair?}

\begin{tabular}{ |l l| }
\hline
\multicolumn{2}{|l|}{pair? (public)} \\
\hline
val & a Lisp value \\
\textit{Returns:} & a boolean \\
\hline
\end{tabular}

\noindent\makebox[\linewidth]{\rule{\linewidth}{0.4pt}}
\begin{lstlisting}
reg pair? ::constcl::pair?
 
proc ::constcl::pair? {val} {
    ::if {[info object isa typeof $val ::constcl::Pair]} {
        return #t
    } elseif {[info object isa typeof [interp alias {} $val] ::constcl::Pair]} {
        return #t
    } else {
        return #f
    }
}
\end{lstlisting}
\noindent\makebox[\linewidth]{\rule{\linewidth}{0.4pt}}

\textbf{show-pair}


Helper procedure to make a string representation of a list.

\begin{tabular}{ |l l| }
\hline
\multicolumn{2}{|l|}{show-pair (internal)} \\
\hline
pair & a pair \\
\textit{Returns:} & a Tcl string \\
\hline
\end{tabular}

\noindent\makebox[\linewidth]{\rule{\linewidth}{0.4pt}}
\begin{lstlisting}
proc ::constcl::show-pair {pair} {
    # take an object and print the car and the cdr of the stored value
    set str {}
    set a [car $pair]
    set d [cdr $pair]
    # print car
    ::append str [$a show]
    ::if {[pair? $d] ne "#f"} {
        # cdr is a cons pair
        ::append str " "
        ::append str [show-pair $d]
    } elseif {[null? $d] ne "#f"} {
        # cdr is nil
        return $str
    } else {
        # it is an atom
        ::append str " . "
        ::append str [$d show]
    }
    return $str
}
\end{lstlisting}
\noindent\makebox[\linewidth]{\rule{\linewidth}{0.4pt}}

\textbf{cons}


\texttt{cons} joins two values in a pair; useful in many operations such as pushing a new value onto a list.

\begin{tabular}{ |l l| }
\hline
\multicolumn{2}{|l|}{cons (public)} \\
\hline
car & a Lisp value \\
cdr & a Lisp value \\
\textit{Returns:} & a pair \\
\hline
\end{tabular}


Example:

\noindent\makebox[\linewidth]{\rule{\linewidth}{0.4pt}}
\begin{lstlisting}
(cons 'a 'b)              ⇒  (a . b)
(cons 'a nil)             ⇒  (a)
(cons 'a (cons 'b nil))   ⇒  (a b)
\end{lstlisting}
\noindent\makebox[\linewidth]{\rule{\linewidth}{0.4pt}}

![a small schematic to make it clearer](/images/consing.png)

\noindent\makebox[\linewidth]{\rule{\linewidth}{0.4pt}}
\begin{lstlisting}
reg cons ::constcl::cons
 
proc ::constcl::cons {car cdr} {
    MkPair $car $cdr
}
\end{lstlisting}
\noindent\makebox[\linewidth]{\rule{\linewidth}{0.4pt}}

\textbf{car}


\texttt{car} gets the contents of the first cell in a pair.

\begin{tabular}{ |l l| }
\hline
\multicolumn{2}{|l|}{car (public)} \\
\hline
pair & a pair \\
\textit{Returns:} & a Lisp value \\
\hline
\end{tabular}


Example:

\noindent\makebox[\linewidth]{\rule{\linewidth}{0.4pt}}
\begin{lstlisting}
(car '(a b))   ⇒  a
\end{lstlisting}
\noindent\makebox[\linewidth]{\rule{\linewidth}{0.4pt}}
\noindent\makebox[\linewidth]{\rule{\linewidth}{0.4pt}}
\begin{lstlisting}
reg car ::constcl::car
 
proc ::constcl::car {pair} {
    $pair car
}
\end{lstlisting}
\noindent\makebox[\linewidth]{\rule{\linewidth}{0.4pt}}

\textbf{cdr}


\texttt{cdr} gets the contents of the second cell in a pair.

\begin{tabular}{ |l l| }
\hline
\multicolumn{2}{|l|}{cdr (public)} \\
\hline
pair & a pair \\
\textit{Returns:} & a Lisp value \\
\hline
\end{tabular}


Example:

\noindent\makebox[\linewidth]{\rule{\linewidth}{0.4pt}}
\begin{lstlisting}
(cdr '(a b))   ⇒  (b)
\end{lstlisting}
\noindent\makebox[\linewidth]{\rule{\linewidth}{0.4pt}}
\noindent\makebox[\linewidth]{\rule{\linewidth}{0.4pt}}
\begin{lstlisting}
reg cdr ::constcl::cdr
 
proc ::constcl::cdr {pair} {
    $pair cdr
}
\end{lstlisting}
\noindent\makebox[\linewidth]{\rule{\linewidth}{0.4pt}}

\textbf{caar} to \textbf{cddddr}


\texttt{car} and \texttt{cdr} can be combined to form 28 composite access operations.

\noindent\makebox[\linewidth]{\rule{\linewidth}{0.4pt}}
\begin{lstlisting}
foreach ads {
    aa
    ad
    da
    dd
    aaa
    ada
    daa
    dda
    aad
    add
    dad
    ddd
    aaaa
    adaa
    daaa
    ddaa
    aada
    adda
    dada
    ddda
    aaad
    adad
    daad
    ddad
    aadd
    addd
    dadd
    dddd
} {
    reg c${ads}r
 
    proc ::constcl::c${ads}r {pair} "
        foreach c \[lreverse \[split $ads {}\]\] {
            ::if {\$c eq \"a\"} {
                set pair \[car \$pair\]
            } else {
                set pair \[cdr \$pair\]
            }
        }
        return \$pair
    "
 
}
\end{lstlisting}
\noindent\makebox[\linewidth]{\rule{\linewidth}{0.4pt}}

\textbf{set-car!}


\texttt{set-car!} sets the contents of the first cell in a pair.

\begin{tabular}{ |l l| }
\hline
\multicolumn{2}{|l|}{set-car! (public)} \\
\hline
pair & a pair \\
val & a Lisp value \\
\textit{Returns:} & a pair \\
\hline
\end{tabular}


Example:

\noindent\makebox[\linewidth]{\rule{\linewidth}{0.4pt}}
\begin{lstlisting}
(let ((pair (cons 'a 'b)) (val 'x)) (set-car! pair val))   ⇒  (x . b)
\end{lstlisting}
\noindent\makebox[\linewidth]{\rule{\linewidth}{0.4pt}}
\noindent\makebox[\linewidth]{\rule{\linewidth}{0.4pt}}
\begin{lstlisting}
reg set-car! ::constcl::set-car!
 
proc ::constcl::set-car! {pair val} {
    $pair set-car! $val
}
\end{lstlisting}
\noindent\makebox[\linewidth]{\rule{\linewidth}{0.4pt}}

\textbf{set-cdr!}


\texttt{set-cdr!} sets the contents of the second cell in a pair.

\begin{tabular}{ |l l| }
\hline
\multicolumn{2}{|l|}{set-cdr! (public)} \\
\hline
pair & a pair \\
val & a Lisp value \\
\textit{Returns:} & a pair \\
\hline
\end{tabular}


Example:

\noindent\makebox[\linewidth]{\rule{\linewidth}{0.4pt}}
\begin{lstlisting}
(let ((pair (cons 'a 'b)) (val 'x)) (set-cdr! pair val))   ⇒  (a . x)
\end{lstlisting}
\noindent\makebox[\linewidth]{\rule{\linewidth}{0.4pt}}
\noindent\makebox[\linewidth]{\rule{\linewidth}{0.4pt}}
\begin{lstlisting}
reg set-cdr! ::constcl::set-cdr!
 
proc ::constcl::set-cdr! {pair val} {
    $pair set-cdr! $val
}
\end{lstlisting}
\noindent\makebox[\linewidth]{\rule{\linewidth}{0.4pt}}

\textbf{list?}


The \texttt{list?} predicate tests if a pair is part of a proper list, one that ends with NIL.

\begin{tabular}{ |l l| }
\hline
\multicolumn{2}{|l|}{list? (public)} \\
\hline
pair & a pair \\
\textit{Returns:} & a boolean \\
\hline
\end{tabular}

\noindent\makebox[\linewidth]{\rule{\linewidth}{0.4pt}}
\begin{lstlisting}
reg list? ::constcl::list?
 
proc ::constcl::list? {pair} {
    set visited {}
    return [listp $pair]
}
\end{lstlisting}
\noindent\makebox[\linewidth]{\rule{\linewidth}{0.4pt}}
\begin{tabular}{ |l l| }
\hline
\multicolumn{2}{|l|}{listp (internal)} \\
\hline
pair & a pair \\
\textit{Returns:} & a boolean \\
\hline
\end{tabular}

\noindent\makebox[\linewidth]{\rule{\linewidth}{0.4pt}}
\begin{lstlisting}
proc ::constcl::listp {pair} {
    upvar visited visited
    ::if {$pair in $visited} {
        return #f
    }
    lappend visited $pair
    ::if {[null? $pair] ne "#f"} {
        return #t
    } elseif {[pair? $pair] ne "#f"} {
        return [listp [cdr $pair]]
    } else {
        return #f
    }
}
\end{lstlisting}
\noindent\makebox[\linewidth]{\rule{\linewidth}{0.4pt}}

\textbf{list}


\texttt{list} constructs a Lisp list from a number of values.

\begin{tabular}{ |l l| }
\hline
\multicolumn{2}{|l|}{list (public)} \\
\hline
args & some Lisp values \\
\textit{Returns:} & a Lisp list of Lisp values \\
\hline
\end{tabular}


Example:

\noindent\makebox[\linewidth]{\rule{\linewidth}{0.4pt}}
\begin{lstlisting}
(list 1 2 3)   ⇒  (1 2 3)
\end{lstlisting}
\noindent\makebox[\linewidth]{\rule{\linewidth}{0.4pt}}
\noindent\makebox[\linewidth]{\rule{\linewidth}{0.4pt}}
\begin{lstlisting}
reg list ::constcl::list
 
proc ::constcl::list {args} {
        return #NIL
    } else {
        set prev #NIL
        foreach obj [lreverse $args] {
            set prev [cons $obj $prev]
        }
        return $prev
    }
}
\end{lstlisting}
\noindent\makebox[\linewidth]{\rule{\linewidth}{0.4pt}}

\textbf{length}


\texttt{length} reports the length of a Lisp list.

\begin{tabular}{ |l l| }
\hline
\multicolumn{2}{|l|}{length (public)} \\
\hline
pair & a pair \\
\textit{Returns:} & a number \\
\hline
\end{tabular}


Example:

\noindent\makebox[\linewidth]{\rule{\linewidth}{0.4pt}}
\begin{lstlisting}
(length '(a b c d))   ⇒  4
\end{lstlisting}
\noindent\makebox[\linewidth]{\rule{\linewidth}{0.4pt}}
\noindent\makebox[\linewidth]{\rule{\linewidth}{0.4pt}}
\begin{lstlisting}
reg length ::constcl::length
 
proc ::constcl::length {pair} {
    check {list? $pair} {LIST expected\n([pn] lst)}
    MkNumber [length-helper $pair]
}
\end{lstlisting}
\noindent\makebox[\linewidth]{\rule{\linewidth}{0.4pt}}
\begin{tabular}{ |l l| }
\hline
\multicolumn{2}{|l|}{length-helper (internal)} \\
\hline
pair & a pair \\
\textit{Returns:} & a Tcl number \\
\hline
\end{tabular}

\noindent\makebox[\linewidth]{\rule{\linewidth}{0.4pt}}
\begin{lstlisting}
proc ::constcl::length-helper {pair} {
    ::if {[null? $pair] ne "#f"} {
    } else {
        return [expr {1 + [length-helper [cdr $pair]]}]
    }
}
\end{lstlisting}
\noindent\makebox[\linewidth]{\rule{\linewidth}{0.4pt}}

\textbf{append}


\texttt{append} joins lists together.


Example:

\noindent\makebox[\linewidth]{\rule{\linewidth}{0.4pt}}
\begin{lstlisting}
(append '(a b) '(c d))   ⇒  (a b c d)
\end{lstlisting}
\noindent\makebox[\linewidth]{\rule{\linewidth}{0.4pt}}
\begin{tabular}{ |l l| }
\hline
\multicolumn{2}{|l|}{append (public)} \\
\hline
args & some lists \\
\textit{Returns:} & a Lisp list of Lisp values \\
\hline
\end{tabular}

\noindent\makebox[\linewidth]{\rule{\linewidth}{0.4pt}}
\begin{lstlisting}
reg append ::constcl::append
 
proc ::constcl::append {args} {
    set prev [lindex $args end]
        set prev [copy-list $r $prev]
    }
    set prev
}
\end{lstlisting}
\noindent\makebox[\linewidth]{\rule{\linewidth}{0.4pt}}
\begin{tabular}{ |l l| }
\hline
\multicolumn{2}{|l|}{copy-list (internal)} \\
\hline
pair & a pair \\
next & a Lisp list of Lisp values \\
\textit{Returns:} & a Lisp list of Lisp values \\
\hline
\end{tabular}

\noindent\makebox[\linewidth]{\rule{\linewidth}{0.4pt}}
\begin{lstlisting}
proc ::constcl::copy-list {pair next} {
    # TODO only fresh conses in the direct chain to NIL
    ::if {[null? $pair] ne "#f"} {
        set next
    } elseif {[null? [cdr $pair]] ne "#f"} {
        cons [car $pair] $next
    } else {
        cons [car $pair] [copy-list [cdr $pair] $next]
    }
}
\end{lstlisting}
\noindent\makebox[\linewidth]{\rule{\linewidth}{0.4pt}}

\textbf{reverse}


\texttt{reverse} produces a reversed copy of a Lisp list.

\begin{tabular}{ |l l| }
\hline
\multicolumn{2}{|l|}{reverse (public)} \\
\hline
vals & a Lisp list of Lisp values \\
\textit{Returns:} & a Lisp list of Lisp values \\
\hline
\end{tabular}


Example:

\noindent\makebox[\linewidth]{\rule{\linewidth}{0.4pt}}
\begin{lstlisting}
(reverse '(a b c))   ⇒  (c b a)
\end{lstlisting}
\noindent\makebox[\linewidth]{\rule{\linewidth}{0.4pt}}
\noindent\makebox[\linewidth]{\rule{\linewidth}{0.4pt}}
\begin{lstlisting}
reg reverse ::constcl::reverse
 
proc ::constcl::reverse {vals} {
    list {*}[lreverse [splitlist $vals]]
}
\end{lstlisting}
\noindent\makebox[\linewidth]{\rule{\linewidth}{0.4pt}}

\textbf{list-tail}


Given a list index, \texttt{list-tail} yields the sublist starting from that index.

\begin{tabular}{ |l l| }
\hline
\multicolumn{2}{|l|}{list-tail (public)} \\
\hline
vals & a Lisp list of Lisp values \\
k & a number \\
\textit{Returns:} & a Lisp list of Lisp values \\
\hline
\end{tabular}


Example:

\noindent\makebox[\linewidth]{\rule{\linewidth}{0.4pt}}
\begin{lstlisting}
(let ((lst '(a b c d e f)) (k 3)) (list-tail lst k))   ⇒  (d e f)
\end{lstlisting}
\noindent\makebox[\linewidth]{\rule{\linewidth}{0.4pt}}
\noindent\makebox[\linewidth]{\rule{\linewidth}{0.4pt}}
\begin{lstlisting}
reg list-tail ::constcl::list-tail
 
proc ::constcl::list-tail {vals k} {
    ::if {[zero? $k] ne "#f"} {
        return $vals
    } else {
        list-tail [cdr $vals] [- $k #1]
    }
}
\end{lstlisting}
\noindent\makebox[\linewidth]{\rule{\linewidth}{0.4pt}}

\textbf{list-ref}


\texttt{list-ref} yields the list item at a given index.

\begin{tabular}{ |l l| }
\hline
\multicolumn{2}{|l|}{list-ref (public)} \\
\hline
vals & a Lisp list of Lisp values \\
k & a number \\
\textit{Returns:} & a Lisp value \\
\hline
\end{tabular}


Example:

\noindent\makebox[\linewidth]{\rule{\linewidth}{0.4pt}}
\begin{lstlisting}
(let ((lst '(a b c d e f)) (k 3)) (list-ref lst k))   ⇒  d
\end{lstlisting}
\noindent\makebox[\linewidth]{\rule{\linewidth}{0.4pt}}
\noindent\makebox[\linewidth]{\rule{\linewidth}{0.4pt}}
\begin{lstlisting}
reg list-ref ::constcl::list-ref
 
proc ::constcl::list-ref {vals k} {
    car [list-tail $vals $k]
}
\end{lstlisting}
\noindent\makebox[\linewidth]{\rule{\linewidth}{0.4pt}}

\textbf{memq}


\textbf{memv}


\textbf{member}


\texttt{memq}, \texttt{memv}, and \texttt{member} return the sublist starting with a given item, or \texttt{\#f} if there is none. They use \texttt{eq?}, \texttt{eqv?}, and \texttt{equal?}, respectively, for the comparison.

\begin{tabular}{ |l l| }
\hline
\multicolumn{2}{|l|}{memq (public)} \\
\hline
val1 & a Lisp value \\
val2 & a Lisp list of Lisp values \\
\textit{Returns:} & a Lisp list of values OR \#f \\
\hline
\end{tabular}


Example:

\noindent\makebox[\linewidth]{\rule{\linewidth}{0.4pt}}
\begin{lstlisting}
(let ((lst '(a b c d e f)) (val 'd)) (memq val lst))   ⇒  (d e f)
\end{lstlisting}
\noindent\makebox[\linewidth]{\rule{\linewidth}{0.4pt}}
\noindent\makebox[\linewidth]{\rule{\linewidth}{0.4pt}}
\begin{lstlisting}
reg memq ::constcl::memq
 
proc ::constcl::memq {val1 val2} {
    return [member-proc eq? $val1 $val2]
}
\end{lstlisting}
\noindent\makebox[\linewidth]{\rule{\linewidth}{0.4pt}}
\begin{tabular}{ |l l| }
\hline
\multicolumn{2}{|l|}{memv (public)} \\
\hline
val1 & a Lisp value \\
val2 & a Lisp list of Lisp values \\
\textit{Returns:} & a Lisp list of values OR \#f \\
\hline
\end{tabular}

\noindent\makebox[\linewidth]{\rule{\linewidth}{0.4pt}}
\begin{lstlisting}
reg memv ::constcl::memv
 
proc ::constcl::memv {val1 val2} {
    return [member-proc eqv? $val1 $val2]
}
\end{lstlisting}
\noindent\makebox[\linewidth]{\rule{\linewidth}{0.4pt}}
\begin{tabular}{ |l l| }
\hline
\multicolumn{2}{|l|}{member (public)} \\
\hline
val1 & a Lisp value \\
val2 & a Lisp list of Lisp values \\
\textit{Returns:} & a Lisp list of values OR \#f \\
\hline
\end{tabular}

\noindent\makebox[\linewidth]{\rule{\linewidth}{0.4pt}}
\begin{lstlisting}
reg member ::constcl::member
 
proc ::constcl::member {val1 val2} {
    return [member-proc equal? $val1 $val2]
}
\end{lstlisting}
\noindent\makebox[\linewidth]{\rule{\linewidth}{0.4pt}}
\begin{tabular}{ |l l| }
\hline
\multicolumn{2}{|l|}{member-proc (internal)} \\
\hline
epred & an equivalence predicate \\
val1 & a Lisp value \\
val2 & a Lisp list of Lisp values \\
\textit{Returns:} & a Lisp list of values OR \#f \\
\hline
\end{tabular}

\noindent\makebox[\linewidth]{\rule{\linewidth}{0.4pt}}
\begin{lstlisting}
 
proc ::constcl::member-proc {epred val1 val2} {
    switch $epred {
        eq? { set name "memq" }
        eqv? { set name "memv" }
        equal? { set name "member" }
    }
    check {list? $val2} {LIST expected\n($name [$val1 show] [$val2 show])}
    ::if {[null? $val2] ne "#f"} {
        return #f
    } elseif {[pair? $val2] ne "#f"} {
        ::if {[$epred $val1 [car $val2]] ne "#f"} {
            return $val2
        } else {
            return [member-proc $epred $val1 [cdr $val2]]
        }
    }
}
\end{lstlisting}
\noindent\makebox[\linewidth]{\rule{\linewidth}{0.4pt}}

\textbf{assq}


\textbf{assv}


\textbf{assoc}


\texttt{assq}, \texttt{assv}, and \texttt{assoc} return the associative item marked with a given item, or \texttt{\#f} if there is none. They use \texttt{eq?}, \texttt{eqv?}, and \texttt{equal?}, respectively, for the comparison. They implement lookup in the form of lookup table known as an association list, or \_alist\_.


Example:

\noindent\makebox[\linewidth]{\rule{\linewidth}{0.4pt}}
\begin{lstlisting}
(define e '((a 1) (b 2) (c 3)))
(assq 'a e)                       ⇒ (a 1)
\end{lstlisting}
\noindent\makebox[\linewidth]{\rule{\linewidth}{0.4pt}}
\begin{tabular}{ |l l| }
\hline
\multicolumn{2}{|l|}{assq (public)} \\
\hline
val1 & a Lisp value \\
val2 & an association list \\
\textit{Returns:} & a Lisp list of values OR \#f \\
\hline
\end{tabular}

\noindent\makebox[\linewidth]{\rule{\linewidth}{0.4pt}}
\begin{lstlisting}
reg assq
 
proc ::constcl::assq {val1 val2} {
    return [assoc-proc eq? $val1 $val2]
}
\end{lstlisting}
\noindent\makebox[\linewidth]{\rule{\linewidth}{0.4pt}}
\begin{tabular}{ |l l| }
\hline
\multicolumn{2}{|l|}{assv (public)} \\
\hline
val1 & a Lisp value \\
val2 & an association list \\
\textit{Returns:} & a Lisp list of values OR \#f \\
\hline
\end{tabular}

\noindent\makebox[\linewidth]{\rule{\linewidth}{0.4pt}}
\begin{lstlisting}
reg assv
 
proc ::constcl::assv {val1 val2} {
    return [assoc-proc eqv? $val1 $val2]
}
\end{lstlisting}
\noindent\makebox[\linewidth]{\rule{\linewidth}{0.4pt}}
\begin{tabular}{ |l l| }
\hline
\multicolumn{2}{|l|}{assoc (public)} \\
\hline
val1 & a Lisp value \\
val2 & an association list \\
\textit{Returns:} & a Lisp list of values OR \#f \\
\hline
\end{tabular}

\noindent\makebox[\linewidth]{\rule{\linewidth}{0.4pt}}
\begin{lstlisting}
reg assoc
 
proc ::constcl::assoc {val1 val2} {
    return [assoc-proc equal? $val1 $val2]
}
\end{lstlisting}
\noindent\makebox[\linewidth]{\rule{\linewidth}{0.4pt}}
\begin{tabular}{ |l l| }
\hline
\multicolumn{2}{|l|}{assoc-proc (internal)} \\
\hline
epred & an equivalence predicate \\
val1 & a Lisp value \\
val2 & an association list \\
\textit{Returns:} & a Lisp list of values OR \#f \\
\hline
\end{tabular}

\noindent\makebox[\linewidth]{\rule{\linewidth}{0.4pt}}
\begin{lstlisting}
proc ::constcl::assoc-proc {epred val1 val2} {
    switch $epred {
        eq? { set name "assq" }
        eqv? { set name "assv" }
        equal? { set name "assoc" }
    }
    check {list? $val2} {LIST expected\n($name [$val1 show] [$val2 show])}
    ::if {[null? $val2] ne "#f"} {
        return #f
    } elseif {[pair? $val2] ne "#f"} {
        ::if {[pair? [car $val2]] ne "#f" && [$epred $val1 [caar $val2]] ne "#f"} {
            return [car $val2]
        } else {
            return [assoc-proc $epred $val1 [cdr $val2]]
        }
    }
}
\end{lstlisting}
\noindent\makebox[\linewidth]{\rule{\linewidth}{0.4pt}}
\subsection{Strings}
\label{strings}

Procedures for dealing with strings of characters.


\textbf{String} class

\noindent\makebox[\linewidth]{\rule{\linewidth}{0.4pt}}
\begin{lstlisting}
oo::class create ::constcl::String {
    superclass ::constcl::NIL
    variable data constant
    constructor {v} {
        set len [::string length $v]
        set vsa [::constcl::vsAlloc $len]
        set idx $vsa
        foreach elt [split $v {}] {
            ::if {$elt eq " "} {
                set c #\\space
            } elseif {$elt eq "\n"} {
                set c #\\newline
            } else {
                set c #\\$elt
            }
            lset ::constcl::vectorSpace $idx [::constcl::MkChar $c]
            incr idx
        }
        set data [::constcl::cons [::constcl::MkNumber $vsa] [::constcl::MkNumber $len]]
    }
    method = {str} {::string equal [my value] [$str value]}
    method cmp {str} {::string compare [my value] [$str value]}
    method length {} {::constcl::cdr $data}
    method ref {k} {
        set k [$k numval]
            ::error "index out of range\n$k"
        }
        lindex [my store] $k
    }
    method store {} {
        set base [[::constcl::car $data] numval]
        set end [expr {[[my length] numval] + $base - 1}]
        lrange $::constcl::vectorSpace $base $end
    }
    method value {} {
        join [lmap c [my store] {$c char}] {}
    }
    method set! {k c} {
        ::if {[my constant]} {
            ::error "string is constant"
        } else {
            set k [$k numval]
                ::error "index out of range\n$k"
            }
            set base [[::constcl::car $data] numval]
            lset ::constcl::vectorSpace $k+$base $c
        }
        return [self]
    }
    method fill! {c} {
        ::if {[my constant]} {
            ::error "string is constant"
        } else {
            set base [[::constcl::car $data] numval]
            set len [[my length] numval]
            for {set idx $base} {$idx < $len+$base} {incr idx} {
                lset ::constcl::vectorSpace $idx $c
            }
        }
        return [self]
    }
    method substring {from to} {
        join [lmap c [lrange [my store] [$from numval] [$to numval]] {$c char}] {}
    }
    method mkconstant {} {set constant 1}
    method constant {} {set constant}
    method write {handle} { puts -nonewline $handle "\"[my value]\"" }
    method display {} { puts -nonewline [my value] }
    method show {} {format "\"[my value]\""}
}
 
interp alias {} MkString {} ::constcl::String new
\end{lstlisting}
\noindent\makebox[\linewidth]{\rule{\linewidth}{0.4pt}}
\begin{tabular}{ |l l| }
\hline
\multicolumn{2}{|l|}{string? (public)} \\
\hline
val & a Lisp value \\
\textit{Returns:} & a boolean \\
\hline
\end{tabular}

\noindent\makebox[\linewidth]{\rule{\linewidth}{0.4pt}}
\begin{lstlisting}
reg string? ::constcl::string?
 
proc ::constcl::string? {val} {
    ::if {[info object isa typeof $val ::constcl::String]} {
        return #t
    } elseif {[info object isa typeof [interp alias {} $val] ::constcl::String]} {
        return #t
    } else {
        return #f
    }
}
\end{lstlisting}
\noindent\makebox[\linewidth]{\rule{\linewidth}{0.4pt}}

\textbf{make-string}


\texttt{make-string} creates a string of \_k\_ characters, optionally filled with \_char\_ characters. If \_char\_ is omitted, the string will be filled with space characters.

\begin{tabular}{ |l l| }
\hline
\multicolumn{2}{|l|}{make-string (public)} \\
\hline
k & a number \\
?char? & a character \\
\textit{Returns:} & a string \\
\hline
\end{tabular}


Example:

\noindent\makebox[\linewidth]{\rule{\linewidth}{0.4pt}}
\begin{lstlisting}
(let ((k 5)) (make-string k))                   ⇒  "     "
(let ((k 5) (char #\A)) (make-string k char))   ⇒  "AAAAA"
\end{lstlisting}
\noindent\makebox[\linewidth]{\rule{\linewidth}{0.4pt}}
\noindent\makebox[\linewidth]{\rule{\linewidth}{0.4pt}}
\begin{lstlisting}
reg make-string ::constcl::make-string
 
proc ::constcl::make-string {k args} {
        return [MkString [::string repeat " " [$k numval]]]
    } else {
        lassign $args char
        return [MkString [::string repeat [$char char] [$k numval]]]
    }
}
\end{lstlisting}
\noindent\makebox[\linewidth]{\rule{\linewidth}{0.4pt}}

\textbf{string}


\texttt{string} constructs a string from a number of Lisp characters.

\begin{tabular}{ |l l| }
\hline
\multicolumn{2}{|l|}{string (public)} \\
\hline
args & some characters \\
\textit{Returns:} & a string \\
\hline
\end{tabular}


Example:

\noindent\makebox[\linewidth]{\rule{\linewidth}{0.4pt}}
\begin{lstlisting}
(string #\f #\o #\o)   ⇒  "foo"
\end{lstlisting}
\noindent\makebox[\linewidth]{\rule{\linewidth}{0.4pt}}
\noindent\makebox[\linewidth]{\rule{\linewidth}{0.4pt}}
\begin{lstlisting}
reg string ::constcl::string
 
proc ::constcl::string {args} {
    set str {}
    foreach char $args {
        check {::constcl::char? $char} {CHAR expected\n([pn] [lmap c $args {$c show}])}
        ::append str [$char char]
    }
    return [MkString $str]
}
\end{lstlisting}
\noindent\makebox[\linewidth]{\rule{\linewidth}{0.4pt}}

\textbf{string-length}


\texttt{string-length} reports a string's length.

\begin{tabular}{ |l l| }
\hline
\multicolumn{2}{|l|}{string-length (public)} \\
\hline
str & a string \\
\textit{Returns:} & a number \\
\hline
\end{tabular}


Example:

\noindent\makebox[\linewidth]{\rule{\linewidth}{0.4pt}}
\begin{lstlisting}
(string-length "foobar")   ⇒ 6
\end{lstlisting}
\noindent\makebox[\linewidth]{\rule{\linewidth}{0.4pt}}
\noindent\makebox[\linewidth]{\rule{\linewidth}{0.4pt}}
\begin{lstlisting}
reg string-length ::constcl::string-length
 
proc ::constcl::string-length {str} {
    check {::constcl::string? $str} {STRING expected\n([pn] [$str show])}
    return [MkNumber [[$str length] numval]]
}
\end{lstlisting}
\noindent\makebox[\linewidth]{\rule{\linewidth}{0.4pt}}

\textbf{string-ref}

\begin{tabular}{ |l l| }
\hline
\multicolumn{2}{|l|}{string-ref (public)} \\
\hline
str & a string \\
k & a number \\
\textit{Returns:} & a character \\
\hline
\end{tabular}


Example:

\noindent\makebox[\linewidth]{\rule{\linewidth}{0.4pt}}
\begin{lstlisting}
(string-ref "foobar" 3)   ⇒ #\b
\end{lstlisting}
\noindent\makebox[\linewidth]{\rule{\linewidth}{0.4pt}}
\noindent\makebox[\linewidth]{\rule{\linewidth}{0.4pt}}
\begin{lstlisting}
reg string-ref ::constcl::string-ref
 
proc ::constcl::string-ref {str k} {
    check {::constcl::string? $str} {STRING expected\n([pn] [$str show] [$k show])}
    check {::constcl::number? $k} {Exact INTEGER expected\n([pn] [$str show] [$k show])}
    return [$str ref $k]
}
\end{lstlisting}
\noindent\makebox[\linewidth]{\rule{\linewidth}{0.4pt}}

\textbf{string-set!}


\texttt{string-set!} replaces the character at \_k\_ with \_char\_ in a non-constant string.

\begin{tabular}{ |l l| }
\hline
\multicolumn{2}{|l|}{string-set! (public)} \\
\hline
str & a string \\
k & a number \\
char & a character \\
\textit{Returns:} & a string \\
\hline
\end{tabular}


Example:

\noindent\makebox[\linewidth]{\rule{\linewidth}{0.4pt}}
\begin{lstlisting}
(let ((str (string #\f #\o #\o)) (k 2) (char #\x)) (string-set! str k char))   ⇒  "fox"
\end{lstlisting}
\noindent\makebox[\linewidth]{\rule{\linewidth}{0.4pt}}
\noindent\makebox[\linewidth]{\rule{\linewidth}{0.4pt}}
\begin{lstlisting}
reg string-set!
 
proc ::constcl::string-set! {str k char} {
    check {string? $str} {STRING expected\n([pn] [$str show] [$k show] [$char show])}
    check {number? $k} {Exact INTEGER expected\n([pn] [$str show] [$k show] [$char show])}
    check {char? $char} {CHAR expected\n([pn] [$str show] [$k show] [$char show])}
    $str set! $k $char
    return $str
}
\end{lstlisting}
\noindent\makebox[\linewidth]{\rule{\linewidth}{0.4pt}}

\textbf{string=?}, \textbf{string-ci=?}


\textbf{string<?}, \textbf{string-ci<?}


\textbf{string>?}, \textbf{string-ci>?}


\textbf{string<=?}, \textbf{string-ci<=?}


\textbf{string>=?}, \textbf{string-ci>=?}


\texttt{string=?}, \texttt{string<?}, \texttt{string>?}, \texttt{string<=?}, \texttt{string>=?} and their case insensitive variants \texttt{string-ci=?}, \texttt{string-ci<?}, \texttt{string-ci>?}, \texttt{string-ci<=?}, \texttt{string-ci>=?} compare strings.

\begin{tabular}{ |l l| }
\hline
\multicolumn{2}{|l|}{string=?, string<?, string>?, string<=?, string>=? (public)} \\
\hline
str1 & a string \\
str2 & a string \\
\textit{Returns:} & a boolean \\
\hline
\end{tabular}

\begin{tabular}{ |l l| }
\hline
\multicolumn{2}{|l|}{string-ci=?, string-ci<?, string-ci>?, string-ci<=?, string-ci>=? (public)} \\
\hline
str1 & a string \\
str2 & a string \\
\textit{Returns:} & a boolean \\
\hline
\end{tabular}

\noindent\makebox[\linewidth]{\rule{\linewidth}{0.4pt}}
\begin{lstlisting}
reg string=? ::constcl::string=?
 
proc ::constcl::string=? {str1 str2} {
    check {string? $str1} {STRING expected\n([pn] [$str1 show] [$str2 show])}
    check {string? $str2} {STRING expected\n([pn] [$str1 show] [$str2 show])}
    ::if {[$str1 value] eq [$str2 value]} {
        return #t
    } else {
        return #f
    }
}
\end{lstlisting}
\noindent\makebox[\linewidth]{\rule{\linewidth}{0.4pt}}
\noindent\makebox[\linewidth]{\rule{\linewidth}{0.4pt}}
\begin{lstlisting}
reg string-ci=? ::constcl::string-ci=?
 
proc ::constcl::string-ci=? {str1 str2} {
    check {string? $str1} {STRING expected\n([pn] [$str1 show] [$str2 show])}
    check {string? $str2} {STRING expected\n([pn] [$str1 show] [$str2 show])}
    ::if {[::string tolower [$str1 value]] eq [::string tolower [$str2 value]]} {
        return #t
    } else {
        return #f
    }
}
\end{lstlisting}
\noindent\makebox[\linewidth]{\rule{\linewidth}{0.4pt}}
\noindent\makebox[\linewidth]{\rule{\linewidth}{0.4pt}}
\begin{lstlisting}
reg string<? ::constcl::string<?
 
proc ::constcl::string<? {str1 str2} {
    check {string? $str1} {STRING expected\n([pn] [$str1 show] [$str2 show])}
    check {string? $str2} {STRING expected\n([pn] [$str1 show] [$str2 show])}
    ::if {[$str1 value] < [$str2 value]} {
        return #t
    } else {
        return #f
    }
}
\end{lstlisting}
\noindent\makebox[\linewidth]{\rule{\linewidth}{0.4pt}}
\noindent\makebox[\linewidth]{\rule{\linewidth}{0.4pt}}
\begin{lstlisting}
reg string-ci<? ::constcl::string-ci<?
 
proc ::constcl::string-ci<? {str1 str2} {
    check {string? $str1} {STRING expected\n([pn] [$str1 show] [$str2 show])}
    check {string? $str2} {STRING expected\n([pn] [$str1 show] [$str2 show])}
    ::if {[::string tolower [$str1 value]] < [::string tolower [$str2 value]]} {
        return #t
    } else {
        return #f
    }
}
\end{lstlisting}
\noindent\makebox[\linewidth]{\rule{\linewidth}{0.4pt}}
\noindent\makebox[\linewidth]{\rule{\linewidth}{0.4pt}}
\begin{lstlisting}
reg string>? ::constcl::string>?
 
proc ::constcl::string>? {str1 str2} {
    check {string? $str1} {STRING expected\n([pn] [$str1 show] [$str2 show])}
    check {string? $str2} {STRING expected\n([pn] [$str1 show] [$str2 show])}
    ::if {[$str1 value] > [$str2 value]} {
        return #t
    } else {
        return #f
    }
}
\end{lstlisting}
\noindent\makebox[\linewidth]{\rule{\linewidth}{0.4pt}}
\noindent\makebox[\linewidth]{\rule{\linewidth}{0.4pt}}
\begin{lstlisting}
reg string-ci>? ::constcl::string-ci>?
 
proc ::constcl::string-ci>? {str1 str2} {
    check {string? $str1} {STRING expected\n([pn] [$str1 show] [$str2 show])}
    check {string? $str2} {STRING expected\n([pn] [$str1 show] [$str2 show])}
    ::if {[::string tolower [$str1 value]] > [::string tolower [$str2 value]]} {
        return #t
    } else {
        return #f
    }
}
\end{lstlisting}
\noindent\makebox[\linewidth]{\rule{\linewidth}{0.4pt}}
\noindent\makebox[\linewidth]{\rule{\linewidth}{0.4pt}}
\begin{lstlisting}
reg string<=? ::constcl::string<=?
 
proc ::constcl::string<=? {str1 str2} {
    check {string? $str1} {STRING expected\n([pn] [$str1 show] [$str2 show])}
    check {string? $str2} {STRING expected\n([pn] [$str1 show] [$str2 show])}
    ::if {[$str1 value] <= [$str2 value]} {
        return #t
    } else {
        return #f
    }
}
\end{lstlisting}
\noindent\makebox[\linewidth]{\rule{\linewidth}{0.4pt}}
\noindent\makebox[\linewidth]{\rule{\linewidth}{0.4pt}}
\begin{lstlisting}
reg string-ci<=? ::constcl::string-ci<=?
 
proc ::constcl::string-ci<=? {str1 str2} {
    check {string? $str1} {STRING expected\n([pn] [$str1 show] [$str2 show])}
    check {string? $str2} {STRING expected\n([pn] [$str1 show] [$str2 show])}
    ::if {[::string tolower [$str1 value]] <= [::string tolower [$str2 value]]} {
        return #t
    } else {
        return #f
    }
}
\end{lstlisting}
\noindent\makebox[\linewidth]{\rule{\linewidth}{0.4pt}}
\noindent\makebox[\linewidth]{\rule{\linewidth}{0.4pt}}
\begin{lstlisting}
reg string>=? ::constcl::string>=?
 
proc ::constcl::string>=? {str1 str2} {
    check {string? $str1} {STRING expected\n([pn] [$str1 show] [$str2 show])}
    check {string? $str2} {STRING expected\n([pn] [$str1 show] [$str2 show])}
    ::if {[$str1 value] >= [$str2 value]} {
        return #t
    } else {
        return #f
    }
}
\end{lstlisting}
\noindent\makebox[\linewidth]{\rule{\linewidth}{0.4pt}}
\noindent\makebox[\linewidth]{\rule{\linewidth}{0.4pt}}
\begin{lstlisting}
reg string-ci>=? ::constcl::string-ci>=?
 
proc ::constcl::string-ci>=? {str1 str2} {
    check {string? $str1} {STRING expected\n([pn] [$str1 show] [$str2 show])}
    check {string? $str2} {STRING expected\n([pn] [$str1 show] [$str2 show])}
    ::if {[::string tolower [$str1 value]] >= [::string tolower [$str2 value]]} {
        return #t
    } else {
        return #f
    }
}
\end{lstlisting}
\noindent\makebox[\linewidth]{\rule{\linewidth}{0.4pt}}

\textbf{substring}


\texttt{substring} yields the substring of \_str\_ that starts at \_start\_ and ends at \_end\_.

\begin{tabular}{ |l l| }
\hline
\multicolumn{2}{|l|}{substring (public)} \\
\hline
str & a string \\
start & a number \\
end & a number \\
\textit{Returns:} & a string \\
\hline
\end{tabular}


Example:

\noindent\makebox[\linewidth]{\rule{\linewidth}{0.4pt}}
\begin{lstlisting}
(substring "foobar" 2 4)   ⇒ "oba"
\end{lstlisting}
\noindent\makebox[\linewidth]{\rule{\linewidth}{0.4pt}}
\noindent\makebox[\linewidth]{\rule{\linewidth}{0.4pt}}
\begin{lstlisting}
reg substring ::constcl::substring
 
proc ::constcl::substring {str start end} {
    check {string? $str} {STRING expected\n([pn] [$str show] [$start show] [$end show])}
    check {number? $start} {NUMBER expected\n([pn] [$str show] [$start show] [$end show])}
    check {number? $end} {NUMBER expected\n([pn] [$str show] [$start show] [$end show])}
    return [MkString [$str substring $start $end]]
}
\end{lstlisting}
\noindent\makebox[\linewidth]{\rule{\linewidth}{0.4pt}}

\textbf{string-append}


\texttt{string-append} joins strings together.

\begin{tabular}{ |l l| }
\hline
\multicolumn{2}{|l|}{string-append (public)} \\
\hline
args & some strings \\
\textit{Returns:} & a string \\
\hline
\end{tabular}


Example:

\noindent\makebox[\linewidth]{\rule{\linewidth}{0.4pt}}
\begin{lstlisting}
(string-append "foo" "bar")   ⇒  "foobar"
\end{lstlisting}
\noindent\makebox[\linewidth]{\rule{\linewidth}{0.4pt}}
\noindent\makebox[\linewidth]{\rule{\linewidth}{0.4pt}}
\begin{lstlisting}
reg string-append ::constcl::string-append
 
proc ::constcl::string-append {args} {
    MkString [::append --> {*}[lmap arg $args {$arg value}]]
}
\end{lstlisting}
\noindent\makebox[\linewidth]{\rule{\linewidth}{0.4pt}}

\textbf{string->list}


\texttt{string->list} converts a string to a Lisp list of characters.

\begin{tabular}{ |l l| }
\hline
\multicolumn{2}{|l|}{string->list (public)} \\
\hline
str & a string \\
\textit{Returns:} & a Lisp list of characters \\
\hline
\end{tabular}


Example:

\noindent\makebox[\linewidth]{\rule{\linewidth}{0.4pt}}
\begin{lstlisting}
(string->list "foo")   ⇒  (#\f #\o #\o)
\end{lstlisting}
\noindent\makebox[\linewidth]{\rule{\linewidth}{0.4pt}}
\noindent\makebox[\linewidth]{\rule{\linewidth}{0.4pt}}
\begin{lstlisting}
reg string->list ::constcl::string->list
 
proc ::constcl::string->list {str} {
    list {*}[$str store]
}
\end{lstlisting}
\noindent\makebox[\linewidth]{\rule{\linewidth}{0.4pt}}

\textbf{list->string}


\texttt{list->string} converts a Lisp list of characters to a string.

\begin{tabular}{ |l l| }
\hline
\multicolumn{2}{|l|}{list->string (public)} \\
\hline
list & a Lisp list of characters \\
\textit{Returns:} & a string \\
\hline
\end{tabular}


Example:

\noindent\makebox[\linewidth]{\rule{\linewidth}{0.4pt}}
\begin{lstlisting}
(list->string '(#\1 #\2 #\3))   ⇒ "123"
\end{lstlisting}
\noindent\makebox[\linewidth]{\rule{\linewidth}{0.4pt}}
\noindent\makebox[\linewidth]{\rule{\linewidth}{0.4pt}}
\begin{lstlisting}
reg list->string ::constcl::list->string
 
proc ::constcl::list->string {list} {
    MkString [::append --> {*}[lmap c [splitlist $list] {$c char}]]
}
\end{lstlisting}
\noindent\makebox[\linewidth]{\rule{\linewidth}{0.4pt}}

\textbf{string-copy}


\texttt{string-copy} makes a copy of a string.

\begin{tabular}{ |l l| }
\hline
\multicolumn{2}{|l|}{string-copy (public)} \\
\hline
str & a string \\
\textit{Returns:} & a string \\
\hline
\end{tabular}


Example:

\noindent\makebox[\linewidth]{\rule{\linewidth}{0.4pt}}
\begin{lstlisting}
\end{lstlisting}
\noindent\makebox[\linewidth]{\rule{\linewidth}{0.4pt}}
\noindent\makebox[\linewidth]{\rule{\linewidth}{0.4pt}}
\begin{lstlisting}
reg string-copy ::constcl::string-copy
 
proc ::constcl::string-copy {str} {
    check {string? $str} {STRING expected\n([pn] [$str show])}
    return [MkString [$str value]]
}
\end{lstlisting}
\noindent\makebox[\linewidth]{\rule{\linewidth}{0.4pt}}

\textbf{string-fill!}


\texttt{string-fill!} \_str\_ \_char\_ fills a non-constant string with \_char\_.

\begin{tabular}{ |l l| }
\hline
\multicolumn{2}{|l|}{string-fill! (public)} \\
\hline
str & a string \\
char & a character \\
\textit{Returns:} & a string \\
\hline
\end{tabular}


Example:

\noindent\makebox[\linewidth]{\rule{\linewidth}{0.4pt}}
\begin{lstlisting}
(let ((str (string-copy "foobar")) (char #\X)) (string-fill! str char))   ⇒  "XXXXXX"
\end{lstlisting}
\noindent\makebox[\linewidth]{\rule{\linewidth}{0.4pt}}
\noindent\makebox[\linewidth]{\rule{\linewidth}{0.4pt}}
\begin{lstlisting}
reg string-fill! ::constcl::string-fill!
 
proc ::constcl::string-fill! {str char} {
    check {string? $str} {STRING expected\n([pn] [$str show] [$char show])}
    $str fill! $char
    return $str
}
\end{lstlisting}
\noindent\makebox[\linewidth]{\rule{\linewidth}{0.4pt}}
\subsection{Symbols}
\label{symbols}

Symbols are like little strings that are used to refer to things (variables, including procedure names, etc) or for comparing against each other.


\textbf{Symbol} class

\noindent\makebox[\linewidth]{\rule{\linewidth}{0.4pt}}
\begin{lstlisting}
oo::class create ::constcl::Symbol {
    superclass ::constcl::NIL
    variable name caseconstant
    constructor {n} {
        ::if {   no &&   $n eq {}} {
            ::error "a symbol must have a name"
        }
        ::constcl::idcheck $n
        set name $n
    }
    method name {} {set name}
    method value {} {set name}
    method = {symname} {expr {$name eq $symname}}
    method mkconstant {} {}
    method constant {} {return 1}
    method make-case-constant {} {set caseconstant 1}
    method case-constant {} {set caseconstant}
    method write {handle} { puts -nonewline $handle [my name] }
    method display {} { puts -nonewline [my name] }
    method show {} {set name}
}
 
proc ::constcl::MkSymbol {n} {
    foreach instance [info class instances ::constcl::Symbol] {
        ::if {[$instance name] eq $n} {
            return $instance
        }
    }
    return [::constcl::Symbol new $n]
}
\end{lstlisting}
\noindent\makebox[\linewidth]{\rule{\linewidth}{0.4pt}}
\begin{tabular}{ |l l| }
\hline
\multicolumn{2}{|l|}{symbol? (public)} \\
\hline
val & a Lisp value \\
\textit{Returns:} & a boolean \\
\hline
\end{tabular}

\noindent\makebox[\linewidth]{\rule{\linewidth}{0.4pt}}
\begin{lstlisting}
reg symbol? ::constcl::symbol?
 
proc ::constcl::symbol? {val} {
    ::if {[info object isa typeof $val ::constcl::Symbol]} {
        return #t
    } elseif {[info object isa typeof [interp alias {} $val] ::constcl::Symbol]} {
        return #t
    } else {
        return #f
    }
}
\end{lstlisting}
\noindent\makebox[\linewidth]{\rule{\linewidth}{0.4pt}}

\textbf{symbol->string}


\texttt{symbol->string} yields a string consisting of the symbol name, usually lower-cased.

\begin{tabular}{ |l l| }
\hline
\multicolumn{2}{|l|}{symbol->string (public)} \\
\hline
sym & a symbol \\
\textit{Returns:} & a string \\
\hline
\end{tabular}

\noindent\makebox[\linewidth]{\rule{\linewidth}{0.4pt}}
\begin{lstlisting}
reg symbol->string ::constcl::symbol->string
 
proc ::constcl::symbol->string {sym} {
    check {symbol? $sym} {SYMBOL expected\n([pn] [$sym show])}
    ::if {![$sym case-constant]} {
        set str [MkString [::string tolower [$sym name]]]
    } else {
        set str [MkString [$sym name]]
    }
    $str mkconstant
    return $str
}
\end{lstlisting}
\noindent\makebox[\linewidth]{\rule{\linewidth}{0.4pt}}

Example:

\noindent\makebox[\linewidth]{\rule{\linewidth}{0.4pt}}
\begin{lstlisting}
(let ((sym 'Foobar)) (symbol->string sym))   ⇒  "foobar"
\end{lstlisting}
\noindent\makebox[\linewidth]{\rule{\linewidth}{0.4pt}}

\textbf{string->symbol}


\texttt{string->symbol} creates a symbol with the name given by the string. The symbol is 'case-constant', i.e. it will not be lower-cased.

\begin{tabular}{ |l l| }
\hline
\multicolumn{2}{|l|}{string->symbol (public)} \\
\hline
str & a string \\
\textit{Returns:} & a symbol \\
\hline
\end{tabular}


Example:

\noindent\makebox[\linewidth]{\rule{\linewidth}{0.4pt}}
\begin{lstlisting}
(define sym (let ((str "Foobar")) (string->symbol str)))
sym                                                        ⇒  Foobar
(symbol->string sym)                                       ⇒  "Foobar"
\end{lstlisting}
\noindent\makebox[\linewidth]{\rule{\linewidth}{0.4pt}}
\noindent\makebox[\linewidth]{\rule{\linewidth}{0.4pt}}
\begin{lstlisting}
reg string->symbol ::constcl::string->symbol
 
proc ::constcl::string->symbol {str} {
    check {string? $str} {STRING expected\n([pn] [$obj show])}
    set sym [MkSymbol [$str value]]
    $sym make-case-constant
    return $sym
}
\end{lstlisting}
\noindent\makebox[\linewidth]{\rule{\linewidth}{0.4pt}}
\subsection{Vectors}
\label{vectors}

Vectors are heterogenous structures of fixed length whose elements are indexed by integers. They are implemented as Tcl lists of Lisp values.


The number of elements that a vector contains (the \_length\_) is set when the vector is created. Elements can be indexed by integers from zero to length minus one.


\textbf{Vector} class

\noindent\makebox[\linewidth]{\rule{\linewidth}{0.4pt}}
\begin{lstlisting}
oo::class create ::constcl::Vector {
    superclass ::constcl::NIL
    variable data constant
    constructor {v} {
        set len [llength $v]
        set vsa [::constcl::vsAlloc $len]
        set idx $vsa
        foreach elt $v {
            lset ::constcl::vectorSpace $idx $elt
            incr idx
        }
        set data [::constcl::cons [::constcl::MkNumber $vsa] [::constcl::MkNumber $len]]
    }
    method length {} {::constcl::cdr $data}
    method ref {k} {
        set k [$k numval]
            ::error "index out of range\n$k"
        }
        lindex [my store] $k
    }
    method store {} {
        set base [[::constcl::car $data] numval]
        set end [expr {[[my length] numval] + $base - 1}]
        lrange $::constcl::vectorSpace $base $end
    }
    method value {} {
        my store
    }
    method set! {k obj} {
        ::if {[my constant]} {
            ::error "vector is constant"
        } else {
            set k [$k numval]
                ::error "index out of range\n$k"
            }
            set base [[::constcl::car $data] numval]
            lset ::constcl::vectorSpace $k+$base $obj
        }
        return [self]
    }
    method fill! {val} {
        ::if {[my constant]} {
            ::error "vector is constant"
        } else {
            set base [[::constcl::car $data] numval]
            set len [[my length] numval]
            for {set idx $base} {$idx < $len+$base} {incr idx} {
                lset ::constcl::vectorSpace $idx $val
            }
        }
        return [self]
    }
    method mkconstant {} {set constant 1}
    method constant {} {set constant}
    method write {handle} { puts -nonewline $handle [my show]}
    method display {} {puts -nonewline [my show]}
    method show {} {format "#(%s)" [join [lmap val [my value] {$val show}] " "]}
}
 
interp alias {} ::constcl::MkVector {} ::constcl::Vector new
\end{lstlisting}
\noindent\makebox[\linewidth]{\rule{\linewidth}{0.4pt}}

\textbf{vector?}

\begin{tabular}{ |l l| }
\hline
\multicolumn{2}{|l|}{vector? (public)} \\
\hline
val & a Lisp value \\
\textit{Returns:} & a boolean \\
\hline
\end{tabular}

\noindent\makebox[\linewidth]{\rule{\linewidth}{0.4pt}}
\begin{lstlisting}
reg vector? ::constcl::vector?
 
proc ::constcl::vector? {val} {
    ::if {[info object isa typeof $val ::constcl::Vector]} {
        return #t
    } elseif {[info object isa typeof [interp alias {} $val] ::constcl::Vector]} {
        return #t
    } else {
        return #f
    }
}
\end{lstlisting}
\noindent\makebox[\linewidth]{\rule{\linewidth}{0.4pt}}

\textbf{make-vector}


\texttt{make-vector} creates a vector with a given length and optionally a fill value. If a fill value isn't given, the empty list will be used.

\begin{tabular}{ |l l| }
\hline
\multicolumn{2}{|l|}{make-vector? (public)} \\
\hline
k & a number \\
?fill? & a Lisp value \\
\textit{Returns:} & a vector \\
\hline
\end{tabular}


Example:

\noindent\makebox[\linewidth]{\rule{\linewidth}{0.4pt}}
\begin{lstlisting}
(let ((k 5)) (make-vector k))                  ⇒  #(() () () () ())
(let ((k 5) (fill #\A)) (make-vector k fill))  ⇒  #(#\A #\A #\A #\A #\A)
\end{lstlisting}
\noindent\makebox[\linewidth]{\rule{\linewidth}{0.4pt}}
\noindent\makebox[\linewidth]{\rule{\linewidth}{0.4pt}}
\begin{lstlisting}
reg make-vector ::constcl::make-vector
 
proc ::constcl::make-vector {k args} {
        set fill #NIL
    } else {
        lassign $args fill
    }
    MkVector [lrepeat [$k numval] $fill]
}
\end{lstlisting}
\noindent\makebox[\linewidth]{\rule{\linewidth}{0.4pt}}

\textbf{vector}


Given a number of Lisp values, \texttt{vector} creates a vector containing them.

\begin{tabular}{ |l l| }
\hline
\multicolumn{2}{|l|}{vector (public)} \\
\hline
args & some Lisp values \\
\textit{Returns:} & a vector \\
\hline
\end{tabular}


Example:

\noindent\makebox[\linewidth]{\rule{\linewidth}{0.4pt}}
\begin{lstlisting}
(vector 'a 'b 'c)   ⇒  #(a b c)
\end{lstlisting}
\noindent\makebox[\linewidth]{\rule{\linewidth}{0.4pt}}
\noindent\makebox[\linewidth]{\rule{\linewidth}{0.4pt}}
\begin{lstlisting}
reg vector ::constcl::vector
 
proc ::constcl::vector {args} {
    MkVector $args
}
\end{lstlisting}
\noindent\makebox[\linewidth]{\rule{\linewidth}{0.4pt}}

\textbf{vector-length}


\texttt{vector-length} returns the length of a vector.

\begin{tabular}{ |l l| }
\hline
\multicolumn{2}{|l|}{vector-length (public)} \\
\hline
vec & a vector \\
\textit{Returns:} & a number \\
\hline
\end{tabular}


Example:

\noindent\makebox[\linewidth]{\rule{\linewidth}{0.4pt}}
\begin{lstlisting}
(vector-length '#(a b c))   ⇒  3
\end{lstlisting}
\noindent\makebox[\linewidth]{\rule{\linewidth}{0.4pt}}
\noindent\makebox[\linewidth]{\rule{\linewidth}{0.4pt}}
\begin{lstlisting}
reg vector-length
 
proc ::constcl::vector-length {vec} {
    check {vector? $vec} {VECTOR expected\n([pn] [$vec show])}
    return [$vec length]
}
\end{lstlisting}
\noindent\makebox[\linewidth]{\rule{\linewidth}{0.4pt}}

\textbf{vector-ref}

\begin{tabular}{ |l l| }
\hline
\multicolumn{2}{|l|}{vector-ref (public)} \\
\hline
vec & a vector \\
k & a number \\
\textit{Returns:} & a Lisp value \\
\hline
\end{tabular}


Example:

\noindent\makebox[\linewidth]{\rule{\linewidth}{0.4pt}}
\begin{lstlisting}
(let ((vec '#(a b c)) (k 1)) (vector-ref vec k))   ⇒  b
\end{lstlisting}
\noindent\makebox[\linewidth]{\rule{\linewidth}{0.4pt}}
\noindent\makebox[\linewidth]{\rule{\linewidth}{0.4pt}}
\begin{lstlisting}
reg vector-ref ::constcl::vector-ref
 
proc ::constcl::vector-ref {vec k} {
    check {vector? $vec} {VECTOR expected\n([pn] [$vec show] [$k show])}
    check {number? $k} {NUMBER expected\n([pn] [$vec show] [$k show])}
    return [$vec ref $k]
}
\end{lstlisting}
\noindent\makebox[\linewidth]{\rule{\linewidth}{0.4pt}}

\textbf{vector-set!}


\texttt{vector-set!}, for a non-constant vector, sets the element at index \_k\_ to \_val\_.

\begin{tabular}{ |l l| }
\hline
\multicolumn{2}{|l|}{vector-set! (public)} \\
\hline
vec & a vector \\
k & a number \\
val & a Lisp value \\
\textit{Returns:} & a vector \\
\hline
\end{tabular}


Example:

\noindent\makebox[\linewidth]{\rule{\linewidth}{0.4pt}}
\begin{lstlisting}
(let ((vec '#(a b c)) (k 1) (val 'x)) (vector-set! vec k val))           ⇒  *error*
(let ((vec (vector 'a 'b 'c)) (k 1) (val 'x)) (vector-set! vec k val))   ⇒  #(a x c)
\end{lstlisting}
\noindent\makebox[\linewidth]{\rule{\linewidth}{0.4pt}}
\noindent\makebox[\linewidth]{\rule{\linewidth}{0.4pt}}
\begin{lstlisting}
reg vector-set! ::constcl::vector-set!
 
proc ::constcl::vector-set! {vec k val} {
    check {vector? $vec} {VECTOR expected\n([pn] [$vec show] [$k show])}
    check {number? $k} {NUMBER expected\n([pn] [$vec show] [$k show])}
    return [$vec set! $k $val]
}
\end{lstlisting}
\noindent\makebox[\linewidth]{\rule{\linewidth}{0.4pt}}

\textbf{vector->list}


\texttt{vector->list} converts a vector value to a Lisp list.

\begin{tabular}{ |l l| }
\hline
\multicolumn{2}{|l|}{vector->list (public)} \\
\hline
vec & a vector \\
\textit{Returns:} & a Lisp list of Lisp values \\
\hline
\end{tabular}


Example:

\noindent\makebox[\linewidth]{\rule{\linewidth}{0.4pt}}
\begin{lstlisting}
(vector->list '#(a b c))   ⇒  (a b c)
\end{lstlisting}
\noindent\makebox[\linewidth]{\rule{\linewidth}{0.4pt}}
\noindent\makebox[\linewidth]{\rule{\linewidth}{0.4pt}}
\begin{lstlisting}
reg vector->list ::constcl::vector->list
 
proc ::constcl::vector->list {vec} {
    list {*}[$vec value]
}
\end{lstlisting}
\noindent\makebox[\linewidth]{\rule{\linewidth}{0.4pt}}

\textbf{list->vector}


\texttt{list->vector} converts a Lisp list value to a vector.

\begin{tabular}{ |l l| }
\hline
\multicolumn{2}{|l|}{list->vector (public)} \\
\hline
list & a Lisp list of Lisp values \\
\textit{Returns:} & a vector \\
\hline
\end{tabular}


Example:

\noindent\makebox[\linewidth]{\rule{\linewidth}{0.4pt}}
\begin{lstlisting}
(list->vector '(1 2 3))   ⇒  #(1 2 3)
\end{lstlisting}
\noindent\makebox[\linewidth]{\rule{\linewidth}{0.4pt}}
\noindent\makebox[\linewidth]{\rule{\linewidth}{0.4pt}}
\begin{lstlisting}
reg list->vector ::constcl::list->vector
 
proc ::constcl::list->vector {list} {
    vector {*}[splitlist $list]
}
\end{lstlisting}
\noindent\makebox[\linewidth]{\rule{\linewidth}{0.4pt}}

\textbf{vector-fill!}


\texttt{vector-fill!} fills a non-constant vector with a given value.

\begin{tabular}{ |l l| }
\hline
\multicolumn{2}{|l|}{vector-fill! (public)} \\
\hline
vec & a vector \\
fill & a Lisp value \\
\textit{Returns:} & a vector \\
\hline
\end{tabular}


Example:

\noindent\makebox[\linewidth]{\rule{\linewidth}{0.4pt}}
\begin{lstlisting}
(define vec (vector 'a 'b 'c))
(vector-fill! vec 'x)             ⇒  #(x x x)
vec                               ⇒  #(x x x)
\end{lstlisting}
\noindent\makebox[\linewidth]{\rule{\linewidth}{0.4pt}}
\noindent\makebox[\linewidth]{\rule{\linewidth}{0.4pt}}
\begin{lstlisting}
reg vector-fill! ::constcl::vector-fill!
 
proc ::constcl::vector-fill! {vec fill} {
    check {vector? $vec} {VECTOR expected\n([pn] [$vec show] [$fill show])}
    $vec fill! $fill
}
\end{lstlisting}
\noindent\makebox[\linewidth]{\rule{\linewidth}{0.4pt}}
\section{Identifier validation}
\label{identifier-validation}

\textbf{idcheckinit}


\textbf{idchecksubs}


\textbf{idcheck}


\textbf{varcheck}


Some routines for checking if a string is a valid identifier. \texttt{idcheckinit} checks the first character, \texttt{idchecksubs} checks the rest. \texttt{idcheck} calls the others and raises errors if they fail. A valid symbol is still an invalid identifier if has the name of some keyword, which \texttt{varcheck} checks, for a set of keywords given in the standard.

\noindent\makebox[\linewidth]{\rule{\linewidth}{0.4pt}}
\begin{lstlisting}
proc ::constcl::idcheckinit {init} {
    ::if {[::string is alpha -strict $init] || $init in {! $ % & * / : < = > ? ^ _ ~}} {
        return true
    } else {
        return false
    }
}
 
proc ::constcl::idchecksubs {subs} {
    foreach c [split $subs {}] {
        ::if {!([::string is alnum -strict $c] || $c in {! $ % & * / : < = > ? ^ _ ~ + - . @})} {
            return false
        }
    }
    return true
}
 
proc ::constcl::idcheck {sym} {
::if {$sym eq {}} {return $sym}
        ![idchecksubs [::string range $sym 1 end]]) && $sym ni {+ - ...}} {
        ::error "Identifier expected ($sym)"
    }
    set sym
}
 
proc ::constcl::varcheck {sym} {
    ::if {$sym in {else => define unquote unquote-splicing quote lambda if set! begin
        cond and or case let let* letrec do delay quasiquote}} {
            ::error "Macro name can't be used as a variable: $sym"
    }
    return $sym
}
\end{lstlisting}
\noindent\makebox[\linewidth]{\rule{\linewidth}{0.4pt}}
\section{Initialization}
\label{initialization}

Initialize the memory space for vector contents.

\noindent\makebox[\linewidth]{\rule{\linewidth}{0.4pt}}
\begin{lstlisting}
 
 
proc ::constcl::vsAlloc {num} {
    # TODO calculate free space
    set va $::constcl::vectorAssign
    incr ::constcl::vectorAssign $num
    return $va
}
\end{lstlisting}
\noindent\makebox[\linewidth]{\rule{\linewidth}{0.4pt}}
\noindent\makebox[\linewidth]{\rule{\linewidth}{0.4pt}}
\begin{lstlisting}
\end{lstlisting}
\noindent\makebox[\linewidth]{\rule{\linewidth}{0.4pt}}

Pre-make a set of constants (mostly symbols but also e.g. \#NIL, \#t, and \#f) and give them aliases for use in source text.

\noindent\makebox[\linewidth]{\rule{\linewidth}{0.4pt}}
\begin{lstlisting}
interp alias {} #NIL {} [::constcl::NIL new]
 
interp alias {} #t {} [::constcl::MkBoolean #t]
 
interp alias {} #f {} [::constcl::MkBoolean #f]
 
interp alias {} #-1 {} [::constcl::MkNumber -1]
 
 
interp alias {} #1 {} [::constcl::MkNumber 1]
 
interp alias {} #B {} [::constcl::MkSymbol begin]
 
interp alias {} #I {} [::constcl::MkSymbol if]
 
interp alias {} #L {} [::constcl::MkSymbol let]
 
interp alias {} #Q {} [::constcl::MkSymbol quote]
 
interp alias {} #U {} [::constcl::MkSymbol unquote]
 
interp alias {} #S {} [::constcl::MkSymbol set!]
 
interp alias {} #x {} [::constcl::MkSymbol x]
 
interp alias {} #y {} [::constcl::MkSymbol y]
 
interp alias {} #λ {} [::constcl::MkSymbol lambda]
 
interp alias {} #+ {} [::constcl::MkSymbol +]
 
interp alias {} #- {} [::constcl::MkSymbol -]
 
interp alias {} #NONE {} [::constcl::None new]
 
interp alias {} #UNSP {} [::constcl::Unspecific new]
 
interp alias {} #UNDF {} [::constcl::Undefined new]
 
interp alias {} #EOF {} [::constcl::EndOfFile new]
 
\end{lstlisting}
\noindent\makebox[\linewidth]{\rule{\linewidth}{0.4pt}}

Initialize the definition register with the queen of numbers (or at least a double floating point approximation).

\noindent\makebox[\linewidth]{\rule{\linewidth}{0.4pt}}
\begin{lstlisting}
dict set ::constcl::defreg pi [::constcl::MkNumber 3.1415926535897931]
\end{lstlisting}
\noindent\makebox[\linewidth]{\rule{\linewidth}{0.4pt}}

In this interpreter, \texttt{nil} refers to the empty list.

\noindent\makebox[\linewidth]{\rule{\linewidth}{0.4pt}}
\begin{lstlisting}
reg nil #NIL
\end{lstlisting}
\noindent\makebox[\linewidth]{\rule{\linewidth}{0.4pt}}

\textbf{atom?}


\texttt{atom?} recognizes an atom by checking for membership in one of the atomic types.

\begin{tabular}{ |l l| }
\hline
\multicolumn{2}{|l|}{atom? (public)} \\
\hline
val & a Lisp value \\
\textit{Returns:} & a boolean \\
\hline
\end{tabular}

\noindent\makebox[\linewidth]{\rule{\linewidth}{0.4pt}}
\begin{lstlisting}
reg atom? ::constcl::atom?
 
proc ::constcl::atom? {val} {
    ::if {[symbol? $val] ne "#f" || [number? $val] ne "#f" || [string? $val] ne "#f" ||
        [char? $val] ne "#f" || [boolean? $val] ne "#f" || [vector? $val] ne "#f" ||
        [port? $val] ne "#f"} {
        return #t
    } else {
        return #f
    }
}
\end{lstlisting}
\noindent\makebox[\linewidth]{\rule{\linewidth}{0.4pt}}
\section{The REPL}
\label{the-repl}

loop that repeatedly \_reads\_ a Scheme source string from the user through the command \texttt{::constcl::input} (breaking the loop if given an empty line) and \texttt{::constcl::parse}, \_evaluates\_ it using \texttt{::constcl::eval}, and \_prints\_ using \texttt{::constcl::write}.


\textbf{input}


\texttt{input} is modelled after the Python 3 function. It displays a prompt and reads a string.

\noindent\makebox[\linewidth]{\rule{\linewidth}{0.4pt}}
\begin{lstlisting}
proc ::constcl::input {prompt} {
    puts -nonewline $prompt
    flush stdout
    set buf [gets stdin]
    set openpars [regexp -all -inline {\(} $buf]
    set clsepars [regexp -all -inline {\)} $buf]
    set openbrak [regexp -all -inline {\[} $buf]
    set clsebrak [regexp -all -inline {\]} $buf]
    while {[llength $openpars] > [llength $clsepars] || [llength $openbrak] > [llength $clsebrak]} {
        ::append buf [gets stdin]
        set openpars [regexp -all -inline {\(} $buf]
        set clsepars [regexp -all -inline {\)} $buf]
        set openbrak [regexp -all -inline {\[} $buf]
        set clsebrak [regexp -all -inline {\]} $buf]
    }
    return $buf
}
\end{lstlisting}
\noindent\makebox[\linewidth]{\rule{\linewidth}{0.4pt}}

\textbf{repl}


\texttt{repl} puts the loop in the read-eval-print loop. It repeats prompting for a string until given a blank input. Given non-blank input, it parses and evaluates the string, printing the resulting value.

\noindent\makebox[\linewidth]{\rule{\linewidth}{0.4pt}}
\begin{lstlisting}
proc ::constcl::repl {{prompt "ConsTcl> "}} {
    set str [input $prompt]
    while {$str ne ""} {
        write [eval [parse $str]]
        set str [input $prompt]
    }
}
\end{lstlisting}
\noindent\makebox[\linewidth]{\rule{\linewidth}{0.4pt}}
\section{Environment class and objects}
\label{environment-class-and-objects}

The class for environments is called \texttt{Environment}. It is mostly a wrapper around a dictionary, with the added finesse of keeping a link to the outer environment (starting a chain that goes all the way to the global environment and then stops at the null environment) which can be traversed by the find method to find which innermost environment a given symbol is bound in.


The long and complex constructor is to accommodate the variations of Scheme parameter lists, which can be empty, a proper list, a symbol, or a dotted list.


\textbf{Environment} class

\noindent\makebox[\linewidth]{\rule{\linewidth}{0.4pt}}
\begin{lstlisting}
catch { ::constcl::Environment destroy }
 
oo::class create ::constcl::Environment {
    variable bindings outer_env
    constructor {syms vals {outer {}}} {
        set bindings [dict create]
        if {[::constcl::null? $syms] eq "#t"} {
            if {[llength $vals]} { error "too many arguments" }
        } elseif {[::constcl::list? $syms] eq "#t"} {
            set syms [::constcl::splitlist $syms]
            set symsn [llength $syms]
            set valsn [llength $vals]
            if {$symsn != $valsn} {
                error "wrong number of arguments, $valsn instead of $symsn"
            }
            foreach sym $syms val $vals {
                my set $sym $val
            }
        } elseif {[::constcl::symbol? $syms] eq "#t"} {
            my set $syms [::constcl::list {*}$vals]
        } else {
            while true {
                if {[llength $vals] < 1} { error "too few arguments" }
                set vals [lrange $vals 1 end]
                if {[::constcl::symbol? [::constcl::cdr $syms]] eq "#t"} {
                    my set [::constcl::cdr $syms] [::constcl::list {*}$vals]
                    set vals {}
                    break
                } else {
                    set syms [::constcl::cdr $syms]
                }
            }
        }
        set outer_env $outer
    }
    method find {sym} {
        if {$sym in [dict keys $bindings]} {
            self
        } else {
            $outer_env find $sym
        }
    }
    method get {sym} {
        dict get $bindings $sym
    }
    method set {sym val} {
        dict set bindings $sym $val
    }
    method setstr {str val} {
        dict set bindings [::constcl::MkSymbol $str] $val
    }
}
\end{lstlisting}
\noindent\makebox[\linewidth]{\rule{\linewidth}{0.4pt}}

On startup, two \texttt{Environment} objects called \texttt{null\_env} (the null environment, not the same as \texttt{null-environment} in Scheme) and \texttt{global\_env} (the global environment) are created.


Make \texttt{null\_env} empty and unresponsive: this is where searches for unbound symbols end up.

\noindent\makebox[\linewidth]{\rule{\linewidth}{0.4pt}}
\begin{lstlisting}
::constcl::Environment create ::constcl::null_env #NIL {}
 
oo::objdefine ::constcl::null_env {
    method find {sym} {self}
    method get {sym} {::error "Unbound variable: [$sym name]"}
    method set {sym val} {::error "Unbound variable: [$sym name]"}
}
\end{lstlisting}
\noindent\makebox[\linewidth]{\rule{\linewidth}{0.4pt}}

Meanwhile, \texttt{global\_env} is populated with all the definitions from the definitions register, \texttt{defreg}. This is where top level evaluation happens.

\noindent\makebox[\linewidth]{\rule{\linewidth}{0.4pt}}
\begin{lstlisting}
namespace eval ::constcl {
    set keys [list {*}[lmap k [dict keys $defreg] {MkSymbol $k}]]
    set vals [dict values $defreg]
    Environment create global_env $keys $vals ::constcl::null_env
}
\end{lstlisting}
\noindent\makebox[\linewidth]{\rule{\linewidth}{0.4pt}}

Thereafter, each time a user-defined procedure is called, a new \texttt{Environment} object is created to hold the bindings introduced by the call, and also a link to the outer environment (the one closed over when the procedure was defined).


Load the Scheme base.

\noindent\makebox[\linewidth]{\rule{\linewidth}{0.4pt}}
\begin{lstlisting}
::constcl::load schemebase.lsp
\end{lstlisting}
\noindent\makebox[\linewidth]{\rule{\linewidth}{0.4pt}}
\subsubsection{Lexical scoping}
\label{lexical-scoping}

Example:

\noindent\makebox[\linewidth]{\rule{\linewidth}{0.4pt}}
\begin{lstlisting}
ConsTcl> (define (circle-area r) (* pi (* r r)))
314.1592653589793
\end{lstlisting}
\noindent\makebox[\linewidth]{\rule{\linewidth}{0.4pt}}

During a call to the procedure \texttt{circle-area}, the symbol \texttt{r} is bound to the possibly clobbering an earlier definition of \texttt{r}. The solution is to use separate (but linked) environments, making \texttt{r}'s binding a *local variable\footnote{See \texttt{https://en.wikipedia.org/wiki/Local\_variable)* in its own environment, which the procedure will be evaluated in. The symbols \texttt{*} and \texttt{pi} will still be available through the local environment's link to the outer global environment. This is all part of *lexical scoping[\#](https://en.wikipedia.org/wiki/Scope\_(computer\_science)\#Lexical\_scope}}*.


In the first image, we see the global environment before we call \texttt{circle-area} (and also the empty null environment which the global environment links to):


![A global environment](/images/env1.png)


During the call. Note how the global \texttt{r} is shadowed by the local one, and how the local environment links to the global one to find \texttt{*} and \texttt{pi}.


![A local environment shadows the global](/images/env2.png)


After the call, we are back to the first state again.


![A global environment](/images/env1.png)


if no {

\section{Lookup tables}
\label{lookup-tables}

Lisp languages have two simple variants of key/value lookup tables: property lists (plists) and association lists (alists).


A property list is simply a list where every odd-numbered item (starting from 1) is a key and every even-numbered item is a value. Example:

\noindent\makebox[\linewidth]{\rule{\linewidth}{0.4pt}}
\begin{lstlisting}
'(a 1 b 2 c 3 d 4 e 5)
\end{lstlisting}
\noindent\makebox[\linewidth]{\rule{\linewidth}{0.4pt}}

Values can be retrieved in a two-step process:

\noindent\makebox[\linewidth]{\rule{\linewidth}{0.4pt}}
\begin{lstlisting}
> (define plist (list 'a 1 'b 2 'c 3 'd 4 'e 5))
> (define v '())
> (set! v (memq 'c plist))
(c 3 d 4 e 5)
> (set! v (cadr v))
3
\end{lstlisting}
\noindent\makebox[\linewidth]{\rule{\linewidth}{0.4pt}}

If a key doesn't occur in the plist, \texttt{memq} returns \texttt{\#f}.


Alternatively, ConsTcl users can use \texttt{get} to access the value in one step.

\noindent\makebox[\linewidth]{\rule{\linewidth}{0.4pt}}
\begin{lstlisting}
> (get plist 'c)
3
\end{lstlisting}
\noindent\makebox[\linewidth]{\rule{\linewidth}{0.4pt}}

\texttt{get} returns \texttt{\#f} if the key isn't present in the plist.


Values can be added with a single statement:

\noindent\makebox[\linewidth]{\rule{\linewidth}{0.4pt}}
\begin{lstlisting}
> (set! plist (append '(f 6) plist))
(f 6 a 1 b 2 c 3 d 4 e 5)
\end{lstlisting}
\noindent\makebox[\linewidth]{\rule{\linewidth}{0.4pt}}

or with the \texttt{put!} macro, which can both update existing values and add new ones:

\noindent\makebox[\linewidth]{\rule{\linewidth}{0.4pt}}
\begin{lstlisting}
> (put! plist 'c 9)
(f 6 a 1 b 2 c 9 d 4 e 5)
> (put! plist 'g 7)
(g 7 f 6 a 1 b 2 c 9 d 4 e 5)
\end{lstlisting}
\noindent\makebox[\linewidth]{\rule{\linewidth}{0.4pt}}

To get rid of a key/value pair, the simplest way is to add a masking pair:

\noindent\makebox[\linewidth]{\rule{\linewidth}{0.4pt}}
\begin{lstlisting}
> (set! plist (append '(d #f) plist))
(d #f g 7 f 6 a 1 b 2 c 3 d 4 e 5)
\end{lstlisting}
\noindent\makebox[\linewidth]{\rule{\linewidth}{0.4pt}}

But instead, one can use the \texttt{del!} macro:

\noindent\makebox[\linewidth]{\rule{\linewidth}{0.4pt}}
\begin{lstlisting}
> plist
(g 7 f 6 a 1 b 2 c 9 d 4 e 5)
> (del! plist 'd)
(g 7 f 6 a 1 b 2 c 9 e 5)
\end{lstlisting}
\noindent\makebox[\linewidth]{\rule{\linewidth}{0.4pt}}

An alist is a list where the items are pairs, with the key as the \texttt{car} and the value as the \texttt{cdr}. Example:

\noindent\makebox[\linewidth]{\rule{\linewidth}{0.4pt}}
\begin{lstlisting}
> (define alist (list (cons 'a 1) (cons 'b 2) (cons 'c 3) (cons 'd 4)))
> alist
((a . 1) (b . 2) (c . 3) (d . 4))
\end{lstlisting}
\noindent\makebox[\linewidth]{\rule{\linewidth}{0.4pt}}

An alist can also be created from scratch using the \texttt{pairlis} procedure:

\noindent\makebox[\linewidth]{\rule{\linewidth}{0.4pt}}
\begin{lstlisting}
> (define alist (pairlis '(a b c) '(1 2 3)))
((a . 1) (b . 2) (c . 3))
\end{lstlisting}
\noindent\makebox[\linewidth]{\rule{\linewidth}{0.4pt}}

The procedure \texttt{assq} retrieves one pair based on the key:

\noindent\makebox[\linewidth]{\rule{\linewidth}{0.4pt}}
\begin{lstlisting}
> (assq 'a alist)
(a . 1)
> (cdr (assq 'a alist))
1
> (assq 'x alist)
#f
\end{lstlisting}
\noindent\makebox[\linewidth]{\rule{\linewidth}{0.4pt}}

As an alternative, the \texttt{get-alist} procedure fetches the value directly, or \#f for a missing item:

\noindent\makebox[\linewidth]{\rule{\linewidth}{0.4pt}}
\begin{lstlisting}
> (get-alist 'a)
1
> (get-alist 'x)
#f
\end{lstlisting}
\noindent\makebox[\linewidth]{\rule{\linewidth}{0.4pt}}

We can add another item to the alist with the \texttt{push!} macro:

\noindent\makebox[\linewidth]{\rule{\linewidth}{0.4pt}}
\begin{lstlisting}
> (push! (cons 'e 5) alist)
((e . 5) (a . 1) (b . 2) (c . 3) (d . 4))
\end{lstlisting}
\noindent\makebox[\linewidth]{\rule{\linewidth}{0.4pt}}

The \texttt{set-alist!} procedure can be used to update a value (it returns the alist unchanged if the key isn't present):

\noindent\makebox[\linewidth]{\rule{\linewidth}{0.4pt}}
\begin{lstlisting}
> alist
((a . 1) (b . 2) (c . 3) (d . 4))
> (set-alist! alist 'b 7)
((a . 1) (b . 7) (c . 3) (d . 4))
\end{lstlisting}
\noindent\makebox[\linewidth]{\rule{\linewidth}{0.4pt}}

}

\section{A Scheme base}
\label{a-scheme-base}
\noindent\makebox[\linewidth]{\rule{\linewidth}{0.4pt}}
\begin{lstlisting}
 
; An assortment of procedures to supplement the builtins.
 
(define (get plist key)
  (let ((v (memq key plist)))
    (if v
      (cadr v)
      #f)))
 
(define (list-find-key lst key)
 
(define (lfk lst key count)
  (if (null? lst)
    -1
    (if (eq? (car lst) key)
      count
      (lfk (cddr lst) key (+ count 2)))))
 
(define (list-set! lst idx val)
  (if (zero? idx)
    (set-car! lst val)
    (list-set! (cdr lst) (- idx 1) val)))
 
(define (delete! lst key)
  (let ((idx (list-find-key lst key)))
      lst
        (set! lst (cddr lst))
        (let ((node-before (delete-seek lst (- idx 1)))
              (node-after (delete-seek lst (+ idx 2))))
          (set-cdr! node-before node-after))))
    lst))
 
(define (delete-seek lst idx)
  (if (zero? idx)
    lst
    (delete-seek (cdr lst) (- idx 1))))
 
(define (get-alist lst key)
  (let ((item (assq key lst)))
    (if item
      (cdr item)
      #f)))
 
(define (pairlis a b)
  (if (null? a)
    ()
    (cons (cons (car a) (car b)) (pairlis (cdr a) (cdr b)))))
 
(define (set-alist! lst key val)
  (let ((item (assq key lst)))
    (if item
      (begin (set-cdr! item val) lst)
      lst)))
 
(define (fact n)
  (if (<= n 1)
    1
    (* n (fact (- n 1)))))
 
\end{lstlisting}
\noindent\makebox[\linewidth]{\rule{\linewidth}{0.4pt}}

\end{document}
